% ===================================================================================== %
%                                  Nomenclature Module                                  %
% ===================================================================================== %


% =========================================================================== %
%                       Entry Independent Commands                            %
% =========================================================================== %
\DefineNewLength{\UWMad@Widest}         {0pt}        % Will have the value of the widest symbol for spacing
\DefineNewLength{\UWMad@WidestTest}     {0pt}        % Used to test for the widest symbol
\DefineNewLength{\NomenclatureTitleSkip}{0em}        % Used for title spacing
\DefineNewLength{\NomenclaturePrintSkip}{1em}        % Used for spacing after nomenclature is printed
\DefineNewLength{\EntryMarginLeft}      {1em}
\DefineNewLength{\EntryMarginBottom}    {0.25em}

\DefineNewCounter{NomenclatureMaximumNesting}   {0}  % Number of Subnomenclatures
%\DefineNewCounter{NomenclatureSectioningLevel}{0}  % Sectioning Level for the top-level Nomenclature environment
\newcount\NomenclatureNestingLevel %
\newcount\NomenclatureSectioningLevel %

\newif\ifMakeNomenclatureStarred\MakeNomenclatureStarredtrue
\let\MakeNomenclatureStarred\MakeNomenclatureStarredtrue
\let\MakeNomenclatureNotStarred\MakeNomenclatureStarredfalse

\newcommand*{\UWMad@UpdateWidest}[1]{
    \settowidth{\UWMad@WidestTest}{#1}
    \ifdim\UWMad@Widest<\UWMad@WidestTest
        \setlength{\UWMad@Widest}{\UWMad@WidestTest}
    \fi
}
%
%
\newcommand{\TheNomenclatureName}{Nomenclature}
\newcommand{\NomenclatureName}[1]
    {\renewcommand{\TheNomenclatureName}{#1}}
%
\newcommand{\TheNomenclatureGroupName}{}
\newcommand{\NomenclatureGroupName}[1]
    {\renewcommand{\TheNomenclatureGroupName}{#1}}
%
%
\newcommand{\NomenclatureNameStyle}
    {\ifdefempty{\TheNomenclatureGroupName}%
        {}%
        {\ifMakeNomenclatureStarred%
            \csname\TheCurrentSectioningCommand\endcsname*%
                {\TheNomenclatureGroupName}%
            \UWMad@FrontMatterRegister%
                [\TheCurrentSectioningCommand]%
                {\TheNomenclatureGroupName}
         \else%
            \csname\TheCurrentSectioningCommand\endcsname%
                {\TheNomenclatureGroupName}%
         \fi}}
%
%
\newcommand{\TheCurrentSectioningCommand}{}
\newcommand{\CurrentSectioningCommand}[1]
    {\renewcommand{\TheCurrentSectioningCommand}{#1}}


% =========================================================================== %
%                       Entry Dependent Commands                              %
% =========================================================================== %
\DefineNewLength{\SymbolWidth}          {0em}
\DefineNewLength{\DescriptionWidth}     {0em}
\DefineNewLength{\SymbolDescriptionPad} {0.75em}


\newcommand{\UpdateAddCommand}%
    {}
%
\newcommand{\TheSymbolArrayName}{}
\newcommand{\TheDescriptionArrayName}{}
\newcommand{\SymbolArrayName}[1]     {\renewcommand{\TheSymbolArrayName}{#1}}
\newcommand{\DescriptionArrayName}[1]{\renewcommand{\TheDescriptionArrayName}{#1}}

\newcommand{\SetTheWidths}{%
    % Define Symbol minipage width
    \setlength  {\SymbolWidth}{\UWMad@Widest}%
    \addtolength{\SymbolWidth}{\SymbolDescriptionPad}%
    %
    % Define Description minipage width
    \setlength  {\DescriptionWidth}{\textwidth}%
    \addtolength{\DescriptionWidth}{-\SymbolWidth}%
    \addtolength{\DescriptionWidth}{-\EntryMarginLeft}%
    \setlength  {\DescriptionWidth}{0.99\DescriptionWidth}%
}

\newcommand{\NomenclatureEntry}[2]{%
    \setstretch{1.1}%
    \hspace{\EntryMarginLeft}%
    \begin{minipage}[t]{\SymbolWidth}%
        #1%
    \end{minipage}%
    \begin{minipage}[t]{\DescriptionWidth}%
        #2%
    \end{minipage}%
    \vskip\EntryMarginBottom%
}

\newcommand{\NomenclatureArrayIterator}{
    \ForEach[1]{\TheSymbolArrayName}%
        {\NomenclatureEntry%
            {\ArrayGet
                {\TheSymbolArrayName}
                {ForLoopCounter}}%
            {\ArrayGet
                {\TheDescriptionArrayName}
                {ForLoopCounter}}}%
}

\newcommand{\NomenclaturePrint}{%
    \NomenclatureNameStyle%         Print the name of the current group
    \ifnum\ArrayValueCount{\TheSymbolArrayName}>0%
        \SetTheWidths%              Set the widths of the minipages
        \NomenclatureArrayIterator% Iterate over the entries
    \fi%
}


% =========================================================================== %
%           Nomenclature Environment Initializer and Finalizer                %
% =========================================================================== %
\newcommand{\NomenclatureInitializer}{
    \SymbolArrayName
        {NomenclatureSymbol}
    \DescriptionArrayName
        {NomenclatureDescription}
    \ArrayMake
        {\TheSymbolArrayName}
    \ArrayMake
        {\TheDescriptionArrayName}
    \newcommand{\Add}[2]
        {\ArrayPush{\TheSymbolArrayName}{##1}
         \ArrayPush{\TheDescriptionArrayName}{##2}
         \UWMad@UpdateWidest{##1}}
    \let\item\Add
    \let\Item\Add
    \let\Entry\Add
}

\newcommand{\NomenclatureFinalizer}[1]{%
    \NomenclaturePrint%                     Print the current group's symbols and descriptions
    #1{\TheSymbolArrayName}%                Garbage Collection: \ArrayDelete or \ArrayReset
    #1{\TheDescriptionArrayName}%           Garbage Collection: \ArrayDelete or \ArrayReset
    \global\setlength{\UWMad@Widest}{0pt}%  Set the widest symbol to 0 for future lists
}


% =========================================================================== %
%          Subnomenclature Environment Initializer and Finalizer              %
% =========================================================================== %

% Nomenclature environment ----------------------------------------------------
\newenvironment{Nomenclature}[1][chapter]%
    {%
        \stepcounter{NomenclatureMaximumNesting} % This is the main Group
        \CurrentSectioningCommand{#1}
        \NomenclatureSectioningLevel=\SectionToLevel{#1}
        \NomenclatureNestingLevel=1
        \NomenclatureGroupName{\TheNomenclatureName}
        \NomenclatureInitializer}
    {%
        \ifnum\theNomenclatureMaximumNesting>1%
            \NomenclatureGroupName{}%
        \fi%
        \NomenclatureFinalizer{\ArrayDelete}%
        \setcounter{NomenclatureMaximumNesting}{0}}


\newenvironment{Group}[1]% #1 = Group Name
    {%
        \NomenclatureFinalizer{\ArrayReset}%
        %
        \ifnum \NomenclatureNestingLevel
        \advance \NomenclatureNestingLevel 1 %
        \ifnum \theNomenclatureMaximumNesting < \NomenclatureNestingLevel
            \stepcounter{NomenclatureMaximumNesting}
        \fi
        %
        \ifstrempty{#1}%
            {\NomenclatureGroupName{}}%
            {\NomenclatureGroupName{#1}%
             \advance \NomenclatureSectioningLevel 1}%
        %
        \CurrentSectioningCommand%
            {\LevelToSection%
                {\number\NomenclatureSectioningLevel}}}%
    {%
        \UWMad@ClassMessage{\number\NomenclatureNestingLevel\space\space\theNomenclatureMaximumNesting}%
        \ifnum \NomenclatureNestingLevel<\theNomenclatureMaximumNesting%
            \NomenclatureGroupName{}%
        \fi%
        \NomenclatureFinalizer{\ArrayReset}%
        \ifnum \NomenclatureNestingLevel=2%
            \setcounter{NomenclatureMaximumNesting}{1}%
        \fi%
        }%


