% ===================================================================================== %
%                                  Nomen Module                                  %
% ===================================================================================== %


% =========================================================================== %
%                       Entry Independent Commands                            %
% =========================================================================== %
\DefineNewLength{\UWMad@Widest}         {0pt}        % Will have the value of the widest symbol for spacing
\DefineNewLength{\UWMad@WidestTest}     {0pt}        % Used to test for the widest symbol
\DefineNewLength{\NomenTitleSkip}       {0em}        % Used for title spacing
\DefineNewLength{\NomenPrintSkip}       {1em}        % Used for spacing after Nomen is printed
\DefineNewLength{\EntryMarginLeft}      {1em}
\DefineNewLength{\EntryMarginBottom}    {0.25em}

\DefineNewCounter{NomenMaxNest}     {0}  % Number of SubNomens
\DefineNewCounter{GroupSiblingCount}{0}  % Number of SubNomens
\DefineNewLocalCounter{NomenNestLevel}{0}
\DefineNewLocalCounter{NomenSectionLevel}{0}

\GlobalNewIf{UWMad@MakeNomenclatureStarred}
\let\MakeNomenclatureStarred\UWMad@MakeNomenclatureStarredtrue
\let\MakeNomenclatureNotStarred\UWMad@MakeNomenclatureStarredfalse

\newcommand*{\UWMad@UpdateWidest}[1]{%
    \settowidth{\UWMad@WidestTest}{#1}%
    \ifdim\UWMad@Widest<\UWMad@WidestTest%
        \setlength{\UWMad@Widest}{\UWMad@WidestTest}%
    \fi%
}
%
%
\newcommand{\TheNomenclatureName}{Nomenclature}
\newcommand{\NomenclatureName}[1]
    {\renewcommand{\TheNomenclatureName}{#1}}
%
\newcommand{\TheNomenGroupName}{}
\newcommand{\NomenGroupName}[1]{
    \renewcommand{\TheNomenGroupName}{#1}
}
%
%
\newcommand{\NomenclatureNameStyle}{
    \ifdefempty{\TheNomenGroupName}%
        {}%
        {\ifUWMad@MakeNomenclatureStarred%
            \csname\TheCurrentSectioningCommand\endcsname*%
                {\TheNomenGroupName}%
            \UWMad@FrontMatterRegister%
                [\TheCurrentSectioningCommand]%
                {\TheNomenGroupName}
         \else%
            \csname\TheCurrentSectioningCommand\endcsname%
                {\TheNomenGroupName}%
         \fi}%
}
%
%
\newcommand{\TheCurrentSectioningCommand}{}
\newcommand{\CurrentSectioningCommand}[1]{
    \renewcommand{\TheCurrentSectioningCommand}{#1}
}


% =========================================================================== %
%                       Entry Dependent Commands                              %
% =========================================================================== %
\DefineNewLength{\SymbolWidth}          {0em}
\DefineNewLength{\DescriptionWidth}     {0em}
\DefineNewLength{\SymbolDescriptionPad} {0.75em}


\newcommand{\UpdateAddCommand}{}
%
\newcommand{\TheSymbolArrayName}{}
\newcommand{\TheDescriptionArrayName}{}
\newcommand{\SymbolArrayName}[1]     {\renewcommand{\TheSymbolArrayName}{#1}}
\newcommand{\DescriptionArrayName}[1]{\renewcommand{\TheDescriptionArrayName}{#1}}

\newcommand{\SetTheWidths}{%
    % Define Symbol minipage width
    \setlength  {\SymbolWidth}{\UWMad@Widest}%
    \addtolength{\SymbolWidth}{\SymbolDescriptionPad}%
    %
    % Define Description minipage width
    \setlength  {\DescriptionWidth}{\textwidth}%
    \addtolength{\DescriptionWidth}{-\SymbolWidth}%
    \addtolength{\DescriptionWidth}{-\EntryMarginLeft}%
    \setlength  {\DescriptionWidth}{0.99\DescriptionWidth}%
}

\newcommand{\NomenEntry}[2]{%
    \setstretch{1.1}%
    \hspace{\EntryMarginLeft}%
    \begin{minipage}[t]{\SymbolWidth}%
        #1%
    \end{minipage}%
    \begin{minipage}[t]{\DescriptionWidth}%
        #2%
    \end{minipage}%
    \vskip\EntryMarginBottom%
}

\newcommand{\NomenArrayIterator}{
    \ForEach[1]{\TheSymbolArrayName}%
        {\NomenEntry%
            {\ArrayGet
                {\TheSymbolArrayName}
                {ForLoopCounter}}%
            {\ArrayGet
                {\TheDescriptionArrayName}
                {ForLoopCounter}}}%
}

\newcommand{\NomenPrint}{%
    \NomenclatureNameStyle%         Print the name of the current group
    \ifnum\ArrayValueCount{\TheSymbolArrayName}>0%
        \SetTheWidths%              Set the widths of the minipages
        \NomenArrayIterator% Iterate over the entries
    \fi%
}


% =========================================================================== %
%                 Nomen Environment Initializer and Finalizer                 %
% =========================================================================== %
\newcommand{\NomenInitializer}{
    \SymbolArrayName
        {NomenSymbol}
    \DescriptionArrayName
        {NomenDescription}
    \ArrayMake
        {\TheSymbolArrayName}
    \ArrayMake
        {\TheDescriptionArrayName}
    \newcommand{\Add}[2]
        {\ArrayPush{\TheSymbolArrayName}{##1}
         \ArrayPush{\TheDescriptionArrayName}{##2}
         \UWMad@UpdateWidest{##1}}
    \let\item\Add
    \let\Item\Add
    \let\Entry\Add
}

\newcommand{\NomenFinalizer}[1]{%
    \NomenPrint%                     Print the current group's symbols and descriptions
    #1{\TheSymbolArrayName}%                Garbage Collection: \ArrayDelete or \ArrayReset
    #1{\TheDescriptionArrayName}%           Garbage Collection: \ArrayDelete or \ArrayReset
    \global\setlength{\UWMad@Widest}{0pt}%  Set the widest symbol to 0 for future lists
}


% =========================================================================== %
%          SubNomen Environment Initializer and Finalizer              %
% =========================================================================== %

% Nomen environment ----------------------------------------------------
\newenvironment{NewNomenclature}[1][chapter]{%
    \SetCounter{NomenNestLevel}{0}% zeroth-ly nested
    
    \SymbolArrayName{NomenSymbol\CounterValue{NomenNestLevel}}
    \DescriptionArrayName{NomenDescription\CounterValue{NomenNestLevel}}
    \MakeCommand{ThisNomenArrayName}{Nomenclature\CounterValue{NomenNestLevel}}
    \ArrayMake{\ThisNomenArrayName}
    
    
    \ArrayCommandPush{\ThisNomenArrayName}{}
    
    %\stepcounter{NomenMaxNest} % This is the main Group
    %\CurrentSectioningCommand{#1}
    %\NomenSectionLevel=\SectionToLevel{#1}
    %\NomenNestLevel=1
    %\NomenGroupName{\TheNomenclatureName}
    %\NomenInitializer
    %\NomenclatureNameStyle
    %\NomenGroupName{}
}{%
    \NomenFinalizer{\ArrayDelete}%
    \setcounter{NomenMaxNest}{0}
}


% Nomen environment ----------------------------------------------------
\newenvironment{Nomenclature}[1][chapter]{%
    \stepcounter{NomenMaxNest} % This is the main Group
    \CurrentSectioningCommand{#1}
    \NomenSectionLevel=\SectionToLevel{#1}
    \NomenNestLevel=1
    \NomenGroupName{\TheNomenclatureName}
    \NomenInitializer
    \NomenclatureNameStyle
    \NomenGroupName{}
}{%
    \NomenFinalizer{\ArrayDelete}%
    \setcounter{NomenMaxNest}{0}
}



\newcommand{\GroupNestHandler}{
    \IfCommandExists{UWMad@Group\theGroupSiblingCount\the\NomenNestLevel}
        {\stepcounter{GroupSiblingCount}
            \GroupNestHandler}
        {\csdef%
            {UWMad@GroupNumber\the\NomenNestLevel}%
            {UWMad@GroupNumber\the\NomenNestLevel}
         \ifnum \NomenNestLevel > \theNomenMaxNest
            \stepcounter{NomenMaxNest}
        \fi
        \advance \NomenNestLevel 1}
        \setcounter{GroupSiblingCount}{0}
}

\newenvironment{Group}[1]{% #1 = Group Name
    \GroupNestHandler%
    \NomenFinalizer{\ArrayReset}%
    %
    \ifstrempty{#1}%
        {\NomenGroupName{}}%
        {\NomenGroupName{#1}%
         \advance \NomenSectionLevel 1}%
    %
    \CurrentSectioningCommand%
        {\LevelToSection%
            {\number\NomenSectionLevel}}
}{%
    \ifnum \NomenNestLevel<\theNomenMaxNest%
        \NomenGroupName{}%
    \fi%
    \NomenFinalizer{\ArrayReset}%
    \ifnum \NomenNestLevel=2%
        \setcounter{NomenMaxNest}{1}%
    \fi
}%


