% =============================================================================================== %
%                        WisconsinThesis Class: Programming Module                                %
% =============================================================================================== %


% This is a convenience command used to define a new length and set its initial value.
% A thin abstraction that declares and intializes dimensions.
\newcommand{\DefineNewLength}[2]{
    \newlength{#1}
    \setlength{#1}{#2}
}

% This is a convenience command used to define a new counter and set its initial value.
% A thin abstraction that declares and intializes lengths.
\newcommand{\DefineNewCounter}[2]{
    \newcounter{#1}
    \setcounter{#1}{#2}
}

% If-switch changes only have block level scope by default.  That is, a \newif that is switched within a block
% reverts to its pre-block state upon exit.  This command creates an if-switch set that has global scope.
\newcommand{\GlobalNewIf}[1]{
    \expandafter\newif\csname if#1Global\endcsname
    \expandafter\def\csname if#1\endcsname{\csname if#1Global\endcsname}
    \expandafter\def\csname#1true\endcsname {\expandafter\global\csname#1Globaltrue\endcsname}
    \expandafter\def\csname#1false\endcsname{\expandafter\global\csname#1Globalfalse\endcsname}
}


% A simple command that acts like LaTeX's \@ifundefined, but I like the eTeX conditional more (in looks and implementation).
% Since it is eTeX, this command will allow for a switch to \@ifundefined if problems arise from non-eTeX users in the future.
\newcommand{\IfCommandExists}[3]{
    \ifcsname#1\endcsname
        #2
    \else 
        #3
    \fi
}


% <= Command - no equivalent in TeX
\newcommand{\IfGreaterThanEqualTo}[4]{
    \ifnum#1>#2
        #3
    \else
        \ifnum#1=#2
            #3
        \else
            #4
        \fi
    \fi
}

% >= Command - no equivalent in TeX
\newcommand{\IfLessThanEqualTo}[4]{
    \ifnum#1<#2
        #3
    \else
        \ifnum#1=#2
            #3
        \else
            #4
        \fi
    \fi
}


% A environment that allows for global defintions.
\newenvironment{MakeGlobal}{\globaldefs=1}{\globaldefs=0}



% Dynamic command creators.
%
% There are structured versions with proper programming whitespace;
% the 'compactified' versions are coded such that no spurious whitespace is added.
%
%   \newcommand{\MakeCommand}[2]{
%       \expandafter\newcommand\csname #1\endcsname{#2}
%   }
% 
%   \newcommand{\ReMakeCommand}[2]{
%       \expandafter\renewcommand\csname #1\endcsname{#2}
%   }
%
\newcommand{\MakeCommand}[2]{\expandafter\newcommand\csname #1\endcsname{#2}}
\newcommand{\ReMakeCommand}[2]{\expandafter\renewcommand\csname #1\endcsname{#2}}




% =========================================================================== %
%                          CSV Creation Commands                              %
% =========================================================================== %

\newcommand{\CSVMake}[3][]{ % #1 = List name, #2 = Token to push on to right

    \@ifundefined{#2}
    {
        \begin{MakeGlobal}
            \MakeCommand{#2}{#3}
            \expandafter\newcommand\csname #2Append\endcsname[1]{\CSVAppend{#2}{##1}}
            \expandafter\newcommand\csname #2Prepend\endcsname[1]{\CSVPrepend{#2}{##1}}
        \end{MakeGlobal}
    }{
        \ifthenelse{\equal{#1}{}}
            {
                \typeout{*** CSV Note: Creation ignored since list already exists. ***}
            }{
                \typeout{*** CSV Note: Overwriting already-created list. ***}
                \ReMakeCommand{#2}{#3}
                \expandafter\renewcommand\csname #2Append\endcsname[1]{\CSVAppend{#2}{##1}}
                \expandafter\renewcommand\csname #2Prepend\endcsname[1]{\CSVPrepend{#2}{##1}}
            }
    }
}

\newcommand{\CSVAppend}[2]{ % #1 = List name, #2 = Token to push on to right

    \@ifundefined{#1}
        {
                \MakeCommand{#1}{#2}
                \expandafter\newcommand\csname #1Append\endcsname[1]{\CSVAppend{#1}{##1}}
                \expandafter\newcommand\csname #1Prepend\endcsname[1]{\CSVPrepend{#1}{##1}}
        }{
            %\begin{MakeGlobal}
                \expandafter\let\expandafter\List\csname #1\endcsname
                \expandafter\def\csname #1\endcsname{\List,#2}
            %\end{MakeGlobal}
        }
}

\newcommand{\CSVPrepend}[2]{ % #1 = List name, #2 = Token to push on to left

    \@ifundefined{#1}
        {
            \MakeCommand{#1}{#2}
            \expandafter\newcommand\csname #1Append\endcsname[1]{\CSVAppend{#1}{##1}}
            \expandafter\newcommand\csname #1Prepend\endcsname[1]{\CSVPrepend{#1}{##1}}
        }{
            \expandafter\global\expandafter\protected@edef\csname #1\endcsname{#2,\csname #1\endcsname}
        }
}






% =========================================================================== %
%                            For-Loop Definition                              %
% =========================================================================== %


% Initilaze the recursive command (see usage below) ----------------------
\newcommand{\ForLoopRecursion}{}

% Define the loop counter ------------------------------------------------
\newcounter{ForLoopCounter}
\setcounter{ForLoopCounter}{0}

% Iterator command -------------------------------------------------------
\newcommand{\UWMadFor}[4][1]{
% Arguments
%   #1 = increment (optional)
%   #2 = start value
%   #3 = end value
%   #4 = <code>

    % Redefine the command used for recursion
    \renewcommand{\ForLoopRecursion}{
        #4                                          % Excute <code>
        \addtocounter{ForLoopCounter}{#1}           % Increment the counter
        \UWMadFor[#1]{\value{ForLoopCounter}}{#3}{#4}    % Recurse
    }

    % Set the counter to the start value 
    % After the recursion begins, #2 is the current value of the counter and not the start value.
    \setcounter{ForLoopCounter}{#2}
    
    % Switch to deal with positive vs. negative increments (decrements).
    \ifnum #1 > 0                                           % If positive increment
        \IfLessThanEqualTo{\value{ForLoopCounter}}{#3}{     % Execute while the LoopCounter is less than or equal to the end value
            \ForLoopRecursion
        }{}
    \else                                                    % If negative increment
        \IfGreaterThanEqualTo{\value{ForLoopCounter}}{#3}{   % Execute while the LoopCounter is greater than or equal to the end value
            \ForLoopRecursion
        }{}
    \fi
}





% =========================================================================== %
%                      Array Building Commands                                %
% =========================================================================== %

\DefineNewCounter{ArrayWorkCounter}{0}
\newcommand{\ArrayStartSuffix}    {ArrayStart}
\newcommand{\ArrayEndSuffix}      {ArrayEnd}
\newcommand{\ArrayPositionSuffix} {ArrayPosition}
\newcommand{\ArrayCountSuffix}    {ArrayCount}

\newcommand{\ArrayMake}[1]{% #1 = Array name

    \MakeCommand{#1\ArrayPositionSuffix}{#1\ArrayPositionSuffix}

    \DefineNewCounter{#1\ArrayPositionSuffix} {0}
    \DefineNewCounter{#1\ArrayStartSuffix}    {1}
    \DefineNewCounter{#1\ArrayEndSuffix}      {0}
    \DefineNewCounter{#1\ArrayCountSuffix}    {0}

}

\newcommand{\ArrayPosition}[1]{% #1 = Array name
    #1\ArrayPositionSuffix
}

\newcommand{\ArrayCount}[1]{% #1 = Array name
    #1\ArrayCountSuffix
}

\newcommand{\ArrayPositionValue}[1]{% #1 = Array name
    \value{\ArrayPosition{#1}}
}

\newcommand{\ArrayCountValue}[1]{% #1 = Array name
    \value{\ArrayCount{#1}}
}

\newcommand{\ArrayClear}[1]{% #1 = Array name
    \For[1]{1}{\value{#1\ArrayEndSuffix}}{
        \ReMakeCommand{#1\roman{ForLoopCounter}}{\relax}
    }
    \ReMakeCommand{#1\ArrayPositionSuffix}{\relax}
    \setcounter{#1\ArrayPositionSuffix} {0}
    \setcounter{#1\ArrayStartSuffix}    {1}
    \setcounter{#1\ArrayEndSuffix}      {0}
    \setcounter{#1\ArrayCountSuffix}    {0}
}



\newcommand{\ArrayPush}[2]{% #1 = Array name, #2 value to push

    % increment counter
    \addtocounter{#1\ArrayCountSuffix}    {1}
    \addtocounter{#1\ArrayEndSuffix}      {1}

    % Naming scheme for the Array commands
    \protected@edef\PushCommand{#1\roman{#1\ArrayEndSuffix}}
    
    % Making the Array commands
    \@ifundefined{\PushCommand}{
        \MakeCommand{\PushCommand}{#2}
    }{
        \ReMakeCommand{\PushCommand}{#2}
    }
}

\newcommand{\ArrayPop}[1]{% #1 = Array name
    \ifnum\value{#1\ArrayCountSuffix}>0
        \csname #1\roman{#1\ArrayEndSuffix}\endcsname
        \addtocounter{#1\ArrayCountSuffix}    {-1}
        \addtocounter{#1\ArrayEndSuffix}      {-1}
    \fi
}

\newcommand{\ArrayPopAndStore}[2]{% #1 = Array name, #2 name of command to store the popped value
    \ifnum\value{#1\ArrayCountSuffix}>0
        \protected@edef\PoppedValue{\csname #1\roman{#1\ArrayPositionSuffix}\endcsname}
        \MakeCommand{#2}{\PoppedValue}
        \addtocounter{#1\ArrayCountSuffix}    {-1}
        \addtocounter{#1\ArrayEndSuffix}      {-1}
    \fi
}

\newcommand{\ArrayGet}[2]{% #1 = Array name, % 2 = Array position to get
    \IfGreaterThanEqualTo{#2}{\value{#1\ArrayStartSuffix}}{
        \IfGreaterThanEqualTo{\value{#1\ArrayEndSuffix}}{#2}{
            \setcounter{ArrayWorkCounter}{#2}
            \csname #1\roman{ArrayWorkCounter}\endcsname
        }{2}
    }{3}
    
}





% =========================================================================== %
%                Hash (Associative Arrays) Creation Commands                  %
% =========================================================================== %
\newcommand{\HashMake}[1]{% #1 = Hash name
    \MakeCommand{#1\ArrayPositionSuffix}{#1\ArrayPositionSuffix}
}

\newcommand{\HashSet}[3]{% #1 = hash name, #2 = key name, #3 = value
    \ReMakeCommand{#1@#2}{#3}
}








