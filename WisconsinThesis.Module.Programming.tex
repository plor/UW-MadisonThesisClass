% =============================================================================================== %
%                        WisconsinThesis Class: Programming Module                                %
% =============================================================================================== %


% This is a convenience command used to define a new length and set its initial value.
% A thin abstraction that declares and intializes dimensions.
\newcommand{\DefineNewLength}[2]{
    \newlength{#1}
    \setlength{#1}{#2}
}

% This is a convenience command used to define a new counter and set its initial value.
% A thin abstraction that declares and intializes lengths.
\newcommand{\DefineNewCounter}[2]{
    \newcounter{#1}
    \setcounter{#1}{#2}
}


% =========================================================================== %
%                    New Counter System: Local and Global                     %
% =========================================================================== %
\newcommand{\UWMad@CounterSuffixLocal} {LOCAL COUNTER}
\newcommand{\UWMad@CounterSuffixGlobal}{GLOBAL COUNTER}

%   Declare a new local counter (meaning that changes to its value only 
%   persist for the current group/scope) using the TeX \newcount
\newcommand{\DefineNewLocalCounter}[2]{
    \csgdef{#1\UWMad@CounterSuffixLocal}{}
    \expandafter\newcount\csname#1\endcsname
    \csname#1\endcsname=#2
}

%   Declare a new global counter (meaning that changes to its value 
%   persist for all groups/scopes) using the LaTeX \newcounter
\newcommand{\DefineNewGlobalCounter}[2]{
    \csgdef{#1\UWMad@CounterSuffixGlobal}{}
    \expandafter\newcount\csname#1\endcsname
    \csname#1\endcsname=#2
}

%   Use the etoolbox commands to determine if the counter name
%   passed is a local (TeX) or global (LaTeX) counter
\newcommand{\LocalGlobalHandler}[1]{%
    \IfCommandExists{#1\UWMad@CounterSuffixGlobal}%
        {\let\UWMad@PreCounter\global}%
        {\IfCommandExists{#1\UWMad@CounterSuffixLocal}%
            {\let\UWMad@PreCounter\relax}%
            {\UWMad@ClassWarning{There is not local or global counter '#1'}}}%
}

%   Increment the counter by #2 basedermine if the counter name
%   passed is a local (TeX) or global (LaTeX) counter
\newcommand{\AddToCounter}[2]{
    \LocalGlobalHandler{#1}%
    \UWMad@PreCounter\expandafter\advance\csname#1\endcsname #2%
}

%   Increment the counter by 1 basedermine if the counter name
%   passed is a local (TeX) or global (LaTeX) counter
\newcommand{\StepCounter}[1]{%
    \AddToCounter{#1}{1}%
}

\newcommand{\SetCounter}[2]{
    \LocalGlobalHandler{#1}%
    \expandafter\UWMad@PreCounter\csname#1\endcsname #2%
}

\newcommand{\CounterValue}[1]{%
    \the\csuse{#1}%
}



% If-switch changes only have block level scope by default.  That is, a \newif that is switched within a block
% reverts to its pre-block state upon exit.  This command creates an if-switch set that has global scope.
\newcommand{\GlobalNewIf}[1]{%
    \csgdef{#1true}%
        {\global\cslet{if#1}{\iftrue}}
    \csgdef{#1false}%
        {\global\cslet{if#1}{\iffalse}}
}


% A simple command that acts like LaTeX's \@ifundefined, but I like the eTeX conditional more (in looks and implementation).
% Since it is eTeX, this command will allow for a switch to \@ifundefined if problems arise from non-eTeX users in the future.
\newcommand{\IfCommandExists}[3]{%
    \ifcsname#1\endcsname%
        #2%
    \else%
        #3%
    \fi%
}%
\newcommand{\IfCommandDoesNotExists}[3]{%
    \ifcsname#1\endcsname%
        #3%
    \else%
        #2%
    \fi%
}%


% <= Command - no equivalent in TeX
\newcommand{\IfGreaterThanEqualTo}[4]{%
    \ifnum#1>#2%
        #3%
    \else%
        \ifnum#1=#2%
            #3%
        \else%
            #4%
        \fi%
    \fi%
}

% >= Command - no equivalent in TeX
\newcommand{\IfLessThanEqualTo}[4]{%
    \ifnum#1<#2%
        #3%
    \else%
        \ifnum#1=#2%
            #3%
        \else%
            #4%
        \fi%
    \fi%
}


% Command combo testing for an empty string or command
\newcommand{\IfEmpty}[3]{%
    \ifdefempty{#1}%
        {#2}%
        {\ifblank{#1}%
            {#2}%
            {#3}%
        }%
}

\newcommand{\IfMatch}[4]{% #1 = left compare, #2 = right compare, #3 = true code, #4 = false code
    \if #1 #2%
        #3%
    \else%
        #4%
    \fi%
}




% A environment that allows for global defintions.
\newenvironment{MakeGlobal}{\globaldefs=1}{\globaldefs=0}

\def\Show#1{
    {
    \edef\ExpandedArgument{#1}
    \show\ExpandedArgument
    }
}

 \newcommand{\Trim}[1]{\ignorespaces#1\unskip} 



% Dynamic command creators.
\newcommand{\MakeCommand}[2]{%
    \IfCommandExists{#1}%
        {\UWMad@ClassWarning{Command '#1' is already defined; could not redefine.}}%
        {\csgdef{#1}{#2}}%
}
\newcommand{\ReMakeCommand}[2]{
    \IfCommandExists{#1}%
        {\csgdef{#1}{#2}}%
        {\UWMad@ClassWarning{Command '#1' is undefined; could not define.}}%
}
\newcommand{\MakeFullyExpandedCommand}[2]{%
    \csxdef{#1}{#2}%
}

\newcommand{\MakeCommandGlobal}[2]{\begin{MakeGlobal}\MakeCommand{#1}{#2}\end{MakeGlobal}}


\newcommand{\MakeCommandUndefined}[1]{% #1 = Command name
    \global\csundef{#1}%
}

\newcommand{\MakeCounterUndefined}[1]{% #1 = Counter name
    \MakeCommandUndefined{#1}
}




% =========================================================================== %
%                              Number Checker                                 %
% =========================================================================== %
\GlobalNewIf{UWMad@IsANumber}
\UWMad@IsANumberfalse

\MakeCommand{UWMad@RegisteredNumber0}{0}
\MakeCommand{UWMad@RegisteredNumber1}{1}
\MakeCommand{UWMad@RegisteredNumber2}{2}
\MakeCommand{UWMad@RegisteredNumber3}{3}
\MakeCommand{UWMad@RegisteredNumber4}{4}
\MakeCommand{UWMad@RegisteredNumber5}{5}
\MakeCommand{UWMad@RegisteredNumber6}{6}
\MakeCommand{UWMad@RegisteredNumber7}{7}
\MakeCommand{UWMad@RegisteredNumber8}{8}
\MakeCommand{UWMad@RegisteredNumber9}{9}

% Split an input into a Tail and a Head (if input is terminated by \Stop)
\def\UWMad@NumberTokenizer#1#2\Stop%
    {\def\Head{#1}\def\Tail{#2}}

% Take an input, and recurse through each token until done or
% a non-registered number is found.
\newcommand{\UWMad@CheckForNumber}[1]{%
    \expandafter\UWMad@NumberTokenizer#1\Stop%
    \IfEmpty{\Head}%
        {}%
        {\IfCommandExists{UWMad@RegisteredNumber\Head}%
            {\UWMad@IsANumbertrue}%
            {\UWMad@IsANumberfalse}%
         \ifUWMad@IsANumber%
            \IfEmpty{\Tail}%
                {}%
                {\UWMad@CheckForNumber{\Tail}}%
         \fi}%
}

% If all tokens are registered numbers, perform true code;
% otherwise perform false code.
\newcommand{\IfNumber}[3]{
    \UWMad@CheckForNumber{#1}%
    \ifUWMad@IsANumber%
        #2%
    \else%
        #3%
    \fi%
}



% =========================================================================== %
%                          CSV Creation Commands                              %
% =========================================================================== %
\newcommand{\CSVSuffix}{CSV LIST}

\newcommand{\IfCSVExists}[3]{%
    \IfCommandExists{#1\CSVSuffix}%
    {#2}%
    {#3}%
}

\newcommand{\CSVMake}[1]{ % #1 = List name, #2 = Token to push on to right
    \IfCSVExists{#1}
        {\UWMad@ClassWarning{CSV '#1' already exists.}}
        {\MakeCommand{#1\CSVSuffix}{}}
}

\newcommand{\CSVAppend}[2]{ % #1 = List name, #2 = Token to push on to right
     \IfCSVExists{#1}
        {\ifcsempty{#1\CSVSuffix}
            {\protected@csxdef{#1\CSVSuffix}{#2}}
            {\protected@csxdef{#1\CSVSuffix}{\csuse{#1\CSVSuffix},#2}}}
        {\CSVMake{#1}
         \CSVAppend{#1}{#2}}
}

\newcommand{\CSVPrepend}[2]{ % #1 = List name, #2 = Token to push on to left
     \IfCSVExists{#1}
        {\ifcsempty{#1\CSVSuffix}
            {\protected@csxdef{#1\CSVSuffix}{#2}}
            {\protected@csxdef{#1\CSVSuffix}{#2,\csuse{#1\CSVSuffix}}}}
        {\CSVMake{#1}
         \CSVPrepend{#1}{#2}}
}

\newcommand{\CSVGet}[1]{% #1 = List name, #2 = Token to push on to left
     \IfCSVExists{#1}%
        {\csuse{#1\CSVSuffix}}%
        {}%
}






% =========================================================================== %
%                      Array Building Commands                                %
% =========================================================================== %

\DefineNewCounter{ArrayWorkCounter}{0}
\newcommand{\ArraySuffix}         {ARRAY}
\newcommand{\ArraySuffixStart}    {\ArraySuffix START   }
\newcommand{\ArraySuffixEnd}      {\ArraySuffix END     }
\newcommand{\ArraySuffixPosition} {\ArraySuffix POSITION}
\newcommand{\ArraySuffixCount}    {\ArraySuffix COUNT   }

\newcommand{\ArrayMake}[1]{% #1 = Array name
    \MakeCommandGlobal{#1\ArraySuffix}             {#1\ArraySuffix}
    \DefineNewGlobalCounter{#1\ArraySuffixPosition} {0}
    \DefineNewGlobalCounter{#1\ArraySuffixStart}    {1}
    \DefineNewGlobalCounter{#1\ArraySuffixEnd}      {0}
    \DefineNewGlobalCounter{#1\ArraySuffixCount}    {0}
}


% Counter names for a given array
\newcommand{\ArrayPosition}[1]{#1\ArraySuffixPosition} % #1 = ArrayName
\newcommand{\ArrayStart}   [1]{#1\ArraySuffixStart}    % #1 = ArrayName
\newcommand{\ArrayEnd}     [1]{#1\ArraySuffixEnd}      % #1 = ArrayName
\newcommand{\ArrayCount}   [1]{#1\ArraySuffixCount}    % #1 = ArrayName
% Counter names for a given array
\newcommand{\ArrayNumberPosition}[1]{\csname#1\ArraySuffixPosition\endcsname}  % #1 = ArrayName
\newcommand{\ArrayNumberStart}   [1]{\csname#1\ArraySuffixStart\endcsname}    % #1 = ArrayName
\newcommand{\ArrayNumberEnd}     [1]{\csname#1\ArraySuffixEnd\endcsname}      % #1 = ArrayName
\newcommand{\ArrayNumberCount}   [1]{\csname#1\ArraySuffixCount\endcsname}    % #1 = ArrayName
% Decrementers
\newcommand{\ArrayDecrementPosition}[1]{\AddToCounter{\ArrayPosition{#1}}{-1}}
\newcommand{\ArrayDecrementStart}   [1]{\AddToCounter{\ArrayStart   {#1}}{-1}}
\newcommand{\ArrayDecrementEnd}     [1]{\AddToCounter{\ArrayEnd     {#1}}{-1}}
\newcommand{\ArrayDecrementCount}   [1]{\AddToCounter{\ArrayCount   {#1}}{-1}}
% Incrementers
\newcommand{\ArrayIncrementPosition}[1]{\AddToCounter{\ArrayPosition{#1}}{+1}}
\newcommand{\ArrayIncrementStart}   [1]{\AddToCounter{\ArrayStart   {#1}}{+1}}
\newcommand{\ArrayIncrementEnd}     [1]{\AddToCounter{\ArrayEnd     {#1}}{+1}}
\newcommand{\ArrayIncrementCount}   [1]{\AddToCounter{\ArrayCount   {#1}}{+1}}
% Numerical values
\newcommand{\ArrayValuePosition}[1]{\CounterValue{#1\ArraySuffixPosition}} % #1 = Array name
\newcommand{\ArrayValueStart}   [1]{\CounterValue{#1\ArraySuffixStart}}    % #1 = Array name
\newcommand{\ArrayValueEnd}     [1]{\CounterValue{#1\ArraySuffixEnd}}      % #1 = Array name
\newcommand{\ArrayValueCount}   [1]{\CounterValue{#1\ArraySuffixCount}}    % #1 = Array name


\newcommand{\ArrayDelete}[1]{% #1 = Array name
    \ForEach{#1}{
        \MakeCommandUndefined{#1\ArraySuffix\CounterValue{ForLoopCounter}}
    }
    \MakeCommandUndefined{#1\ArraySuffix}
    \MakeCounterUndefined{#1\ArraySuffixPosition}
    \MakeCounterUndefined{#1\ArraySuffixStart}
    \MakeCounterUndefined{#1\ArraySuffixEnd}
    \MakeCounterUndefined{#1\ArraySuffixCount}
}

\newcommand{\ArrayReset}[1]{% #1 = Array name
    \ForEach{#1}{
        \MakeCommandUndefined{#1\ArraySuffix\CounterValue{ForLoopCounter}}
    }
    \SetCounter{#1\ArraySuffixPosition} {0}
    \SetCounter{#1\ArraySuffixStart}    {1}
    \SetCounter{#1\ArraySuffixEnd}      {0}
    \SetCounter{#1\ArraySuffixCount}    {0}
}

\newcommand{\ArrayPush}[2]{% #1 = Array name, #2 value to push

    % increment counters
    \ArrayIncrementCount{#1}
    \ArrayIncrementEnd  {#1}

    % Naming scheme for the Array storage commands: <Name>\roman{<Counter>}
    \MakeCommand{#1\ArraySuffix\ArrayValueEnd{#1}}{#2}%
}

% \newcommand{\ArrayCommandPush}[2]{% #1 = Array name, #2 value to push
% 
%     % increment counters
%     \ArrayIncrementCount{#1}
%     \ArrayIncrementEnd  {#1}
% 
%     % Naming scheme for the Array storage commands: <Name>\roman{<Counter>}
%     \MakeCommand{#1\ArraySuffix\ArrayValueEnd{#1}}{#2}%
% }

\newcommand{\ArrayPop}[1]{% #1 = Array name
    \ifnum\ArrayNumberCount{#1}>0%
        \csuse{#1\ArraySuffix\ArrayValueEnd{#1}}%
        % decrement counters
        \ArrayDecrementCount{#1}%
        \ArrayDecrementEnd  {#1}%
    \fi%
}

\newcommand{\ArrayPopAndStore}[2]{% #1 = Array name, #2 name of command to store the popped value
    \ifnum\ArrayCount{#1}>0%
        \MakeFullyExpandedCommand{#2}%
            {\csuse{#1\ArraySuffix\ArrayValueEnd{#1}}}
        % decrement counters
        \ArrayDecrementCount{#1}
        \ArrayDecrementEnd  {#1}
    \fi
}

\newcommand{\ArrayGet}[2]{% #1 = Array name, % 2 = Counter Name for index
    \IfArrayExists{#1}%
        {\IfNumber{#2}%
            {\def\UWMad@LocalIndex{#2}}%
            {\def\UWMad@LocalIndex{\CounterValue{#2}}}%
         \IfArrayIndexDefined{#1}{\UWMad@LocalIndex}%
            {\csuse{#1\ArraySuffix\UWMad@LocalIndex}}%
            {\UWMad@ClassWarning{Index for array '#1' is undefined.}}}%
        {}%
}

\newcommand{\IfArrayIndexDefined}[4]{% #1 = Array name, % #2 = numeric value, % #3 = True code, % #4 = False code
    \IfCommandExists{#1\ArraySuffix#2}%
        {#3}%
        {#4}%
}

\newcommand{\IfArrayIndexNotDefined}[4]{% #1 = Array name, % #2 = numeric value, % #3 = True code, % #4 = False code
    \IfCommandExists{#1\ArraySuffix#2}%
        {#4}%
        {#3}%
}%

\newcommand{\IfArrayExists}[3]{% #1 = Array name, #2 = True code, #3 = False code
    \IfCommandExists{#1\ArraySuffix}%
        {#2}%
        {#3}%
}%




% =========================================================================== %
%                 Hashes (Associative Arrays) with LaTeX3                     %
% =========================================================================== %


% ************ LaTeX3 ON ************ %
\ExplSyntaxOn

\DeclareDocumentCommand{\HashMake}{m}%
    {\prop_new:c {g_UWMad_Hash_#1}}

\DeclareDocumentCommand{\HashSet}{mmm}%
    {\prop_if_exist:cTF {g_UWMad_Hash_#1}%
      {\prop_gput:cnn   {g_UWMad_Hash_#1}{#2}{#3}}%
        {\HashMake{#1}%
         \prop_gput:cnn {g_UWMad_Hash_#1}{#2}{#3}}}

\DeclareDocumentCommand{\HashGet}{mm}%
    {\prop_if_exist:cTF  {g_UWMad_Hash_#1}%
        {\prop_get:cnNTF {g_UWMad_Hash_#1} {#2} \l_UWMad_Hash_Value_tl%
            {\l_UWMad_Hash_Value_tl}%
            {}}%
        {}}

\DeclareDocumentCommand{\HashUnset}{mm}%
    {\prop_if_exist:cTF    {g_UWMad_Hash_#1}%
        {\prop_get:cnNTF   {g_UWMad_Hash_#1} {#2} \l_UWMad_Hash_Value_tl%
            {\prop_gdel:cn {g_UWMad_Hash_#1} {#2}}
            {}}}

\DeclareDocumentCommand{\HashDelete}{m}%
    {\prop_if_exist:cTF {g_UWMad_Hash_#1}%
        {\prop_gclear:c {g_UWMad_Hash_#1}%
         \cs_undefine:c {g_UWMad_Hash_#1}}
        {}}

\ExplSyntaxOff
% ************ LaTeX3 OFF *********** %




% *************************************************************************** %
% *********************** LaTeX2e Code (deprecated) ************************* %
% *************************************************************************** %
%
% % =========================================================================== %
% %                Hash (Associative Arrays) Creation Commands                  %
% % =========================================================================== %
% \newcommand{\HashSuffix}{HASH}
% \newcommand{\HashSuffixKeys}{\HashSuffix KEYS}
%
% \newcommand{\IfHashExists}[3]{%
%     \IfCommandExists{#1\HashSuffix}%
%         {#2}%
%         {#3}%
% }
%
% \newcommand{\HashMake}[1]{% #1 = Hash name
%     \IfHashExists{#1}
%         {\UWMad@ClassWarning{Hash '#1' already exists.}}
%         {\MakeCommand{#1\HashSuffix}{}
%          \ArrayMake{#1\HashSuffixKeys}}
% }
%
% \newcommand{\HashSet}[3]{% #1 = hash name, #2 = key name, #3 = value
%
%     \IfHashExists{#1}{}{
%         \HashMake{#1}   % If it doesn't exist, make it.
%     }
%
%     \MakeFullyExpandedCommand{Key}{#2}
%
%     \IfCommandExists{#1\HashSuffix\Key}{
%         \ReMakeCommand{#1\HashSuffix\Key}{#3}%    Only need to remake the command.
%     }{
%         \ArrayPush{#1\HashSuffixKeys}{\Key}%      Push key to Array storing all keys for this hash
%         \MakeCommand{#1\HashSuffix\Key}{#3}%      Command storing the value assoicated with the key
%     }
% }
%
% \newcommand{\HashGet}[2]{% #1 = hash name, #2 = key name
%     \IfCommandExists{#1\HashSuffix}{%
%         \IfCommandExists{#1\HashSuffix#2}{%
%             \csuse{#1\HashSuffix#2}%
%         }{%
%             \UWMad@ClassWarning{There is no key '#2' set for hash '#1'}%
%         }%
%     }{%
%         \UWMad@ClassWarning{Hash '#1' does not exist}%
%     }%
% }%
%
% \newcommand{\IfHashKeyExists}[4]{%
%     \IfCommandExists{#1\HashSuffix#2}%
%         {#3}%
%         {#4}%
% }
%
%
% \newcommand{\HashGetKeyByIndex}[2]{% #1 = hash name, #2 = index value
%     \IfHashExists{#1}%
%         {\IfArrayIndexDefined{#1\HashSuffixKeys}{#2}%
%             {\ArrayGet{#1\HashSuffixKeys}{#2}}%
%             {\UWMad@ClassWarning{Index '#2' is invalid for Hash '#1'.}}}%
%         {\UWMad@ClassWarning{Hash '#1' does not exist}}%
% }%
%
% \newcommand{\HashGetValueByIndex}[2]{% #1 = hash name, #2 = index value
%     \HashGet{#1}{\HashGetKeyByIndex{#1}{#2}}%
% }%
%
%
% \newcommand{\HashDelete}[1]{% #1 = hash name, #2 = index value
%     \IfCommandExists{#1\HashSuffix}%
%         {\ForEach[1]{#1}{%
%             \MakeCommandUndefined{#1\HashGetKeyByIndex{#1}{ForLoopCounter}\HashSuffix}%
%          }%
%          \ArrayDelete{#1\HashSuffixKeys}%
%          \MakeCommandUndefined{#1\HashSuffix}}%
%         {\UWMad@ClassWarning{Hash '#1' does not exist.}}%
% }%







% =========================================================================== %
%                            For-Loop Definition                              %
% =========================================================================== %


% Initilaze the recursive command (see usage below) ----------------------
\newcommand{\ForLoopRecursion}{}

% Define the loop counter ------------------------------------------------
\DefineNewLocalCounter{ForLoopCounter}{0}

% Iterator command -------------------------------------------------------
\newcommand{\For}[4][1]{%
    {
    % Arguments
    %   #1 = increment (optional)
    %   #2 = start value
    %   #3 = end value
    %   #4 = <code>
    % Redefine the command used for recursion
    \renewcommand{\ForLoopRecursion}{%
        #4%                                         % Excute <code>
        \AddToCounter{ForLoopCounter}{#1}%          % Increment the counter
        \For[#1]{\CounterValue{ForLoopCounter}}{#3}{#4}%   % Recurse
    }%
    %
    % Set the counter to the start value 
    % After the recursion begins, #2 is the current value of the counter and not the start value.
    \SetCounter{ForLoopCounter}{#2}%
    %
    % Switch to deal with positive vs. negative increments (decrements).
    \ifnum #1 > 0%                                    % If positive increment
        \IfLessThanEqualTo{\ForLoopCounter}{#3}{%     % Execute while the LoopCounter is less than or equal to the end value
            \ForLoopRecursion%
        }{}%
    \else%                                                   % If negative increment
        \IfGreaterThanEqualTo{\ForLoopCounter}{#3}{%   % Execute while the LoopCounter is greater than or equal to the end value
            \ForLoopRecursion%
        }{}%
    \fi%
    }
}


\newcommand{\ForEach}[3][1]{
% Arguments
%   #1 = increment (optional)
%   #2 = Array/Hash name
%   #3 = <code>
    %
    \IfCommandExists{#2\ArraySuffix}{%
        \For[#1]{\ArrayNumberStart{#2}}{\ArrayNumberEnd{#2}}{#3}%
    }{%
        \IfCommandExists{#2\HashSuffix}{%
            \For[#1]{\ArrayNumberStart{#2\HashSuffixKeys}}{\ArrayNumberEnd{#2\HashSuffixKeys}}{#3}%
        }{%
            \UWMad@ClassWarning{Could not find Array or Hash named '#2'}%
        }%
    }%
}







