%   \iffalse
%<*PROTECT>
%   \fi
%   \iffalse
\documentclass{UWMadThesisL3Doc}
%
    % \usepackage[english]{babel}

    \ExplSyntaxOn
    \DeclareDocumentCommand \LaTeXPL { } {
        \texttt{expl3}
    }
    \ExplSyntaxOff

   \Title{UWMadThesis Manual}
   \Author{Troy Christopher Haskin}

    \begin{document}
        \DocInput{\jobname.dtx}
    \end{document}
%
%
%   \fi
%
%
% \CheckSum{0}
%
%   \CharacterTable
%   {Upper-case \A\B\C\D\E\F\G\H\I\J\K\L\M\N\O\P\Q\R\S\T\U\V\W\X\Y\Z
%    Lower-case \a\b\c\d\e\f\g\h\i\j\k\l\m\n\o\p\q\r\s\t\u\v\w\x\y\z
%    Digits     \0\1\2\3\4\5\6\7\8\9
%    Exclamation   \!    Double quote \"     Hash (number) \#
%    Dollar        \$    Percent      \%     Ampersand     \&
%    Acute accent  \'    Left paren   \(     Right paren   \)
%    Asterisk      \*    Plus         \+     Comma         \,
%    Minus         \-    Point        \.     Solidus       \/
%    Colon         \:    Semicolon    \;     Less than     \<
%    Equals        \=    Greater than \>     Question mark \?
%    Commercial at \@    Left bracket \[     Backslash     \\
%    Right bracket \]    Circumflex   \^     Underscore    \_
%    Grave accent  \`    Left brace   \{     Vertical bar  \|
%    Right brace   \}    Tilde \~}
%
%
%   \UWPart{User Guide}
%   \UWFeature{Programming}
The Programming Module has no immediate user-facing features.
The Implementation section for this module outlines the programming layer
and is intended for average use.

\UWFeature{Math}\label{UG:Math}
As the feature name may suggest, all of the commands in this section deal with mathematical typesetting.

\UWSubFeature{Derivative Commands}
These command set deal with quick and easy typesetting of derivatives.

\begin{function}{\deriv , \pderiv , \tderiv}
    \begin{syntax}
        \cs{deriv}  \Arg{function} \Arg{variable} \Arg{order}    \\
        \cs{pderiv} \Arg{function} \Arg{variable} \Arg{order}    \\
        \cs{tderiv} \Arg{function} \Arg{variable} \Arg{order}
    \end{syntax}
    This function set is meant to typeset three different kinds of derivatives: ordinary, partial, and total (i.e., material or Lagragian).
    The only difference between them is the differential symbol: \cs{deriv} uses `$\mathrm{d}$', \cs{pderiv} uses `$\partial$', and \cs{tderiv} used `$\mathrm{D}$'.

    These commands typeset the derivative of a given \Arg{function} with respect to \Arg{variable} of $n$-th \Arg{order} using Leibniz's notation.
    The \Arg{order} is optional and defaults to empty (first derivative).
    For example, the input
    \begin{verbatim}
        \begin{align}
            \deriv{y}{x}{2} + \deriv{y}{x} + y(x)         &= 0    \\[0.50em]
            \pderiv{T}{t} - \alpha \pderiv{T}{z}{2}       &= 0    \\[0.50em]
            \tderiv{\rho{u}}{t} + \pderiv{P}{z}  - \rho g &= 0
        \end{align}
    \end{verbatim}
    and is typeset as
        \begin{align}
            \deriv{y}{x}{2} + \deriv{y}{x} + y(x)         &= 0    \\[0.50em]
            \pderiv{T}{t} - \alpha \pderiv{T}{z}{2}       &= 0    \\[0.50em]
            \tderiv{(\rho{u})}{t} + \pderiv{P}{z}  - \rho g &= 0
        \end{align}
\end{function}

\begin{function}{\derivbig,\pderivbig,\tderivbig}
    \begin{syntax}
        \cs{derivbig}   \oarg{left delim} \Arg{function} \oarg{right delim} \Arg{variable} \Arg{order} \\
        \cs{pderivbig}  \oarg{left delim} \Arg{function} \oarg{right delim} \Arg{variable} \Arg{order} \\
        \cs{tderivbig}  \oarg{left delim} \Arg{function} \oarg{right delim} \Arg{variable} \Arg{order}
    \end{syntax}
    This function set is identical to the non-|big| versions above, except that \Arg{function} is placed to the right of the derivative operator and wrapped by |\left| and |\right|.
    The default delimiters for the stretch commands are `[' and ']', and either can be individually overridden via the two optional arguments.
    For example, the input
    \begin{verbatim}
        \begin{align}
            -\derivbig{ p(x) \deriv{y}{x} }{x} +
                    q(x) (1 - \lambda) y(x)  &= 0 \\[0.50em]
            \tderivbig{ \rho{i} + \frac{1}{2} \rho u^2 }[(]{t} -
                    \pderivbig[\lvert]{ \kappa \pderiv{T}{z} }{z} &= 0
        \end{align}
    \end{verbatim}
    and is typeset as
        \begin{align}
            -\derivbig{ p(x) \deriv{y}{x} }{x} +
                    q(x) (1 - \lambda) y(x)  &= 0 \\[0.50em]
            \tderivbig{ \rho{i} + \frac{1}{2} \rho u^2 }[(]{t} -
                    \pderivbig[\lvert]{ \kappa \pderiv{T}{z} }{z} &= 0
        \end{align}
\end{function}

\begin{function}{\DerivativeGeneral,\DerivativeGeneralBig}
    \begin{syntax}
        \cs{DerivativeGeneral}     \Arg{function} \Arg{variable} \Arg{order} \Arg{symbol}
        \cs{DerivativeGeneralBig}  \Arg{function} \Arg{variable} \Arg{order} \Arg{symbol} \Arg{left delim} \Arg{right delim}
    \end{syntax}
    These commands are lower-level commands used by the |deriv| family above.
    All of the arguments are mandatory.
    If a change to the general style of the derivatives or another version of the |deriv| family is desire, these commands are available for usage.
\end{function}

\begin{function}{\derivSymbol,\pderivSymbol,\tderivSymbol}
    \begin{syntax}
        \cs{derivSymbol}
    \end{syntax}
    These commands take no arguments and expand to the current symbol used for the associated |deriv| command.
    The defaults require math mode to be typeset.
    Therefore, |$\pderivSymbol$| will be appear as $\pderivSymbol$.
\end{function}

\begin{function}{\derivSymbolChange,\pderivSymbolChange,\tderivSymbolChange}
    \begin{syntax}
        \cs{derivSymbolChange} \Arg{symbol} \\[0.50em]
    \end{syntax}
    These commands will \textsc{temporarily} change the symbol used by the associated |deriv| commands.
    The symbol will revert back to the original, default value after leaving the \TeX{} group where the switch was made (more often than not for \LaTeX{} users, this means ``upon exiting an environment'').
    For example:
    \begin{verbatim}
        \begin{equation}
            \deriv{U}{t} =
            \derivSymbolChange{\delta}
            \deriv{Q}{t} - \deriv{W}{t}
        \end{equation}
    \end{verbatim}
    typesets as
    \begin{equation}
            \deriv{U}{t} =
            \derivSymbolChange{\delta}
            \deriv{Q}{t} - \deriv{W}{t}
    \end{equation}
    and now, after the environment, the \cs{derivSymbol} is once again `$\derivSymbol$'.
\end{function}

\begin{function}{\derivSymbolChangeDefault,\pderivSymbolChangeDefault,\tderivSymbolChangeDefault}
    \begin{syntax}
        \cs{derivSymbolChangeDefault} \Arg{symbol} \\[0.50em]
    \end{syntax}
    These commands will \textsc{permanently} change the symbol used by the associated |deriv| commands.
    For example:
    \begin{verbatim}
        \begin{equation}
            \deriv{U}{t} =
            \derivSymbolChangeDefault{\delta}
            \deriv{Q}{t} - \deriv{W}{t}
        \end{equation}
    \end{verbatim}
    typesets as
    \begin{equation}
            \deriv{U}{t} =
            \derivSymbolChangeDefault{\delta}
            \deriv{Q}{t} - \deriv{W}{t}
    \end{equation}
    and now, after the environment, the \cs{derivSymbol} is `$\derivSymbol$'.
\end{function}

\begin{function}{\DelimiterChangeDefault}
    \begin{syntax}
        \cs{DelimiterChangeDefault} \Arg{left delim} \Arg{right delim}
    \end{syntax}
    This command changes the default delimiters used by the |big| commands above.
    Any valid delimiters can be used.
    For example:
    \begin{verbatim}
        \DelimiterChangeDefault{(}{)}
        \begin{equation}
            -\derivbig{ p(x) \deriv{y}{x} }{x} +
                    q(x) (1 - \lambda) y(x) = 0 \\[0.50em]
        \end{equation}
    \end{verbatim}
    and is typeset as
    \DelimiterChangeDefault{(}{)}
    \begin{equation}
        -\derivbig{ p(x) \deriv{y}{x} }{x} +
                q(x) (1 - \lambda) y(x) = 0 \\[0.50em]
    \end{equation}
    and notice that the \cs{derivSymbol} is still $\derivSymbol$.
\end{function}

\UWSubFeature{Operators}
These operators are added to the standard set using the \AmS{} operator system.
Some are new while others are simply in a camel-cased versions of the standard ones.

\begin{function}{\Sup,\Inf}
    Supremum and Infinum operators using the math operator system.
    For example, the input
    \begin{verbatim}
        \begin{align}
            \Inf_{x \in \mathbb{R}} \{0 < x  < 1\} &= 0 \\[0.50em]
            \Sup_{x \in \mathbb{R}} \{0 < x  < 1\} &= 1
        \end{align}
    \end{verbatim}
    is typeset as
    \begin{align}
        \Inf_{x \in \mathbb{R}}
            \{0 \LessThan x \LessThan 1\} &= 0 \\[0.50em]
        \Sup_{x \in \mathbb{R}}
            \{0 \LessThan x \LessThan 1\} &= 1
    \end{align}
\end{function}

\begin{function}{\Lim}
    The limit operator:
    \begin{verbatim}
        \begin{equation}
            \Lim_{n \rightarrow \infty} \left(1 + \frac{1}{n}\right)^n = \mathrm{e}
        \end{equation}
    \end{verbatim}
    is typeset as
        \begin{equation}
            \Lim_{n \rightarrow \infty} \left(1 + \frac{1}{n}\right)^n = \mathrm{e}
        \end{equation}
\end{function}

\begin{function}{\Min,\Max}
    The maximum and minimum value operators
    \begin{verbatim}
        \begin{equation}
            \begin{align}
                \Min_{x \in \mathbb{R}} \Sin(x) &= -1 \\[0.50em]
                \Max_{x \in \mathbb{R}} \Sin(x) &= +1
            \end{align}
        \end{equation}
    \end{verbatim}
    is typeset as
    \begin{align}
        \Min_{x \in \mathbb{R}} \Sin(x) &= -1 \\[0.50em]
        \Max_{x \in \mathbb{R}} \Sin(x) &= +1
    \end{align}
\end{function}

\begin{function}{\ArgMin,\ArgMax}
    The maximum and minimum argument operators
    \begin{verbatim}
        \begin{equation}
            \begin{align}
                \ArgMin_{x \in \mathbb{R}} \Sin(x) &= \frac{3\pi}{2} + 2 \pi n \\[0.50em]
                \ArgMax_{x \in \mathbb{R}} \Sin(x) &= \frac{\pi}{2} + 2 \pi n
            \end{align}
        \end{equation}
    \end{verbatim}
    is typeset as
    \begin{align}
                \ArgMin_{x \in \mathbb{R}} \Sin(x) &= \frac{3\pi}{2} + 2 \pi n \\[0.50em]
                \ArgMax_{x \in \mathbb{R}} \Sin(x) &= \frac{\pi}{2} + 2 \pi n
    \end{align}
\end{function}

\begin{function}{\Abs,\Ln,\Log,\Exp}
    Common set of operators in uppercase form.
\end{function}

\begin{function}{\Cos,\Sin,\Tan,\Sec,\Csc,\Cot}
    Standard trigonometric functions and their reciprocals.
\end{function}
\begin{function}{\Cosh,\Sinh,\Tanh,\Sech,\Csch,\Coth}
    Hyperbolic trigonometric functions and their reciprocals.
\end{function}
\begin{function}{\ArcCos,\ArcSin,\ArcTan,\ArcSec,\ArcCsc,\ArcCot}
    Standard inverse trigonometric functions and their reciprocals.
\end{function}
\begin{function}{\ArcCosh,\ArcSinh,\ArcTanh,\ArcSech,\ArcCsch,\ArcCoth}
    Hyperbolic inverse trigonometric functions and their reciprocals.
\end{function}

\UWSubFeature{Miscellaneous Commands}

\begin{function}{\Sqrt}
    \begin{syntax}
        \cs{Sqrt} \oarg{n} \Arg{argument}
    \end{syntax}
    This command typesets the \oarg{n}-th root of a given \Arg{argument} with a closing tail.
    This command differs from the default \cs{sqrt} in appearance only:
    \begin{equation}
        \sqrt[3]{\frac{f(x)}{g(x)}} = \Sqrt[3]{\frac{f(x)}{g(x)}}
    \end{equation}
\end{function}

\begin{function}{\IfMathModeTF}
    \begin{syntax}
        \cs{IfMathModeTF} \Arg{math mode code} \Arg{text mode code}
    \end{syntax}
    This is an abstraction of |expl3|'s |\mode_if_math:TF| function.
    It was added to give more control on the following \cs{subs} and \cs{sups} commands since |expl3|'s syntax is disabled to make |_| a subscript shift and not a letter.
\end{function}

\begin{function}{\subs,\sups,\subsups}
    \begin{syntax}
        \cs{subs}    \oarg{space} \Arg{text subscript} \\[0.50em]
        \cs{sups}    \oarg{space} \Arg{text superscript} \\[0.50em]
        \cs{subsups} \oarg{subscript space} \Arg{text subscript} \oarg{superscript space} \Arg{text superscript}
    \end{syntax}
    These command typeset a subscript or superscript \textsc{in text mode}.
    They are useful if the subscript or superscript are not variable, and therefore should be in non-math text, or for making subscripts or superscripts in text mode.
    The optional argument \oarg{space} is meant for adjusting the spacing of the scripts and exists in \textsc{in math mode}, so technically, any valid math statement can be used.
    However, it is encouraged to only use this argument for spacing.
    For example, the input |`T\subs{P}, $T\subs{P}$, $T_P$'| is typeset as `T\subs{P}, $T\subs{P}$, $T_P$', and the input |`T\subs[\!]{P}, T\subs[\:]{P}'| is typeset as `T\subs[\!]{P}, T\subs[\:]{P}'.
    T\sups{P}
\end{function}

\begin{function}{\OneOver,\oneo}
    \begin{syntax}
        \cs{OneOver}  \Arg{denominator}
    \end{syntax}
    A simple command the typesets a fraction whose numerator is always one.
    For example, the input
    \begin{verbatim}
        \begin{equation}
            \OneOver{\Sqrt{x^2 + 1}}
        \end{equation}
    \end{verbatim}
    is typeset as
        \begin{equation}
            \OneOver{\Sqrt{x^2 + 1}}
        \end{equation}
\end{function}

\begin{function}{\dd}
    \begin{syntax}
        \cs{dd}  \Arg{variable}
    \end{syntax}
    A simple command the typesets a non-math `d' in math mode and is meant to be used for differentials.
    For example, the input
    \begin{verbatim}
        \derivSymbolChangeDefault{\mathrm{d}}
        \begin{equation}
            f(b) - f(a) = \int_a^b \deriv{f}{t} \dd{t}
        \end{equation}
    \end{verbatim}
    is typeset as
        \derivSymbolChangeDefault{\mathrm{d}}
        \begin{equation}
            f(b) - f(a) = \int_a^b \deriv{f}{t} \dd{t}
        \end{equation}
\end{function}

\begin{function}{\dprime,\tprime}
    \begin{syntax}
        \cs{dprime}
    \end{syntax}
    These commands take no arguments and simply mean `double prime' and `triple prime'.
    For example, the input
    \begin{verbatim}
        \begin{equation}
            q^\prime = q^\dprime 2\pi{R} = q^\tprime \pi{R^2}
        \end{equation}
    \end{verbatim}
    is typeset as
        \begin{equation}
            q^\prime = q^\dprime 2\pi{R} = q^\tprime \pi{R^2}
        \end{equation}
\end{function}


\UWFeature{List Environments}

The \UWMadClass{} has a special set of functions from creating list environments (called |ListOf| in the implementation).
The functions use queues and associative arrays to store and use data before it is typeset.
These data structures allow for operations to be carried out without writing external files or repeating compilation; of course, there is added memory usage which could lead to problems on older systems.

The primary motivation for such a system was the creation of a nomenclature environment and, subsequently, an acronym environment/system.
These two similar features are discussed here.

\UWSubFeature{Nomenclature}
The |Nomenclature| environment is, by default, a list of |(symbol, description)| entries.
There is a user option for changing the system to a list of |(symbol, units, description)| entries if a separate unit column is desired.
For every set of entries, the nomenclature system measures the width of the |symbol| and (if present) |units| to determine the maximum width of the |description| such that no text overflows into the margins of the page.

When first adding entries to a nomenclature, the symbols are part of the so-called Main group.
The Main group has a title and a section level associated with it.
By default, the Main group title is ``Nomenclature'' and the section is ``chapter''.
The entries can be put into two lower sectioned groups using the \cs{Group} and \cs{Subgroup} commands described below.
The grouping commands allows a set of symbols to be classified as ``Greek Symbols'' while another is ``Subscripts''.
The default titles for these lower groups are empty by default and the default section is ``section'' and ``subsection''.

All of these defaults can be changed by the \cs{NomenclatureSetup} command described below.

\UWSubSubFeature{Command Descriptions}

A sketch of the |Nomenclature| implementation would be: \vskip-0.50em
    \hspace{1em}|\begin{Nomenclature}|\oarg{section}\marg{title}    \vskip-0.25em
    \hspace{3em}|\Entry|\marg{symbol}\marg{description}             \vskip-1em
    \hspace{3em}|\Group|\marg{group title}                          \vskip-1em
    \hspace{6em}|\Entry|\marg{symbol}\marg{description}             \vskip-1em
    \hspace{6em}|\Subgroup|\marg{subgroup title}                    \vskip-1em
    \hspace{9em}|\Entry|\marg{symbol}\marg{description}             \vskip-0.25em
    \hspace{1em}|\end{Nomenclature}|                                \vskip-0.50em

The square brace-delimited \oarg{section} is \textsc{optional} and overrides the default Main group section.
The curly brace-delimited  \marg{title} is \textsc{optional} and overrides the default Main group title.

\begin{function}{\Entry}
    \begin{syntax}
        \cs{Entry}\marg{symbol}\marg{description}
        \cs{Entry}\marg{symbol}\marg{units}\marg{description}
    \end{syntax}
    Within the environment, entries are added to the nomenclature using the \cs{Entry} command above.
    All arguments are required.
    The second version above is if a units column is requested (see \RefSubSubFeature{Customization}).
\end{function}

\begin{function}{\Group,\Subgroup}
    \begin{syntax}
        \cs{Group}\marg{group title}
        \cs{Subgroup}\marg{subgroup title}
    \end{syntax}
    Creates a group or subgroup with the indicated title and using the default section.
    The default section can be changed by the user (see \RefSubSubFeature{Customization}).
\end{function}

\UWSubSubFeature{Examples}
As an example, the following input
\begin{verbatim}
    \begin{Nomenclature}[subsubsection]{Symbol Table}
        \Entry{$\rho$}{Density}
        \Entry{LongNotRealSymbol}{
            In publishing and graphic design, lorem ipsum is a placeholder
            text commonly used to demonstrate the graphic elements of a
            document or visual presentation. By replacing the distraction
            of meaningful content with filler text of scrambled Latin it
            allows viewers to focus on graphical elements such as font,
            typography, and layout.}
        \Entry{$\mu$}{Viscosity}
    \end{Nomenclature}
\end{verbatim}
would be typeset as:

\setcounter{section}{1}
\NomenclatureSetup{include-in-toc = false}

\rule{\textwidth}{0.1em}
    \begin{Nomenclature}[subsection]{Symbol Table}
        \Entry{$\rho$}{Density}
        \Entry{LongNotRealSymbol}{
            In publishing and graphic design, lorem ipsum is a placeholder
            text commonly used to demonstrate the graphic elements of a
            document or visual presentation. By replacing the distraction
            of meaningful content with filler text of scrambled Latin it
            allows viewers to focus on graphical elements such as font,
            typography, and layout.}
        \Entry{$\mu$}{Viscosity}
    \end{Nomenclature}
\rule{\textwidth}{0.1em}
As can be seen, the symbol column is as wide as the widest symbol (plus some padding) and lengthy text can be put into the description without penalty.
Of course, this example is purposefully extreme.
We can tweak the example a bit more by putting the second two items under a group:

\rule{\textwidth}{0.1em}
    \begin{Nomenclature}[subsection]{Symbol Table}
        \Entry{$\rho$}{Density}

        \Group{Group 1 Title}
        \Entry{LongNotRealSymbol}{
            In publishing and graphic design, lorem ipsum is a placeholder
            text commonly used to demonstrate the graphic elements of a
            document or visual presentation. By replacing the distraction
            of meaningful content with filler text of scrambled Latin it
            allows viewers to focus on graphical elements such as font,
            typography, and layout.}
        \Entry{$\mu$}{Viscosity}
    \end{Nomenclature}
\rule{\textwidth}{0.1em}
By default, the section level used by \cs{Group} is one below that of the main nomenclature section; therefore, since the nomenclature's section level is defined as |subsection|, the \cs{Group} is a |subsubsection|.
Not shown: using \cs{Subgroup} would typeset the title as a |paragraph| in this example.

\UWSubSubFeature{Customization}

As mentioned, there are several options available to the user for customizing the nomenclature.
These options are set by giving a comma-separate list of key-value pairs to the function \cs{NomenclatureSetup}

\begin{function}{\NomenclatureSetup}
    \begin{syntax}
        \cs{NomenclatureSetup}\marg{key-value CSV}
    \end{syntax}
    The format is more appropriately shown as
    \begin{verbatim}
        \NomenclatureSetup {
            key-one = option,
            key-two = {option two},
            ...
            key-n =  {option n},
        }
    \end{verbatim}
    A table of the keys, meaning, defaults, and allow value is given in \cref{Table:NomenclatureKeyValue}.
\end{function}

\clearpage
\UWSubFeature{Acronym}

\UWSubSubFeature{Description}
The |Acronym| environment is a specialized extension of the |Nomenclature| environment.
It has the same basic syntax, but a |units| column is not supported.
Also, instead of \cs{Entry} taking |(symbol, description)| pairs, it takes |(acronym,meaning)| pairs.
Lastly, it comes equipped with a new command: \cs{Acro}.

\begin{function}{\Acro}
    \begin{syntax}
        \cs{Acro}\marg{acronym}
    \end{syntax}
    \cs{Acro} is meant to be used throughout the document to reference back to the |Acronym| environment where it was defined.
    If an |Acronym| environment contains the line |\Entry{TBD}{To be determined}|, the first usage of |\Arco{TBD}| will be typeset as `To be determined (TBD)' while subsequent uses will simply be `TBD'.
    Also, if links are not turned off (they are on by default), the acronym will be a link back to the original environment entry.
\end{function}

\begin{function}{\AcronymSetup}
    \begin{syntax}
        \cs{AcronymSetup}\marg{key-value CSV}
    \end{syntax}
    An exact copy of \cs{NomenclatureSetup}.
\end{function}

\UWSubSubFeature{Example}
The following input
\begin{verbatim}
    \AcronymSetup {
        main-section  = section,
        main-title = {Acronym Table},
        entry-padding = 1in
    }
    \begin{Acronym}
        \Entry{RCCS}{Reactor Cavity Cooling System}
        \Entry{NRC}{Nuclear Regulatory Commission}
    \end{Acronym}
\end{verbatim}
is typeset as

\clearpage

\rule{\textwidth}{0.1em}
    \AcronymSetup {
        main-section  = section,
        main-title = {Acronym Table},
        entry-padding = 1in
    }
    \begin{Acronym}
        \Entry{RCCS}{Reactor Cavity Cooling System}
        \Entry{NRC}{Nuclear Regulatory Commission}
    \end{Acronym}
\rule{\textwidth}{0.1em}

The first usage of |\Acro{NRC}| is `\Acro{NRC}' while the second usage is `\Acro{NRC}'.

\begin{table}[H]
\begin{center}
    \caption{List of key-value pairs for Nomenclature customization.}
    \label{Table:NomenclatureKeyValue}
    \begin{tabular}{c c c c}
        \toprule
        Key & Meaning & Default & Allow value \\
        \midrule
        title-skip          & Vertical space following the printed title     & 0pt          & dimension \\[10pt]
        print-skip          & Vertical space following a printing of entries & 1em          & dimension \\[10pt]
        entry-margin-left   & Horizontal margin left of an entry             & 1em          & dimension \\[10pt]
        entry-margin-bottom & Vertical margin below a printed entry          & 0.25em       & dimension \\[10pt]
        entry-padding       & Horizontal space between columns               & 0.75em       & dimension \\[10pt]
        main-section        & Section level for Main group                   & chapter      & section \\[10pt]
        group-section       & Section level for \cs{Group} command           & section      & section \\[10pt]
        subgroup-section    & Section level for \cs{Subgroup} command        & subsection   &  section \\[10pt]
        main-title          & Title for the nomenclature                     & Nomenclature & any text \\[10pt]
        group-title         & Title for the \cs{Group} command               & ---          & any text \\[10pt]
        subgroup-title      & Title for the \cs{Subgroup} command            & ---          &  any text \\[10pt]
        include-in-toc      & Include the nomenclature in the TOC            & true         & boolean \\[10pt]
        with-units          & Include a units column                         & false        & boolean \\
        \bottomrule
    \end{tabular}
\end{center}
\end{table}

\begin{table}[H]
\begin{center}
    \caption{Additional key-value pairs for Acronym environment.}
    \label{Table:AcronymKeyValue}
    \begin{tabular}{c c c c}
        \toprule
        Key & Meaning & Default & Allow value \\
        \midrule
        use-links           & Create hyperlink to Acronym entry  & true & boolean \\[10pt]
        link-color          & Color of hyperlink text            & blue & color   \\
        \bottomrule
    \end{tabular}
\end{center}
\end{table}


\UWFeature{Thesis and PDF Information}

\vskip3em
In order for the \RefSubFeature{Title Page} to function properly, a certain amount of information about the thesis must be given.
The \UWMadClass{} has a set of commands to provide both the thesis information and PDF metadata to \LaTeX{}.

It is highly encouraged to use all of these commands in the preamble such that any PDF metadata can be directly set before the document begins.
If the commands are used within the |document| environment, it will require another \LaTeX{} compilation to include the metadata since \UWMadClass{} will automatically write the information to an external file.

\UWSubFeature{Required}
    These commands are required for the document to be typeset properly.
    It is encouraged to use these commands in the preamble of the document.

    \begin{function} {
        \Title,
        \Author,
        \Program,
        \Degree
    }
        \begin{syntax}
            \vspace*{3pt}
            \setstretch{1.30}
            \cs{Title}   \marg{title}
            \cs{Author}  \marg{author name}
            \cs{Program} \marg{program}
            \cs{Degree}  \marg{degree}
        \end{syntax}
        Each of these commands must be used once; if not, their respective variables be empty while being typeset.
        They can, of course, be used more than once, but the additional usages would only redefine the value of the associated variable.
    \end{function}

    \begin{function} {
        \DefenseDate,
        \DefenceDate,
        \Date
    }
        \begin{syntax}
            \vspace*{3pt}
            \setstretch{1.30}
            \cs{DefenseDate} \marg{defense date}
            \cs{DefenceDate} \marg{defense date}
            \cs{Date}        \marg{defense date}
        \end{syntax}
        Only one of these commands is needed since they all point to the same variable \marg{defense date}.
        The aliases were created for personal preference only.

        Since \marg{defense date} has no parsing performed on it, it may be entered any which way and will be typeset as-entered.
    \end{function}

    \begin{function} {
        \Institution,
        \University
    }
        \begin{syntax}
            \vspace*{3pt}
            \setstretch{1.30}
            \cs{Institution} \marg{institution name}
            \cs{University}  \marg{institution name}
        \end{syntax}
        Only one of these commands is needed since they both point to the same variable \marg{institution name}.
        The aliases were created for personal preference only.
    \end{function}

    \begin{function} {\CommitteeMember}
        \begin{syntax}
            \vspace*{3pt}
            \setstretch{1.30}
            \cs{CommitteeMember} \marg{member name} \marg{member position}
        \end{syntax}
        \cs{CommitteeMember} can be used as many times as required.
        However, if the list of members becomes too large, formatting of the \RefSubFeature[title page]{Title Page} will suffer.
        This may be fixed in the future but would require a much more sophisticated template for the title page (possibly using |expl3| |coffins|).
    \end{function}

\UWSubFeature{Optional}
These commands are not required for the document to be typeset properly.
However, they do provide metadata for the PDF (e.g., keywords and document subject) that is convenient for searching and categorization.
It is encouraged to use these commands in the preamble of the document.

\begin{function} {
    \Advisor,
    \Adviser
    }
    \begin{syntax}
        \vspace*{3pt}
        \setstretch{1.305}
        \cs{Advisor} \marg{advisor name} \marg{advisor position}
        \cs{Adviser} \marg{advisor name} \marg{advisor position}
    \end{syntax}
    Using either of these commands automatically adds the advisor/adviser to the top of the committee list created by \cs{CommitteeMember}.
    Also, on the title page's committee list, the advisor/adviser is marked as such by ``(Advisor)'' or ``(Adviser)''.
    This is a rare exception where the choice of alias has a side-effect.
\end{function}

\begin{function} {
    \Subject,
    \Keywords
    }
    \begin{syntax}
        \vspace*{3pt}
        \setstretch{1.305}
        \cs{Subject}  \marg{document subject}
        \cs{Keywords} \marg{list of keywords}
    \end{syntax}
\end{function}

\begin{function} {
    \Producer,
    \Creator
    }
    \begin{syntax}
        \vspace*{3pt}
        \setstretch{1.305}
        \cs{Producer} \marg{pdf producer}
        \cs{Creator}  \marg{pdf creator}
    \end{syntax}
\end{function}


\UWFeature{Special Pages}

\UWSubFeature{Title Page}
This is a replace for the default \cs{maketitlepage}.
Per the example provided by the \UWMadLong{} Graduate School's sample, the sample page flows (in order): thesis title, author by-line, partial fulfillment clause, degree, program, university identification, oral defense date, and oral committee list.
The styles can be re-worked by redefining the commands as presented in the \RefSubModule{MakeTitlePage} implementation.
The formatting of the commands is standard \LaTeXe{} to facilitate customization.

\textsc{Note:} The \cs{MakeTitlePage} command needs the required thesis information from the commands described in the \RefSubFeature[Required subsection]{Required}.

\UWSubFeature{License Page}

There are two main licenses \UWMadClass{} supports: Copyright and Creative Commons.
If an author wishes to use these supported licenses to create a license page, all of the commands listed must be placed within a |LicensePage| environment, or the commands will not work (by design).

To declare a simple Copyright input
\begin{verbatim}
    \begin{LicensePage}
        \Copyright
    \end{LicensePage}
\end{verbatim}
To declare a simple Creative Commons input
\begin{verbatim}
    \begin{LicensePage}
        \CreativeCommons
    \end{LicensePage}
\end{verbatim}
There are more features for the Creative Commons license and are discussed below.

The above examples will automatically create a page using default values for license owner (the \RefSubFeature[thesis author]{Required}), year (the current year), and license specifics (outlined below).
If either is incorrect for the current usage, use the following commands:
\begin{function} {
        \LicenseOwner,
        \LicenseYear
}
    \begin{syntax}
        \vspace*{3pt}
        \setstretch{1.30}
        \cs{LicenseOwner} \marg{owner name}
        \cs{LicenseYear}  \marg{year}
    \end{syntax}
    These commands override the default values with the supplied, mandatory argument.
\end{function}

\UWSubSubFeature{Copyright}
The Copyright Act of 1976 (\href{http://www.copyright.gov/title17}{Title 17 of the United States Code}, section 106) lists the following six exclusive rights the owner of copyright and any other sanctioned parties have:
\begin{enumerate}
    \item{to reproduce the copyrighted work in copies or phonorecords}
    \item{to prepare derivative works based upon the copyrighted work}
    \item{  to distribute copies or phonorecords of the copyrighted work to the public by sale or other transfer of ownership, or by rental, lease, or lending}
    \item{in the case of literary, musical, dramatic, and choreographic works, pantomimes, and motion pictures and other audiovisual works, to perform the copyrighted work publicly}
    \item{in the case of literary, musical, dramatic, and choreographic works, pantomimes, and pictorial, graphic, or sculptural works, including the individual images of a motion picture or other audiovisual work, to display the copyrighted work publicly}
    \item{in the case of sound recordings, to perform the copyrighted work publicly by means of a digital audio transmission}
\end{enumerate}
There are a number of exceptions and limitations to these rights as outlined by subsequent sections (Title 17 of the United States Code, sections 107 -- 122), but these will not be discussed.
Under section 302 of the Copyright Act, the exclusive rights granted to a singular author of a work persist for 70 years following her death.

Section 401 of the Copyright Act requires a Form of Notice of copyright.
It consists of the elements: the copyright symbol \copyright{} (or the word ``Copyright''), the year of first publication (with more requirements for derivative works), and the name of the owner of the copyright (or some other designation).
All works containing this notice of copyright fall under the protection of the Copyright Law of the United States.

Section 408 of the Copyright Act states: for any work produced after 1978, ``the owner of copyright or of any exclusive right in the work may obtain registration of the copyright claim by delivering to the Copyright Office the deposit specified by this section, together with the application and fee''.
In others words, a copy of the work can be submitted to the Copyright Office and subsequently placed in the Library of Congress for official recognition of copyright.
However, registration is not compulsory since ``[s]uch registration is not a condition of copyright protection''.

\UWSubSubFeature{Creative Commons}

\begin{LicensePage}
    \CreativeCommons
    \NonCommercial
    \ShareAlike
\end{LicensePage}


%   \GetFileInfo{UWMadThesis.cls}
%   \StopEventually{}
%
%
%
%
%
%
%^^A ==================================================== !
%^^A                                                      !
%^^A                    BEGIN IMPLEMENTATION              !
%^^A                                                      !
%^^A ==================================================== !
%
%   \iffalse
%</PROTECT>
%   \fi
%   \iffalse
%<*Code>
%   \fi
%
%
%   \UWPart{Implementation}
%   \UWModule{Front Matter}
%
%   Much of this class is written using the \LaTeX3 Programming Layer;
%   this will be denoted as \LaTeXPL{}.  The \LaTeXPL{} is the first
%   piece of a new system designed to succeed \LaTeXe{} in the future.
%   However, while the programming layer is solid and remarkable,
%   a lot of presentation work still needs to be done.  Therefore,
%   this class uses \LaTeXe{} code where necessary and will hopefully
%   be slowly pulled out as needed.  The good news is that since everything
%   is more-or-less an abstraction of \TeX{}, it should work together well.
%
%   \UWSubModule{expl3 Package and Identification}
%   The |expl3| package loads the \LaTeXPL{} and is therefore required.
%   If the package is not recent enough, the class aborts and requests
%   the user update.
%    \begin{macrocode}
\RequirePackage{expl3}[2013/07/28]
\@ifpackagelater{expl3}{2013/07/28} {} {%
    \PackageError{UWMadThesis}{Version of l3kernel is too old}
      {%
        Please install an up to date version of l3kernel\MessageBreak
        using your TeX package manager or from CTAN.
      }%
    \endinput
}%
%    \end{macrocode}
%
%   Assuming the the |expl3| package is recent enough, we provide the class
%   using the \LaTeXPL{}'s provide function.
%    \begin{macrocode}
\ProvidesExplClass
    {UWMadThesis}{2013/08/21}{1.0}{LaTeX2e+~Thesis~Class~for~UW-Madison}
%    \end{macrocode}
%
%
%   \UWSubModule{Identification and Defaults}
%
%   Now, we define some identification variables (token lists).
%    \begin{macrocode}
\tl_const:Nn \c_UWMad_ClassName_tl        {UWMadThesis}
\tl_const:Nn \c_UWMad_UniversityLong_tl   {University~of~Wisconsin-Madison}
\tl_const:Nn \c_UWMad_UniversityShort_tl  {UW-Madison}
\tl_const:Nn \c_UWMad_ClassVersion_tl     {1.0}
\tl_const:Nn \c_UWMad_ClassVersionDate_tl {2012/01/09}
%    \end{macrocode}
%
%   And since these identifications may be desired in typsetting more,
%   where \LaTeXPL{}'s syntax will be turned off, we define some aliases.
%    \begin{macrocode}
\cs_new_eq:NN \UWMadClassName    \c_UWMad_ClassName_tl
\cs_new_eq:NN \UWMadClass        \c_UWMad_ClassName_tl
\cs_new_eq:NN \UWMadLong         \c_UWMad_UniversityLong_tl
\cs_new_eq:NN \UWMadShort        \c_UWMad_UniversityShort_tl
\cs_new_eq:NN \UWMadClassVersion \c_UWMad_ClassVersion_tl
\cs_new_eq:NN \UWMadClassDate    \c_UWMad_ClassVersionDate_tl
%    \end{macrocode}
%
%   In an effort to allow the thesis class to adapt to new underlying classes,
%   the class that \UWMadClassName{} loads is decalred as a mutable
%   token list.  The default is the \LaTeX{} base class |report|.
%    \begin{macrocode}
\tl_new:N   \g_UWMad_ParentClass_tl
\tl_gset:Nn \g_UWMad_ParentClass_tl {report}
%    \end{macrocode}
%
%
%
%   \UWSubModule{Options}
%
%   First, a command is created to throw a warning if an option that
%   violates \UWMadLong{}'s dissertation guidelines.
%    \begin{macrocode}
\msg_new:nnn{UWMadThesis}{Options/StyleViolation}{
    Option~'#1'~violates~\c_UWMadUniversityShort_tl{}~
    Dissertation~Guidelines;~consider~omission
}
\cs_new:Nn \__UWMad_FrontMatter_StyleWarning:n {
    \msg_warning{UWMadThesis}{Options/StyleViolation}
        {#1}
   \PassOptionsToClass{#1}{\g_UWMad_ParentClass_tl}
}
%    \end{macrocode}
%
%   This command is used to suppress warning issued from
%   \UWMadClass{}. The first argument is a coonditional that
%   would normally determine if a warning were to be thrown, but the
%   decision is now superceeded by a switch to determine if warnings
%   are disabled or not.
%    \begin{macrocode}
\cs_new:Nn \__UWMad_ThrowWarnings:NTF {
    \bool_if:NTF \g__UWMad_ThrowWarnings_bool {
        \bool_if:NTF #1 {
            #2
        } {
            #3
        }
    } {}
}
\cs_new:Nn \__UWMad_ThrowWarnings:TF {
    \bool_if:NTF \g__UWMad_ThrowWarnings_bool {
        #1
    } {
        #2
    }
}
%    \end{macrocode}
%
%   Now, declare booleans for the option processing.  All new booleans
%   are false by default.
%    \begin{macrocode}
\bool_new:N       \g__UWMad_MathTweaks_bool
\bool_gset_true:N \g__UWMad_MathTweaks_bool
\bool_new:N       \g__UWMad_ThrowWarnings_bool
\bool_gset_true:N \g__UWMad_ThrowWarnings_bool
\bool_new:N       \g__UWMad_Hyperlinks_bool
\bool_gset_true:N \g__UWMad_Hyperlinks_bool
%    \end{macrocode}
%
%   Declare the options.
%    \begin{macrocode}
\DeclareOption{NoMath} {
    \bool_gset_false:N \g__UWMad_MathTweaks_bool
}
\DeclareOption{NoLinks} {
    \bool_gset_false:N \g__UWMad_Hyperlinks_bool
}
\DeclareOption{Quiet} {
    \bool_gset_false:N \g__UWMad_ThrowWarnings_bool
}
%    \end{macrocode}
%
%   Catch the couple of default options that violate the requirements:
%   8.5 by 11 paper for single-sided printing.
%    \begin{macrocode}
\DeclareOption{a4paper} {
    \__UWMad_ThrowWarnings:TF {
        \__UWMad_FrontMatter_StyleWarning:n {\CurrentOption}
    } { }
}
\DeclareOption{twoside} {
    \__UWMad_ThrowWarnings:TF {
        \__UWMad_FrontMatter_StyleWarning:n {\CurrentOption}
    } { }
}
%    \end{macrocode}
%
%   These options change the default report class to the
%   ones indicated.
%    \begin{macrocode}
\DeclareOption{article} {
    \tl_gset:Nn \g_UWMad_ParentClass_tl {article}
}
%    \end{macrocode}
%
%   This is a special class option for generating the documentation.
%   Users should not use this unless they know what they're doing.
%   The line below the |ParentClass| class prevents the \pkg{thumbpdf}
%   package from being loaded.
%    \begin{macrocode}
\DeclareOption{l3doc} {
    \tl_gset:Nn \g_UWMad_ParentClass_tl {l3doc}
    \tl_const:cn {ver@thumbpdf.sty} {}
}
%    \end{macrocode}
%
%   Pass all remaining options to the base class.
%    \begin{macrocode}
\DeclareOption*{
    \PassOptionsToClass
        {\CurrentOption}{\g_UWMad_ParentClass_tl}
}
%    \end{macrocode}
%
%   Process the options with some defaults and load the base class.
%    \begin{macrocode}
\ExecuteOptions{oneside,12pt}
\ProcessOptions\relax
\LoadClass{\g_UWMad_ParentClass_tl}[1995/12/01]
%    \end{macrocode}
%
%
%
%   \UWSubModule{Package Loads}
%   Load some packages that give nice features and are not
%   hyperlink sensitive.
%
%   If links were not negated by the options, \pkg{bookmark} and
%   \pkg{hyperref} are loaded.
%    \begin{macrocode}
\bool_if:NTF \g__UWMad_Hyperlinks_bool {
    \RequirePackage{hyperref}
    \RequirePackage{bookmark}
} {
    \DeclareDocumentCommand \href { o m m } {
        #3
    }
    \cs_gset_eq:NN \phantomsection \relax
}
%    \end{macrocode}
%
%    \begin{macrocode}
\RequirePackage{xparse}
\RequirePackage{fixltx2e}
\RequirePackage{array}
\RequirePackage{graphicx}
\RequirePackage{setspace}
\RequirePackage{geometry}
%    \end{macrocode}
%
%   If math was not negated by options, the \AmS{} suite is loaded.
%    \begin{macrocode}
\bool_if:NTF \g__UWMad_MathTweaks_bool {
    \RequirePackage{amsmath}
    \RequirePackage{amsfonts}
    \RequirePackage{amssymb}
    \RequirePackage{mathtools}
} { }
%    \end{macrocode}
%
%   And now we load some packages that give nice features and are
%   hyperlink sensitive.
%    \begin{macrocode}
\RequirePackage[noabbrev]{cleveref}
\RequirePackage[usenames,dvipsnames,svgnames,table,hyperref]{xcolor}
\RequirePackage{subfig}
\RequirePackage{caption}
%    \end{macrocode}
%
%   \iffalse
%</Code>
%   \fi
%   \iffalse
%<*Code>
%   \fi
%
%^^A
%^^A  Module Name: Programming
%^^A  Author:
%^^A    Name:           Troy C. Haskin
%^^A    E-mail:         UWMadThesis@hask.in
%^^A  Version:
%^^A    Number:         1.0
%^^A    Description:    Initial release
%^^A    Date:           06/01/2013
%^^A  Purpose:
%^^A    Provide a programming layer for the UW-Madison Thesis package
%^^A    Most of the module is designed to overcome the lack of such a
%^^A    layer in LaTeX2e.  Some LaTeX3 is being used for certain advanced
%^^A    features, and this module may become obsolete when/if it is upgraded
%^^A    to a pure LaTeX3 implementation; though a thin abatraction layer may
%^^A    still be desired.
%
%   \UWModule{Programming}
%   This section outlines the Programming module for the \UWMadClassName{}.
%   It encompasses thin abstractions from the standard \LaTeXPL{}'s type
%   and collection systems and provides \LaTeXe{} abstractions for
%   several other features.
%
%
%
%   \UWSubModule{Utility Commands}
%
%   Define some messages for the rest of the module.
%    \begin{macrocode}
\msg_new:nnn {UWMadThesis} {Programming/UnregisteredVariable} {
    `#1'~is~not~a~registered~#2.~~The~#2~must~be~defined~
    before~usage~by~the~function~\string\UWMad_#2_DefineLocal:n~or~
    \string\UWMad_#2_DefineGlobal:n.
}
\msg_new:nnn {UWMadThesis} {Programming/Undefined} {
    The~#2~`#1'~is~undefined.~~The~#2~must~be~defined~
    before~usage~by~the~function~\string\UWMad_#2_Define:n.
}
\msg_new:nnn {UWMadThesis} {Programming/Defined} {
    The~#2~`#1'~is~already~defined~and~will~not~altered.
}
%    \end{macrocode}
%
%
%
%   \begin{macro} {
%       \UWMad_Hook_Prepend:nn,
%       \UWMad_Hook_Append:nn}
%   These commands allow additional code to be prepended or appended to a
%   specified command.
%
%    \begin{macrocode}
\cs_new:Nn \UWMad_Hook_Prepend:nn{
    \cs_new_eq:cc  {#1-Default:} {#1}
    \cs_gset:cn    {#1:}         {#2 \cs:w #1-Default:\cs_end:}
    \cs_undefine:c {#1}
    \cs_new_eq:cc  {#1}          {#1:}
}
\cs_new:Nn \UWMad_Hook_Append:nn{
    \cs_new_eq:cc  {#1-Default:} {#1}
    \cs_gset:cn    {#1:}        {\cs:w #1-Default:\cs_end: #2}
    \cs_undefine:c {#1}
    \cs_new_eq:cc  {#1}          {#1:}
}
\cs_new:Nn \UWMad_Hook_Prepend:Nn{
    \cs_new_eq:cN  {\string#1-Default:} #1
    \cs_gset:cn    {\string#1:}         {#2 \cs:w\string#1-Default:\cs_end:}
    \cs_undefine:N  #1
    \cs_new_eq:Nc   #1          {\string#1:}
}
\cs_new:Nn \UWMad_Hook_Append:Nn{
    \cs_new_eq:cN  {\string#1-Default:} #1
    \cs_gset:cn    {\string#1:}         {\cs:w\string#1-Default:\cs_end: #2}
    \cs_undefine:N  #1
    \cs_new_eq:Nc   #1          {\string#1:}
}
%    \end{macrocode}
%   \end{macro}
%
%
%
%   \begin{macro} {
%       \UWMad_Definition_Swap:nn,
%       \UWMad_Definition_Reset:nn}
%   These commands \enquote{swap} in a new definition of a command and,
%   when called, reset it to it's default definition.
%
%    \begin{macrocode}
\cs_new:Nn \__UWMad_Definition_Swap:Nn {
    \cs_new_eq:cN  {\string#1-Default:}  #1
    \cs_new:cn     {\string#1-Swap:}    {#2}
    \cs_undefine:N  #1
    \cs_gset_eq:Nc  #1          {\string#1-Swap:}
}
\cs_new:Nn \__UWMad_Definition_Swap:cn {
    \cs_new_eq:cc  {#1-Default:} {#1}
    \cs_new:cn     {#1-Swap:}    {#2}
    \cs_undefine:c {#1}
    \cs_gset_eq:cc {#1}          {#1-Swap:}
}
\cs_new:Nn \UWMad_Definition_Swap:cn {
    \cs_if_exist:cTF {#1} {
        \__UWMad_Definition_Swap:cn{#1}{#2}

        \cs_if_exist:cTF {end#1} {
            \__UWMad_Definition_Swap:cn{end#1}{#2}
        }{}
    } {
        \cs_new:cn {#1} {#2}
    }
}
\cs_new:Nn \UWMad_Definition_Reset:c {
    \cs_if_exist:cTF {#1-Default:} {
        \cs_gset_eq:cc  {#1}          {#1-Default:}
        \cs_undefine:c  {#1-Default:}

        \cs_if_exist:cTF {end#1-Default:} {
            \cs_gset_eq:cc  {#1}          {end#1-Default:}
            \cs_undefine:c  {end#1-Default:}
        }{}
    } {}
}
\cs_new:Nn \UWMad_Definition_Swap:Nn {
    \cs_if_exist:NTF #1 {
        \__UWMad_Definition_Swap:Nn #1 {#2}
    } {
        \cs_new:Nn #1 {#2}
    }
}
\cs_new:Nn \UWMad_Definition_Reset:N {
    \cs_if_exist:cTF {\string#1-Default:} {
        \cs_gset_eq:Nc  #1  {\string#1-Default:}
        \cs_undefine:c      {\string#1-Default:}
    } {}
}
%    \end{macrocode}
%   \end{macro}
%
%
%
%   \begin{macro}[internal]{
%       \__UWMad_IfDefined:nnnnT,
%       \__UWMad_IfUndefined:nnnnT}
%   These commands accept a \marg{Prefix}, an \marg{ID}, a \marg{Suffix},
%   a \marg{Type}, and \marg{Code}.  It determines if a command named by the
%   concatenation of \marg{Prefix}, \marg{ID}, and \marg{Suffix}
%   is defined or not and executes \marg{Code} depending on the existence.
%
%   \begin{Usage}
%       \item |\__UWMad_IfUndefined:nnnnT|
%           \marg{Prefix}\marg{ID}\marg{Suffix}\marg{Type}\marg{Code}
%   \end{Usage}
%
%    \begin{macrocode}
\cs_new:Nn \__UWMad_IfDefined:nnnnT{
    \cs_if_exist:cTF {#1#2#3} {
        #5
    }{
            \msg_error:nnnn
                {UWMadThesis}
                {Programming/Undefined}
                {#2}
                {#4}
    }
}
\cs_new:Nn \__UWMad_IfUndefined:nnnnT{
    \cs_if_free:cTF {#1#2#3} {
        #5
    }{
            \msg_warning:nnnn
                {UWMadThesis}
                {Programming/Defined}
                {#2}
                {#4}
    }
}
%    \end{macrocode}
%   \end{macro}
%
%
%
%   \begin{macro}[internal]{
%       \__UWMad_IfDefined:nT,
%       \__UWMad_IfUndefined:nT}
%   These commands are simplifications of the above commands and
%   that only take a \marg{CommandName} and \marg{TrueCode}.
%
%   \begin{Usage}
%       \item |\__UWMad_IfUndefined:nT|
%           \marg{CommandName}\marg{TrueCode}
%   \end{Usage}
%
%    \begin{macrocode}
\cs_new:Nn \__UWMad_IfDefined:nT{
    \_UWMad_IfDefined:nnnnT{#1}{}{}{command}{#2}
}
\cs_new:Nn \__UWMad_IfUndefined:nT{
    \_UWMad_IfUndefined:nnnnT{#1}{}{}{command}{#2}
}
%    \end{macrocode}
%   \end{macro}
%
%
%
%
%
%^^A ==================================================================== %
%^^A                        Core Programming Systems                      %
%^^A ==================================================================== %
%   \UWSubModule{Type System}
%
%   The commands in this section form the core of the programming module.
%   All of the Core systems are written using the \LaTeXPL and have
%   extremely long names.  Most of the commands as thin abstractions from
%   the already-written \LaTeXPL modules and are designed to provide
%   a more stream-lined and robust environment in providing
%   useful warnings and errors where needed.
%
%   One of the main features added by the core programming system is more
%   transparent local-global handling.  The onus of remembering which
%   vairables of local-global is all that's needed with the commands
%   for altering them being all of the same.
%
%
%
%   \begin{macro}{\__UWMad_IfLocal:nnnTF}
%   Certain subsystems of the Programming Module make a distinction
%   between local and global variables where scope is determined by
%   \TeX{} groups.  This command takes five arguments designed to
%   increase maintainability and readability in the subsystems that
%   use it.
%
%   This command accepts a \marg{Prefix}, an \marg{ID}, a \marg{Type},
%   \marg{LocalCode}, and \marg{GlobalCode}.  All subsystems that use
%   this command have (in theory) already defined a command that concatenates
%   the \marg{Prefix}, \marg{ID}, and |_Local| or |_Global|
%   of a specific \marg{Type}.  If either of the commands is defined, the
%   appropriate code is executed (in an fully expandable fashion, which is
%   used to reduce code duplication where possible).  If neither of those
%   commands exist, the variable is not registered with the system and an
%   exception is thrown.
%
%   \begin{Usage}
%       \item |\__UWMad_IfLocal:nnnTF| \marg{Prefix}\marg{ID}\marg{Type}
%                               \marg{LocalCode}, and \marg{GLobalCode}
%   \end{Usage}
%
%    \begin{macrocode}
\cs_new:Nn \__UWMad_IfLocal:nnnTF {
    \cs_if_exist:cTF     {#1#2_Local}{
        #4
    }{
        \cs_if_exist:cTF {#1#2_Global}{
            #5
        }{
            \msg_error:nnnn
                {UWMadThesis}
                {Programming/UnregisteredVariable}
                {#2}
                {#3}
        }
    }
}
%    \end{macrocode}
%   \end{macro}
%
%
%
%
%
%
%^^A ==================================================================== %
%^^A                            Boolean System                            %
%^^A ==================================================================== %
%
%   \UWSubSubModule{Boolean System}
%   This subsystem was made to give a \LaTeX{}-like branching system that
%   can create both local and global switches.
%
%   The system is a thin abstraction of
%   \LaTeXPL's |bool| module in the |l3prg| package to avoid developing
%   a one-shot system while allowing more endeavouring authors access to
%   to the simple feature without learning \LaTeX3{} programming.
%
%
%
%   \begin{macro}[internal]{
%       \__UWMad_Boolean_IfLocal:nTF,
%       \__UWMad_Boolean_IfUndefined:nnT}
%   These commands are shortcuts to the more general commands
%   outlined above.
%
%    \begin{macrocode}
\cs_new:Nn \__UWMad_Boolean_IfLocal:nTF {
    \__UWMad_IfLocal:nnnTF
        {g__UWMad_Boolean_}{#1}{Boolean}{#2}{#3}
}
\cs_new:Nn \__UWMad_Boolean_IfUndefined:nnT{
    \__UWMad_IfUndefined:nnnnT
        {g__UWMad_Boolean_}{#1}{#2}{Boolean}{#3}
}
%    \end{macrocode}
%   \end{macro}
%
%
%   \begin{macro}{
%       \UWMad_Boolean_DefineLocal:n,
%       \UWMad_Boolean_DefineGlobal:n}
%   Create a new Boolean with local or global scope.
%
%    \begin{macrocode}
\cs_new:Nn \UWMad_Boolean_DefineLocal:n {
    \__UWMad_Boolean_IfUndefined:nnT {#1} {_Local} {
        \bool_new:c {g__UWMad_Boolean_#1_Local}
    }
}
\cs_new:Nn \UWMad_Boolean_DefineGlobal:n {
    \__UWMad_Boolean_IfUndefined:nnT {#1} {_Global} {
        \bool_new:c {g__UWMad_Boolean_#1_Global}
    }
}
%    \end{macrocode}
%   \end{macro}
%
%   \begin{macro}{
%       \UWMad_Boolean_DefineLocalSetTrue:n,
%       \UWMad_Boolean_DefineLocalSetFalse:n}
%   Define a new  local Boolean with an explicit initial state.
%
%    \begin{macrocode}
\cs_new:Nn \UWMad_Boolean_DefineLocalSetTrue:n {
    \__UWMad_Boolean_IfUndefined:nnT {#1} {_Local} {
        \bool_new:c       {g__UWMad_Boolean_#1_Local}
        \bool_gset_true:c {g__UWMad_Boolean_#1_Local}
    }
}
\cs_new:Nn \UWMad_Boolean_DefineLocalSetFalse:n {
    \__UWMad_Boolean_IfUndefined:nnT {#1} {_Local} {
        \bool_new:c        {g__UWMad_Boolean_#1_Local}
        \bool_gset_false:c {g__UWMad_Boolean_#1_Local}
    }
}
%    \end{macrocode}
%    \end{macro}
%
%
%   \begin{macro}{\UWMad_Boolean_Delete:n}
%   Delete a defined boolean.
%
%    \begin{macrocode}
\cs_new:Nn \UWMad_Boolean_Delete:n {
    \__UWMad_Boolean_IfLocal:nTF {#1} {
        \cs_undefine:c {g__UWMad_Boolean_#1_Local}
    } {
        \cs_undefine:c {g__UWMad_Boolean_#1_Global}
    }
}
%    \end{macrocode}
%   \end{macro}
%   \begin{macro}{
%       \UWMad_Boolean_DefineGlobalSetTrue:n,
%       \UWMad_Boolean_DefineGlobalSetFalse:n}
%   Define a new  global Boolean with an explicit initial state.
%
%    \begin{macrocode}
\cs_new:Nn \UWMad_Boolean_DefineGlobalSetTrue:n {
    \__UWMad_Boolean_IfUndefined:nnT {#1} {_Global} {
        \bool_new:c       {g__UWMad_Boolean_#1_Global}
        \bool_gset_true:c {g__UWMad_Boolean_#1_Global}
    }
}
\cs_new:Nn \UWMad_Boolean_DefineGlobalSetFalse:n {
    \__UWMad_Boolean_IfUndefined:nnT {#1} {_Global} {
        \bool_new:c        {g__UWMad_Boolean_#1_Global}
        \bool_gset_false:c {g__UWMad_Boolean_#1_Global}
    }
}
%    \end{macrocode}
%    \end{macro}
%
%
%   \begin{macro}{
%       \UWMad_Boolean_SetTrue:n,
%       \UWMad_Boolean_SetFalse:n}
%   Set any defined Boolean to true or false.
%   Local-global assignment is handled by the Boolean's initial
%   definition.  If a local and global Boolean exist with the
%   same name, a local assignment is carried out.
%
%    \begin{macrocode}
\cs_new:Nn \UWMad_Boolean_SetTrue:n {
    \__UWMad_Boolean_IfLocal:nTF {#1} {
        \bool_set_true:c  {g__UWMad_Boolean_#1_Local}
    }{
        \bool_gset_true:c {g__UWMad_Boolean_#1_Global}
    }
}
\cs_new:Nn \UWMad_Boolean_SetFalse:n {
    \__UWMad_Boolean_IfLocal:nTF {#1} {
        \bool_set_false:c  {g__UWMad_Boolean_#1_Local}
    }{
        \bool_gset_false:c {g__UWMad_Boolean_#1_Global}
    }
}
%    \end{macrocode}
%    \end{macro}
%
%
%   \begin{macro}{
%       \UWMad_Boolean_IfTrue:n,
%       \UWMad_Boolean_IfFalse:n}
%   Conditionally execute code depending on the current state
%   of the Boolean.
%
%    \begin{macrocode}
\cs_new:Nn \UWMad_Boolean_IfTrue:nTF {
    \__UWMad_Boolean_IfLocal:nTF {#1} {
        \bool_if:cTF {g__UWMad_Boolean_#1_Local}
    }{
        \bool_if:cTF {g__UWMad_Boolean_#1_Global}
    }
    {#2}
    {#3}
}
\cs_new:Nn \UWMad_Boolean_IfFalse:nTF {
    \__UWMad_Boolean_IfLocal:nTF {#1} {
        \bool_if:cTF {g__UWMad_Boolean_#1_Local}
    }{
        \bool_if:cTF {g__UWMad_Boolean_#1_Global}
    }
    {#3}
    {#2}
}
%    \end{macrocode}
%    \end{macro}
%
%
%
%
%
%^^A ==================================================================== %
%^^A                            Length System                             %
%^^A ==================================================================== %
%
%   \UWSubSubModule{Length System}
%   This subsystem was made to give a \LaTeX{}-like length system that
%   can create both local and global lengths.
%
%   The system is a thin abstraction of
%   \LaTeXPL's |dim| module in the |l3skip| package to avoid developing
%   a one-shot system while allowing more endeavouring authors access to
%   to the simple feature without learning \LaTeX3{} programming.
%
%
%
%   \begin{macro}[internal]{
%       \__UWMad_Length_IfLocal:nTF,
%       \__UWMad_Length_IfUndefined:nnT}
%   These commands are shortcuts to the more general
%   commands outlines above.
%
%    \begin{macrocode}
\cs_new:Nn \__UWMad_Length_IfLocal:nTF {
    \__UWMad_IfLocal:nnnTF
        {g__UWMad_Length_}{#1}{Length}{#2}{#3}
}
\cs_new:Nn \__UWMad_Length_IfUndefined:nnT{
    \__UWMad_IfUndefined:nnnnT{g__UWMad_Length_}{#1}{#2}{Length}{#3}
}
%    \end{macrocode}
%    \end{macro}
%
%
%   \begin{macro}{
%       \UWMad_Length_DefineLocal:nn,
%       \UWMad_Length_DefineGlobal:nn}
%   Define a new Length with an explicit initial value having either
%   local or global scope.
%
%    \begin{macrocode}
\cs_new:Nn \UWMad_Length_DefineLocal:nn {
    \__UWMad_Length_IfUndefined:nnT {#1} {_Local} {
        \dim_new:c   {g__UWMad_Length_#1_Local}
        \dim_gset:cn {g__UWMad_Length_#1_Local} {#2}
    }
}
\cs_new:Nn \UWMad_Length_DefineGlobal:nn {
    \__UWMad_Length_IfUndefined:nnT {#1} {_Global} {
        \dim_new:c   {g__UWMad_Length_#1_Global}
        \dim_gset:cn {g__UWMad_Length_#1_Global} {#2}
    }
}
%    \end{macrocode}
%    \end{macro}
%
%
%   \begin{macro}{\UWMad_Length_Add:nn}
%   Add a dimension to a defined Length.
%   Local-global assignment is handled by the Length's initial
%   definition.  If a local and global Length exist with the
%   same name, a local assignment is carried out.
%
%    \begin{macrocode}
\cs_new:Nn \UWMad_Length_Add:nn {
    \__UWMad_Length_IfLocal:nTF {#1} {
        \dim_add:cn  {g__UWMad_Length_#1_Local}  {#2}
    }{
        \dim_gadd:cn {g__UWMad_Length_#1_Global} {#2}
    }
}
%    \end{macrocode}
%    \end{macro}
%
%
%   \begin{macro}{\UWMad_Length_Set:nn}
%   Set the dimension of a defined Length.
%   Local-global assignment is handled by the Length's initial
%   definition.  If a local and global Length exist with the
%   same name, a local assignment is carried out.
%
%    \begin{macrocode}
\cs_new:Nn \UWMad_Length_Set:nn {
    \__UWMad_Length_IfLocal:nTF {#1} {
        \dim_set:cn  {g__UWMad_Length_#1_Local}  {#2}
    }{
        \dim_gset:cn {g__UWMad_Length_#1_Global} {#2}
    }
}
%    \end{macrocode}
%    \end{macro}
%
%
%   \begin{macro}{\UWMad_Length_Of:n}
%   Retreive the value of a defined Length.
%
%    \begin{macrocode}
\cs_new:Nn \UWMad_Length_Of:n {
    \__UWMad_Length_IfLocal:nTF {#1} {
        \dim_use:c {g__UWMad_Length_#1_Local}
    }{
        \dim_use:c {g__UWMad_Length_#1_Global}
    }
}
%    \end{macrocode}
%    \end{macro}
%
%
%   \begin{macro}{\UWMad_Length_If:nnnTF}
%   Conditionally execute true/false code based  on the
%   value of a defined Length given an operator and second
%   operand.
%
%   \begin{Usage}
%       \item |\UWMad_Length_If:nnnTF|
%               \marg{LengthID}\marg{Operator}\marg{Operand}
%                   \marg{True}\marg{False}
%   \end{Usage}
%
%    \begin{macrocode}
\cs_new:Nn \UWMad_Length_If:nnnTF {
    \dim_compare:nTF{ \UWMad_Length_Of:n{#1} #2 #3 }{
        #4
    }{
        #5
    }
}
%    \end{macrocode}
%    \end{macro}
%
%
%
%
%
%^^A ==================================================================== %
%^^A                            Counter System                            %
%^^A ==================================================================== %
%
%   \UWSubSubModule{Counter System}
%   This subsystem was made to give a \LaTeX{}-like counter system that
%   can create both local and global counters.
%
%
%   \begin{macro}[internal]{
%       \__UWMad_Counter_IfLocal:nTF,
%       \__UWMad_Counter_IfUndefined:nTF}
%   These commands are shortcuts to the more general
%   commands outlines above.
%
%    \begin{macrocode}
\cs_new:Nn \__UWMad_Counter_IfLocal:nTF {
    \__UWMad_IfLocal:nnnTF
        {g__UWMad_Counter_}{#1}{Counter}{#2}{#3}
}
\cs_new:Nn \__UWMad_Counter_IfUndefined:nnT{
    \__UWMad_IfUndefined:nnnnT{g__UWMad_Counter_}{#1}{#2}{Counter}{#3}
}
%    \end{macrocode}
%   \end{macro}
%
%
%
%   \begin{macro}{
%       \UWMad_Counter_DefineLocal:nn,
%       \UWMad_Counter_DefineGlobal:nn}
%   This pair creates either a local or global counter named
%   \marg{Counter Name} with \marg{Initial Value}. The counters
%   are registered, defined to be local or global, initialized
%   by |\newcount|, and set to \marg{Initial Value}.
%
%   \begin{Usage}
%       \item |\UWMad_Counter_DefineLocal:nn|\marg{Counter Name}{Initial Value}
%       \item |\UWMad_Counter_DefineGlobal:nn|\marg{Counter Name}{Initial Value}
%   \end{Usage}
%
%    \begin{macrocode}
\cs_new:Nn \UWMad_Counter_DefineLocal:nn {
    \__UWMad_Counter_IfUndefined:nnT {#1} {_Local} {
        \int_new:c   {g__UWMad_Counter_#1_Local}
        \int_gset:cn {g__UWMad_Counter_#1_Local} {#2}
    }
}
\cs_new:Nn \UWMad_Counter_DefineGlobal:nn {
    \__UWMad_Counter_IfUndefined:nnT {#1} {Global} {
        \int_new:c   {g__UWMad_Counter_#1_Global}
        \int_gset:cn {g__UWMad_Counter_#1_Global} {#2}
    }
}
%    \end{macrocode}
%   \end{macro}
%
%
%
%   \begin{macro}{\UWMad_Counter_Add:nn}
%   This command adds \marg{Increment} to the current value of counter
%   \marg{CounterName}. Local vs. global advancement is set at definition
%   and is handled transparently.
%
%   \begin{Usage}
%       \item |\UWMad_Counter_Add:nn|\marg{CounterName}\marg{Increment}
%   \end{Usage}
%
%    \begin{macrocode}
\cs_new:Nn \UWMad_Counter_Add:nn {
    \__UWMad_Counter_IfLocal:nTF {#1} {
        \int_add:cn  {g__UWMad_Counter_#1_Local}  {#2}
    }{
        \int_gadd:cn {g__UWMad_Counter_#1_Global} {#2}
    }
}
%    \end{macrocode}
%   \end{macro}
%
%
%
%   \begin{macro}{\UWMad_Counter_Step:n}
%   Adds $1$ to the counter \marg{Counter Name}.
%   |\UWMad_IsLocal:nnn| handles the local vs. global advancement.
%
%   \begin{Usage}
%       \item |\UWMad_Counter_Step:n|\marg{Counter Name}
%   \end{Usage}
%
%    \begin{macrocode}
\cs_new:Nn \UWMad_Counter_Step:n {
    \UWMad_Counter_Add:nn{#1}{1}
}
%    \end{macrocode}
%   \end{macro}
%
%
%
%   \begin{macro}{\UWMad_Counter_Set:nn}
%   This command sets the value of counter \marg{Counter Name} to
%   \marg{Value}.
%   |\UWMad_IsLocal:nnn| handles the local vs. global assignment.
%
%   \begin{Usage}
%       \item |\UWMad_Counter_Set:nn|\marg{Counter Name}\marg{Value}
%   \end{Usage}
%
%    \begin{macrocode}
\cs_new:Nn \UWMad_Counter_Set:nn {
    \__UWMad_Counter_IfLocal:nTF {#1} {
        \int_set:cn  {g__UWMad_Counter_#1_Local}  {#2}
    }{
        \int_gset:cn {g__UWMad_Counter_#1_Global} {#2}
    }
}
%    \end{macrocode}
%   \end{macro}
%
%
%
%   \begin{macro}{
%       \UWMad_Counter_SetAndAdd:nnn,
%       \UWMad_Counter_SetAndStep:nn}
%   Combinations of |\SetCounter|, |AddToCounter|, and |\StepCounter|.
%
%   \begin{Usage}
%       \item |\UWMad_Counter_SetAndAdd:nnn|\marg{Counter Name}
%                   \marg{Initial Value}\marg{Value}
%       \item |\UWMad_Counter_SetAndStep:nn|\marg{Counter Name}
%                   \marg{Initial Value}
%   \end{Usage}
%
%    \begin{macrocode}
\cs_new:Nn \UWMad_Counter_SetAndAdd:nnn {
    \UWMad_Counter_Set:nn{#1}{#2}
    \UWMad_Counter_Add:nn{#1}{#3}
}
\cs_new:Nn \UWMad_Counter_SetAndStep:nn {
    \UWMad_Counter_SetAndAdd:nnn {#1}{#2}{1}
}
%    \end{macrocode}
%   \end{macro}
%
%
%
%   \begin{macro}{\UWMad_Counter_Value:n}
%   Combinations of |\SetCounter|, |AddToCounter|, and |\StepCounter|.
%
%   \begin{Usage}
%       \item |\UWMad_Counter_Value:n|\marg{Counter Name}
%   \end{Usage}
%
%    \begin{macrocode}
\cs_new:Nn \UWMad_Counter_Value:n {
    \__UWMad_Counter_IfLocal:nTF {#1} {
        \int_use:c {g__UWMad_Counter_#1_Local}
    }{
        \int_use:c {g__UWMad_Counter_#1_Global}
    }
}
%    \end{macrocode}
%   \end{macro}
%
%
%
%   \begin{macro}{\UWMad_Counter_Compare:nnnTF}
%
%   \begin{Usage}
%       \item |\UWMad_Counter_Compare:nnnTF| {}{}{}{}{}
%   \end{Usage}
%
%    \begin{macrocode}
\cs_new:Nn \UWMad_Counter_Compare:nnnTF {
    \int_compare:nTF {\UWMad_Counter_Value:n{#1} #2 #3} {
        #4
    }{
        #5
    }
}
%    \end{macrocode}
%   \end{macro}
%
%
%
%   \begin{macro}{
%       \UWMad_Counter_arabic:n,
%       \UWMad_Counter_alpha:n,
%       \UWMad_Counter_Alpha:n,
%       \UWMad_Counter_roman:n,
%       \UWMad_Counter_Roman:n,}
%   This set of functions converts the counter \marg{CounterName}
%   into a formatted number.
%
%   \begin{Usage}
%       \item |\UWMad_Counter_arabic:n| \marg{CounterName}
%   \end{Usage}
%
%    \begin{macrocode}
\cs_new:Nn \UWMad_Counter_arabic:n {
    \int_to_arabic:n {\UWMad_Counter_Value:n{#1}}
}
\cs_new:Nn \UWMad_Counter_alpha:n {
    \int_to_alph:n {\UWMad_Counter_Value:n{#1}}
}
\cs_new:Nn \UWMad_Counter_Alpha:n {
    \int_to_Alph:n {\UWMad_Counter_Value:n{#1}}
}
\cs_new:Nn \UWMad_Counter_roman:n {
    \int_to_roman:n {\UWMad_Counter_Value:n{#1}}
}
\cs_new:Nn \UWMad_Counter_Roman:n {
    \int_to_Roman:n {\UWMad_Counter_Value:n{#1}}
}
%    \end{macrocode}
%   \end{macro}
%
%
%
%
%
%
%^^A ======================================================================= %
%^^A                     Collection Creation Commands                        %
%^^A ======================================================================= %
%
%   \UWSubModule{Collections}
%   In the following subsections, commands that create and manipulate
%   various collection data types will be discussed.  The collections
%   currently implemented are comma-separate values (CSVs), stacks (LIFO),
%   queues (FIFO), deques (LIFO+FIFO), and hashes (key-value pairs).
%
%   All of the collection systems are thin abstractions of \LaTeXPL{}'s
%   |l3clist|, |l3tl|, |l3seq|, and |l3prop| modules to avoid developing
%   one-shot systems while allowing more endeavoring authors access to the
%   features without learning \LaTeX3{} programming if they load the
%   abstractions.
%
%
%
%^^A ======================================================================= %
%^^A                          CSV Creation Commands                          %
%^^A ======================================================================= %
%
%   \UWSubSubModule{CSVs}
%   This set of commands is a simple system for comma-separated value (CSV)
%   list creation.  It consists of only a few functions: initialize list,
%   append value, prepend value, get list, and erase list.
%
%   This feature was created solely to export |hyperref| meta-data to an
%   external file for later reading.  The system is a thin abstraction of
%   \LaTeXPL's |l3clist| package to avoid developing
%   a one-shot system while allowing more endeavouring authors access to
%   to the simple feature without learning \LaTeX3{} programming.
%
%
%   \begin{macro}[internal]{
%       \__UWMad_CSV_IfDefined:nT,
%       \__UWMad_CSV_IfUndefined:nT}
%   Shortcuts for the more general commands outlined above.
%
%    \begin{macrocode}
\cs_new:Nn \__UWMad_CSV_IfDefined:nT {
    \__UWMad_IfDefined:nnnnT{g__UWMad_CSV_}{#1}{}{CSV}{#2}
}
\cs_new:Nn \__UWMad_CSV_IfUndefined:nT{
    \__UWMad_IfUndefined:nnnnT{g__UWMad_CSV_}{#1}{}{CSV}{#2}
}
%    \end{macrocode}
%   \end{macro}
%
%
%   \begin{macro}{\UWMad_CSV_Define:n}
%   Define a new CSV.
%
%    \begin{macrocode}
\cs_new:Nn \UWMad_CSV_Define:n {
    \__UWMad_CSV_IfUndefined:nT {#1} {
        \clist_new:c {g__UWMad_CSV_#1}
    }
}
%    \end{macrocode}
%   \end{macro}
%
%
%   \begin{macro}{\UWMad_CSV_Clear:n}
%   Define a clear a defined CSV but do not undefine.
%
%    \begin{macrocode}
\cs_new:Nn \UWMad_CSV_Clear:n {
    \__UWMad_CSV_IfDefined:nT {#1} {
        \clist_gclear:c {g__UWMad_CSV_#1}
    }
}
%    \end{macrocode}
%   \end{macro}
%
%
%   \begin{macro}{\UWMad_CSV_Delete:n}
%   Define a clear and undefine a defined CSV.
%
%    \begin{macrocode}
\cs_new:Nn \UWMad_CSV_Delete:n {
    \__UWMad_CSV_IfDefined:nT {#1} {
        \clist_gclear:c {g__UWMad_CSV_#1}
        \cs_undefine:c  {g__UWMad_CSV_#1}
    }
}
%    \end{macrocode}
%   \end{macro}
%
%
%   \begin{macro}{\UWMad_CSV_Append:nn}
%   Push a value on to the right side of a defined CSV.
%
%    \begin{macrocode}
\cs_new:Nn \UWMad_CSV_Append:nn {
    \__UWMad_CSV_IfDefined:nT {#1} {
        \clist_gput_right:cn {g__UWMad_CSV_#1} {#2}
    }
}
%    \end{macrocode}
%   \end{macro}
%
%
%   \begin{macro}{\UWMad_CSV_Prepend:nn}
%   Push a value on to the left side of a defined CSV.
%
%    \begin{macrocode}
\cs_new:Nn \UWMad_CSV_Prepend:nn {
    \__UWMad_CSV_IfDefined:nT {#1} {
        \clist_gput_left:cn {g__UWMad_CSV_#1} {#2}
    }
}
%    \end{macrocode}
%   \end{macro}
%
%
%   \begin{macro}{\UWMad_CSV_Get:n}
%   Return a defined CSV to the input stream.
%
%    \begin{macrocode}
\cs_new:Nn \UWMad_CSV_Get:n {
    \__UWMad_CSV_IfDefined:nT {#1} {
        \use:c {g__UWMad_CSV_#1}
    }
}
%    \end{macrocode}
%   \end{macro}
%
%
%   \begin{macro}{\UWMad_CSV_IfNotEmpty:nTF}
%   Execute true/false code depending on if the
%   defined CSV is empty or not.
%
%    \begin{macrocode}
\cs_new:Nn \UWMad_CSV_IfNotEmpty:nTF {
    \__UWMad_CSV_IfDefined:nT {#1} {
        \clist_if_empty:cTF {g__UWMad_CSV_#1} {
            #3
        }{
            #2
        }
    }
}
%    \end{macrocode}
%    \end{macro}
%
%
%
%
%
%
%^^A ======================================================================= %
%^^A                          Stack Creation Commands                        %
%^^A ======================================================================= %
%
%   \UWSubSubModule{Stacks}
%   This set of commands is a simple system for creating and working with
%   stacks.  Stacks are a last-in first-out collection data type; this means
%   that the data element (in this any unexpanded token/token list) last
%   pushed on to the stack is the first popped.  Data elements can also be
%   walked (iterated over) with an inline callback in a LIFO sense.
%
%
%
%   \begin{macro}[internal]{
%       \__UWMad_Stack_IfDefined:nT,
%       \__UWMad_Stack_IfUndefined:nT}
%   Shortcuts for the more general commands outlined above.
%
%    \begin{macrocode}
\cs_new:Nn \__UWMad_Stack_IfDefined:nT {
    \__UWMad_IfDefined:nnnnT{g__UWMad_Stack_}{#1}{}{Stack}{#2}
}
\cs_new:Nn \__UWMad_Stack_IfUndefined:nT{
    \__UWMad_IfUndefined:nnnnT{g__UWMad_Stack_}{#1}{}{Stack}{#2}
}
%    \end{macrocode}
%   \end{macro}
%
%
%   \begin{macro}{\UWMad_Stack_Define:n}
%   Define a new Stack.
%
%    \begin{macrocode}
\cs_new:Nn \UWMad_Stack_Define:n {
    \__UWMad_Stack_IfUndefined:nT {#1} {
        \tl_new:c {g__UWMad_Stack_#1}
    }
}
%    \end{macrocode}
%   \end{macro}
%
%
%   \begin{macro}{\UWMad_Stack_Clear:n}
%   Clear but do not undefine a defined Stack.
%
%    \begin{macrocode}
\cs_new:Nn \UWMad_Stack_Clear:n {
    \__UWMad_Stack_IfDefined:nT {#1} {
        \tl_gclear:c   {g__UWMad_Stack_#1}
    }
}
%    \end{macrocode}
%   \end{macro}
%
%
%   \begin{macro}{\UWMad_Stack_Delete:n}
%   Clear and undefine a defined Stack.
%
%    \begin{macrocode}
\cs_new:Nn \UWMad_Stack_Delete:n {
    \__UWMad_Stack_IfDefined:nT {#1} {
        \tl_gclear:c   {g__UWMad_Stack_#1}
        \cs_undefine:c {g__UWMad_Stack_#1}
    }
}
%    \end{macrocode}
%   \end{macro}
%
%
%   \begin{macro}{\UWMad_Stack_Push:nn}
%   Push a value on to a defined Stack.
%
%    \begin{macrocode}
\cs_new:Nn \UWMad_Stack_Push:nn {
    \__UWMad_Stack_IfDefined:nT {#1} {
        \tl_gput_left:cn {g__UWMad_Stack_#1} {#2}
    }
}
%
%
\cs_generate_variant:Nn \tl_head:N { c }
\cs_generate_variant:Nn \tl_tail:N { c }
%    \end{macrocode}
%   \end{macro}
%
%
%   \begin{macro}{\UWMad_Stack_Pop:n}
%   Pop a value off a defined Stack and place it in the
%   input stream.
%
%    \begin{macrocode}
\cs_new:Nn \UWMad_Stack_Pop:n {
    \__UWMad_Stack_IfDefined:nT {#1} {
        \tl_set:Nf \l_tmpa_tl          {\tl_head:c {g__UWMad_Stack_#1}}
        \tl_set:cf {g__UWMad_Stack_#1} {\tl_tail:c {g__UWMad_Stack_#1}}
        \tl_use:N \l_tmpa_tl
    }
}
%    \end{macrocode}
%   \end{macro}
%
%
%   \begin{macro}{\UWMad_Stack_Walk:nn}
%   Iterate of the elements of a defined Stack in a FILO sense
%   with supplied code.
%
%    \begin{macrocode}
\cs_new:Nn \UWMad_Stack_Walk:nn {
    \tl_map_inline:cn {g__UWMad_Stack_#1} {#2}
}
%    \end{macrocode}
%    \end{macro}
%
%
%
%^^A ======================================================================= %
%^^A                          Queue Creation Commands                        %
%^^A ======================================================================= %
%
%   \UWSubSubModule{Queues}
%   This set of commands is a simple system for creating and working with
%   queue.  Queues are a first-in first-out collection data type; this means
%   that the data element (in this any unexpanded token/token list) first
%   pushed on to the queue is the first popped.  Data elements can also be
%   walked (iterated over) with an inline callback in a FIFO sense.
%
%
%
%   \begin{macro}[internal]{
%       \__UWMad_Queue_IfDefined:nT,
%       \__UWMad_Queue_IfUndefined:nT}
%   Shortcuts for the more general commands outlined above.
%
%    \begin{macrocode}
\cs_new:Nn \__UWMad_Queue_IfDefined:nT {
    \__UWMad_IfDefined:nnnnT{g__UWMad_Queue_}{#1}{}{Queue}{#2}
}
\cs_new:Nn \__UWMad_Queue_IfUndefined:nT{
    \__UWMad_IfUndefined:nnnnT{g__UWMad_Queue_}{#1}{}{Queue}{#2}
}
%    \end{macrocode}
%   \end{macro}
%
%
%   \begin{macro}{\UWMad_Queue_Define:n}
%   Define a new Queue.
%
%    \begin{macrocode}
\cs_new:Nn \UWMad_Queue_Define:n {
    \__UWMad_Queue_IfUndefined:nT {#1} {
        \tl_new:c {g__UWMad_Queue_#1}
    }
}
%    \end{macrocode}
%   \end{macro}
%
%
%   \begin{macro}{\UWMad_Queue_Clear:n}
%   Clear but do not undefine a defined Queue.
%
%    \begin{macrocode}
\cs_new:Nn \UWMad_Queue_Clear:n {
    \__UWMad_Queue_IfDefined:nT {#1} {
        \tl_gclear:c {g__UWMad_Queue_#1}
    }
}
%    \end{macrocode}
%   \end{macro}
%
%
%   \begin{macro}{\UWMad_Queue_Delete:n}
%   Clear and undefine a defined Queue.
%
%    \begin{macrocode}
\cs_new:Nn \UWMad_Queue_Delete:n {
    \__UWMad_Queue_IfDefined:nT {#1} {
        \tl_gclear:c    {g__UWMad_Queue_#1}
         \cs_undefine:c {g__UWMad_Queue_#1}
    }
}
%    \end{macrocode}
%   \end{macro}
%
%
%   \begin{macro}{\UWMad_Queue_Pop:nn}
%   Push an item on to the start of a defined Queue.
%
%    \begin{macrocode}
\cs_new:Nn \UWMad_Queue_Push:nn {
    \__UWMad_Queue_IfDefined:nT {#1} {
        \tl_gput_left:cn {g__UWMad_Queue_#1} {{#2}}
    }
}
%
%
\cs_generate_variant:Nn \tl_head:N { c }
\cs_generate_variant:Nn \tl_tail:N { c }
%    \end{macrocode}
%   \end{macro}
%
%
%   \begin{macro}{\UWMad_Queue_Pop:n}
%   Pop an item from the end of a defined Queue
%   and place it in the input stream.
%
%    \begin{macrocode}
\cs_new:Nn \UWMad_Queue_Pop:n {
    \__UWMad_Queue_IfDefined:nT {#1} {
        \tl_reverse:c   {g__UWMad_Queue_#1}
        \tl_set:Nf \l_tmpa_tl
            {\tl_head:c {g__UWMad_Queue_#1}}
        \tl_set:cf      {g__UWMad_Queue_#1}
            {\tl_tail:c {g__UWMad_Queue_#1}}
        \tl_reverse:c   {g__UWMad_Queue_#1}
        \tl_use:N \l_tmpa_tl
    }
}
%    \end{macrocode}
%   \end{macro}
%
%
%   \begin{macro}{\UWMad_Queue_Walk:nn}
%   Iterate of the elements of a defined Queue in a FIFO sense
%   with supplied code.
%
%    \begin{macrocode}
\cs_new:Nn \UWMad_Queue_Walk:nn {
    \__UWMad_Queue_IfDefined:nT {#1} {
        \group_begin:
            \tl_reverse:c     {g__UWMad_Queue_#1}
            \tl_map_inline:cn {g__UWMad_Queue_#1} {#2}
        \group_end:
    }
}
%    \end{macrocode}
%   \end{macro}
%
%
%   \begin{macro}{\UWMad_Queue_IfEmpty:nTF}
%   Execute true/false code depending on the emptiness
%   of a defined Queue.
%
%    \begin{macrocode}
\cs_new:Nn \UWMad_Queue_IfEmpty:nTF {
    \__UWMad_Queue_IfDefined:nT {#1} {
        \tl_if_empty:cTF {g__UWMad_Queue_#1}{
            #2
        }{
            #3
        }
    }
}
%    \end{macrocode}
%    \end{macro}
%
%
%^^A ======================================================================= %
%^^A                          Deque Creation Commands                        %
%^^A ======================================================================= %
%
%   \UWSubSubModule{Deques}
%   This set of commands is a simple system for creating and working with
%   double-ended queues (deques, pronounced \textit{deck}).  Deques are a
%   generalization of stacks and queues in that data can be pushed, popped,
%   and walked from either end of the list (i.e., LIFO+FIFO).
%
%
%
%   \begin{macro}[internal]{
%       \__UWMad_Deque_IfDefined:nT,
%       \__UWMad_Deque_IfUndefined:nT}
%   Shortcuts for the more general  commands outlined above.
%
%    \begin{macrocode}
\cs_new:Nn \__UWMad_Deque_IfDefined:nT {
    \__UWMad_IfDefined:nnnnT{g__UWMad_Deque_}{#1}{}{Deque}{#2}
}
\cs_new:Nn \__UWMad_Deque_IfUndefined:nT{
    \__UWMad_IfUndefined:nnnnT{g__UWMad_Deque_}{#1}{}{Deque}{#2}
}
%    \end{macrocode}
%   \end{macro}
%
%
%   \begin{macro}{\UWMad_Deque_Define:n}
%   Define a new Deque.
%
%    \begin{macrocode}
\cs_new:Nn \UWMad_Deque_Define:n {
    \__UWMad_Deque_IfUndefined:nT {#1} {
        \seq_new:c {g__UWMad_Deque_#1}
    }
}
%    \end{macrocode}
%   \end{macro}
%
%
%   \begin{macro}{\UWMad_Deque_Clear:n}
%   Clear but do not undefine a defined Deque.
%
%    \begin{macrocode}
\cs_new:Nn \UWMad_Deque_Clear:n {
    \__UWMad_Deque_IfDefined:nT {#1} {
        \seq_gclear:c  {g__UWMad_Deque_#1}
    }
}
%    \end{macrocode}
%   \end{macro}
%
%
%   \begin{macro}{\UWMad_Deque_Clear:n}
%   Clear and undefine a defined Deque.
%
%    \begin{macrocode}
\cs_new:Nn \UWMad_Deque_Delete:n {
    \__UWMad_Deque_IfDefined:nT {#1} {
        \seq_gclear:c  {g__UWMad_Deque_#1}
        \cs_undefine:c {g__UWMad_Deque_#1}
    }
}
%    \end{macrocode}
%   \end{macro}
%
%
%   \begin{macro}{
%       \UWMad_Deque_PushLeft:nn,
%       \UWMad_Deque_PushRight:nn}
%   Push an element on to the left or right of a defined Deque.
%
%    \begin{macrocode}
\cs_new:Nn \UWMad_Deque_PushLeft:nn {
    \__UWMad_Deque_IfDefined:nT {#1} {
        \seq_gput_left:cn  {g__UWMad_Deque_#1} {#2}
    }
}
\cs_new:Nn \UWMad_Deque_PushRight:nn {
    \__UWMad_Deque_IfDefined:nT {#1} {
        \seq_gput_right:cn {g__UWMad_Deque_#1} {#2}
    }
}
%    \end{macrocode}
%   \end{macro}
%
%
%   \begin{macro}{
%       \UWMad_Deque_PushLeft:nn,
%       \UWMad_Deque_PushRight:nn}
%   Pop an element from the left or right of a defined Deque
%   and place it into the input stream.
%
%    \begin{macrocode}
\cs_new:Nn \UWMad_Deque_PopLeft:n {
    \__UWMad_Deque_IfDefined:nT {#1} {
        \seq_gpop_left:cN  {g__UWMad_Deque_#1} \l_tmpa_tl
        \tl_use:N \l_tmpa_tl
    }
}
\cs_new:Nn \UWMad_Deque_PopRight:n {
    \__UWMad_Deque_IfDefined:nT {#1} {
        \seq_gpop_right:cN {g__UWMad_Deque_#1} \l_tmpa_tl
        \tl_use:N \l_tmpa_tl
    }
}
%    \end{macrocode}
%   \end{macro}
%
%
%   \begin{macro}{
%       \UWMad_Deque_WalkLeftToRight:nn,
%       \UWMad_Deque_WalkRightToLeft:nn}
%   Iterate over the elements left-to-right or right-to-left
%   of a defined Deque with supplied code.
%
%    \begin{macrocode}
\cs_new:Nn \UWMad_Deque_WalkLeftToRight:nn {
    \__UWMad_Deque_IfDefined:nT {#1} {
        \seq_map_inline:cn {g__UWMad_Deque_#1} {#2}
    }
}
\cs_generate_variant:Nn \seq_reverse:N {c}
\cs_new:Nn \UWMad_Deque_WalkRightToLeft:nn {
    \__UWMad_Deque_IfDefined:nT {#1} {
        \group_begin:
            \seq_reverse:c     {g__UWMad_Deque_#1}
            \seq_map_inline:cn {g__UWMad_Deque_#1} {#2}
        \group_end:
    }
}
%    \end{macrocode}
%    \end{macro}
%
%
%
%
%
%^^A =========================================================================== %
%^^A                 Hashes (Associative Arrays) with LaTeX3                     %
%^^A =========================================================================== %
%
%   \UWSubSubModule{Hashes}
%   This set of commands is a simple system for creating and working with
%   hashes (more often called associative arrays or dictionaries, but erring
%   on the side of usablility, Ruby's jargon will be used). Hashes are a
%   type of array that indexes values by (at least in \LaTeX{}) alphanumeric
%   keys instead of just integers.
%   Data can be set by key, retrieved by key, unset by key, deleted, and walked.
%
%   A hash walk, like the collection walks above, iterates through all of the
%   keys and values in the hash while applying a user supplied function.
%   However, unlike the collection walks, \textbf{a hash's walk order is not
%   gauranteed to be the set order}.  If walk order is needed to be
%   gauranteed, see the previous collection data types.
%
%   The system is a thin abstraction of \LaTeXPL's
%   |l3prop| module to avoid developing a one-shot system while allowing more
%   endeavoring authors access to the feature without learning \LaTeX3{}
%   programming.
%
%
%    \begin{macrocode}
\cs_generate_variant:Nn \prop_gput:Nnn   { c x n   }
\cs_generate_variant:Nn \prop_if_in:NnTF { c x TF  }
\cs_generate_variant:Nn \prop_if_in:NnTF { c f TF  }
\cs_generate_variant:Nn \prop_get:Nn     { c x     }
\cs_generate_variant:Nn \prop_get:Nn     { c f     }
\cs_generate_variant:Nn \prop_get:NnNTF  { c x N TF}
\cs_generate_variant:Nn \prop_gremove:Nn { c x     }
%    \end{macrocode}
%
%
%   \begin{macro}[internal]{
%       \__UWMad_Hash_IfDefined:nT,
%       \__UWMad_Hash_IfUndefined:nT}
%   Shortcuts for the more general commands outlined above.
%
%    \begin{macrocode}
\cs_new:Nn \__UWMad_Hash_IfDefined:nT {
    \__UWMad_IfDefined:nnnnT{g__UWMad_Hash_}{#1}{}{Hash}{#2}
}
\cs_new:Nn \__UWMad_Hash_IfUndefined:nT{
    \__UWMad_IfUndefined:nnnnT{g__UWMad_Hash_}{#1}{}{Hash}{#2}
}
%    \end{macrocode}
%   \end{macro}
%
%
%   \begin{macro}{\UWMad_Hash_Define:n}
%   Define a new Hash.
%
%    \begin{macrocode}
\cs_new:Nn \UWMad_Hash_Define:n {
    \__UWMad_Hash_IfUndefined:nT {#1} {
        \prop_new:c {g__UWMad_Hash_#1}
    }
}
%    \end{macrocode}
%   \end{macro}
%
%
%   \begin{macro}{\UWMad_Hash_Set:nnn}
%   Set the value of a key of a defined Hash.
%
%   \begin{Usage}
%       \item|\UWMad_Hash_Set:nnn|\marg{HashID}\marg{Key}\marg{Value}
%   \end{Usage}
%
%    \begin{macrocode}
\cs_new:Nn \UWMad_Hash_Set:nnn {
    \__UWMad_Hash_IfDefined:nT {#1} {
        \prop_gput:cxn {g__UWMad_Hash_#1}{#2}{#3}
    }
}
%    \end{macrocode}
%   \end{macro}
%
%
%   \begin{macro}{\UWMad_Hash_Get:nn}
%   Get the value of a key of a defined Hash and place
%   it into the input stream.
%
%    \begin{macrocode}
\cs_generate_variant:Nn \prop_get:cn {cf}
\cs_new:Nn \UWMad_Hash_Get:nn {
    \__UWMad_Hash_IfDefined:nT {#1} {
        \prop_get:cf {g__UWMad_Hash_#1}{#2}
    }
}
%    \end{macrocode}
%   \end{macro}
%
%
%   \begin{macro}{\UWMad_Hash_Unset:nn}
%   Undefine a key-value pair in a defined Hash.
%
%    \begin{macrocode}
\cs_new:Nn \UWMad_Hash_Unset:nn {
    \__UWMad_Hash_IfDefined:nT {#1} {
        \prop_gremove:cx {g__UWMad_Hash_#1} {#2}
    }
}
%    \end{macrocode}
%   \end{macro}
%
%
%   \begin{macro}{\UWMad_Hash_IfKeySet:nnTF}
%   Execute true/false code depending on if a key is set in
%   a defined Hash.
%
%    \begin{macrocode}
\cs_generate_variant:Nn \tl_to_lowercase:n {f}
\cs_new:Nn \UWMad_Hash_IfKeySet:nnTF {
    \__UWMad_Hash_IfDefined:nT {#1} {
        \prop_if_in:cfTF {g__UWMad_Hash_#1} {\tl_trim_spaces:n{#2}} {
            #3
        }{
            #4
        }
    }
}
%    \end{macrocode}
%   \end{macro}
%
%
%   \begin{macro}{\UWMad_Hash_Walk:nn}
%   Iterate over the key-value pairs of a defined Hash with
%   supplied code. \textbf{No order is gauranteed.}
%
%    \begin{macrocode}
\cs_new:Nn \UWMad_Hash_Walk:nn {
    \__UWMad_Hash_IfDefined:nT {#1} {
        \prop_map_inline:cn {g__UWMad_Hash_#1} {#2}
    }
}
%    \end{macrocode}
%   \end{macro}
%
%
%   \begin{macro}{\UWMad_Hash_Delete:n}
%   Clear and undefine a defined Hash.
%
%    \begin{macrocode}
\cs_new:Nn \UWMad_Hash_Delete:n {
    \__UWMad_Hash_IfDefined:nT {#1} {
        \prop_gclear:c {g__UWMad_Hash_#1}
        \cs_undefine:c {g__UWMad_Hash_#1}
    }
}
%    \end{macrocode}
%    \end{macro}
%
%
%
%
%^^A ==================================================================== %
%^^A                         LaTeX2e Abstractions                         %
%^^A ==================================================================== %
%
%   \UWSubModule{User-Level Abstractions}
%
%   The commands that follow are \LaTeXe{}-like commands that use the
%   \LaTeXPL{} as the underlying system.  \textbf{The commands are not loaded
%   by default; they must be invoked by calling the following command.}
%
%
%   \UWSubSubModule{Utility Commands}
%
%   \begin{macro}{\LoadUtilityAPI}
%   \textbf{This command needs to be invoked to load these Utilities
%   for usage.}
%
%    \begin{macrocode}
\DeclareDocumentCommand \LoadUtilityAPI { } {
%    \end{macrocode}
%   \end{macro}
%
%
%   \begin{macro}{\IfCommandExists,\IfCommandDoesNotExist}
%   This command pair is used instead of \LaTeX{}'s |\@ifundefined|.
%   Since it is \eTeX{}, this command will allow for a switch to
%   |\@ifundefined| if problems arise from non-\eTeX{} users in the
%   future.
%
%   \begin{Usage}
%       \item |\IfCommandExists|\marg{Command Name}\marg{True}\marg{False}
%       \item |\IfCommandDoesNotExist|\marg{Command Name}\marg{True}\marg{False}
%   \end{Usage}
%
%    \begin{macrocode}
\DeclareDocumentCommand \IfCommandExists { m +m +m }{
    \cs_if_exist:cTF {##1}{
        ##2
    }{
        ##3
    }
}
\DeclareDocumentCommand \IfCommandDoesNotExist { m +m +m }{
    \cs_if_free:cTF {##1}{
        ##2
    }{
        ##3
    }
}
%    \end{macrocode}
%   \end{macro}
%
%
%
%   \begin{macro}{\IfStringEmpty}
%   Checks if a given string is empty.
%   It uses the |etoolbox|'s |\ifblank|.
%   This command will not expand input.
%
%   \begin{Usage}
%       \item |\IfStringEmpty|\marg{String}\marg{True}\marg{False}
%   \end{Usage}
%
%    \begin{macrocode}
\DeclareDocumentCommand \IfStringEmpty { m +m +m }{
    \tl_if_blank:nTF {##1}{
        ##2
    }{
        ##3
    }
}
%    \end{macrocode}
%   \end{macro}
%
%
%
%   \begin{macro}{\IfCommandEmpty}
%   Uses the |etoolbox|'s |\ifdefempty| command to test if a command expands
%   to an empty string and is followed by the given conditional code.
%
%   \begin{Usage}
%       \item |IfCommandEmpty|\marg{Command}\marg{True}\marg{False}
%   \end{Usage}
%
%    \begin{macrocode}
\DeclareDocumentCommand \IfCommandEmpty { m +m +m }{
    \tl_if_blank:oTF{##1}{
        ##2
    }{
        ##3
    }
}
%    \end{macrocode}
%   \end{macro}
%
%
%   Close |\LoadUtilityAPI|.
%    \begin{macrocode}
}
%    \end{macrocode}
%
%
%
%
%
%
%^^A ==================================================================== %
%^^A                      Command Creator System                          %
%^^A ==================================================================== %
%
%   \UWSubSubModule{Command Creators}
%
%
%   \begin{macro}{\LoadCommandAPI}
%   \textbf{This command needs to be invoked to load these Command Creators
%   for usage.}
%
%    \begin{macrocode}
\DeclareDocumentCommand \LoadCommandAPI { } {
%    \end{macrocode}
%   \end{macro}
%
%   \begin{macro}{\MakeCommand,\ReMakeCommand}
%   This command pair uses the |etoolbox|'s |\csdef| to define a commands
%   via a supplied string \marg{Command Name} and a set of \marg{Code}.
%   If the requested command is not defined, |\MakeCommand| will create it;
%   however, if the requested command is already defined, |\MakeCommand| will
%   throw a warning and not make the command.
%   If the requested command is defined, |\ReMakeCommand| will redefine it;
%   however, if the requested command is not defined, |\ReMakeCommand| will
%   throw a warning and not make the command.
%
%   \begin{Usage}
%       \item |\MakeCommand|\marg{Command Name}\marg{Code}
%       \item |\ReMakeCommand|\marg{Command Name}\marg{Code}
%   \end{Usage}
%
%    \begin{macrocode}
\DeclareDocumentCommand \MakeCommand { O{} m +m } {
    \cs_if_free:cTF {##2} {
        \cs_set:cpn {##2} ##1 {##3}
    }{
        \msg_warning:nnnn
            {UWMadThesis}{Programming/Defined}{##2}{command}
    }
}
\DeclareDocumentCommand \ReMakeCommand { O{} m +m }{
    \cs_if_exist:cTF {##2} {
        \cs_set:cpn {##2} ##1 {##3}
    }{
        \msg_error:nnnn
            {UWMadThesis}{Programming/Undefined}{##2}{command}
    }
}
%    \end{macrocode}
%   \end{macro}
%
%
%
%   \begin{macro}{\MakeGlobalCommand}
%   Similar to |\MakeCommand| except the creation is made regardless of the
%   requested command's definition and the creation is global.
%
%   \begin{Usage}
%       \item |\MakeGlobalCommand|\marg{Command Name}\marg{Code}
%   \end{Usage}
%
%    \begin{macrocode}
\DeclareDocumentCommand \MakeGlobalCommand { O{} +m m } {
    \cs_gset:cpn {##2} ##1 {##3}
}
%    \end{macrocode}
%   \end{macro}
%
%
%
%   \begin{macro}{\MakeExpandedCommand}
%   This command creates a command in the spirit of |\MakeCommand|
%   but with several differences.  First, the command simply creates
%   the requested command without regard to its existence.  Secondly,
%   the \marg{Code} supplied is fully expanded without protection.
%   Lastly, the definitions are global.
%
%   \begin{Usage}
%       \item |\MakeExpandedCommand|\marg{Command Name}\marg{Code}
%   \end{Usage}
%
%    \begin{macrocode}
\DeclareDocumentCommand \MakeExpandedCommand { O{} m +m } {
    \cs_get:cpx {##2} ##1 {##3}
}
%    \end{macrocode}
%   \end{macro}
%
%
%
%   \begin{macro}{\MakeCommandUndefined}
%   Globally undefines the command specified by \marg{Command Name}.
%
%   \begin{Usage}
%       \item |\MakeCommandUndefined|\marg{Command Name}
%   \end{Usage}
%
%    \begin{macrocode}
\DeclareDocumentCommand \MakeCommandUndefined { m } {
    \cs_undefine:c {##1}
}
%    \end{macrocode}
%   \end{macro}
%
%
%
%   \begin{macro}{\CopyCommand}
%   Copies the defintion of the command named \marg{Command Name 1} to
%   a new command named \marg{Command Name 2}.  If \marg{Command Name 2}
%   already has a definition, |\CopyCommand| will throw a warning
%   \emph{but} still make the copy.
%
%   \begin{Usage}
%       \item |\CopyCommand|\marg{Command Name 1}\marg{Command Name 2}
%   \end{Usage}
%
%    \begin{macrocode}
\DeclareDocumentCommand \CopyCommand { m m } {
    \cs_if_free:cTF {##1} {
        \cs_if_free:cTF {##2} {
            \cs_gset_eq:cc {##2}{##1}
        }{
            \msg_warning:nnnn
                {UWMadThesis}{Programming/Defined}{##2}{command}
        }
    }{
        \msg_warning:nnnn
            {UWMadThesis}{Programming/Defined}{##1}{command}
    }
}
%    \end{macrocode}
%   \end{macro}
%
%
%   Close |\LoadCommandAPI|.
%    \begin{macrocode}
}
%    \end{macrocode}
%
%
%
%
%   \UWSubSubModule{Types}
%
%   \begin{macro}{\LoadTypesAPI}
%   \textbf{This command needs to be invoked to load these Type Commands
%   for usage.}
%
%    \begin{macrocode}
\DeclareDocumentCommand \LoadTypesAPI { } {
%    \end{macrocode}
%   \end{macro}
%
%   \begin{macro}{
%       \BooleanDefineLocal,
%       \BooleanDefineGlobal,
%       \BooleanDefineLocalAndSetTrue,
%       \BooleanDefineLocalAndSetFalse,
%       \BooleanDefineGlobalAndSetTrue,
%       \BooleanDefineGlobalAndSetFalse,
%       \BooleanSetTrue,
%       \BooleanSetFalse,
%       \BooleanSetIfTrue,
%       \BooleanSetIfFalse}
%   \LaTeXe{} version of the Boolean Type system above.
%    \begin{macrocode}
\cs_gset_eq:NN
    \BooleanDefineLocal             \UWMad_Boolean_DefineLocal:n
\cs_gset_eq:NN
    \BooleanDefineGlobal            \UWMad_Boolean_DefineGlobal:n
\cs_gset_eq:NN
    \BooleanDefineLocalAndSetTrue   \UWMad_Boolean_DefineLocalAndSetTrue:n
\cs_gset_eq:NN
    \BooleanDefineLocalAndSetFalse  \UWMad_Boolean_DefineLocalAndSetFalse:n
\cs_gset_eq:NN
    \BooleanDefineGlobalAndSetTrue  \UWMad_Boolean_DefineGlobalAndSetTrue:n
\cs_gset_eq:NN
    \BooleanDefineGlobalAndSetFalse \UWMad_Boolean_DefineGlobalAndSetFalse:n
\cs_gset_eq:NN
    \BooleanSetTrue                 \UWMad_Boolean_SetTrue:n
\cs_gset_eq:NN
    \BooleanSetFalse                \UWMad_Boolean_SetFalse:n
\cs_gset_eq:NN
    \BooleanSetIfTrue               \UWMad_Boolean_SetIfTrue:nTF
\cs_gset_eq:NN
    \BooleanSetIfFalse              \UWMad_Boolean_SetIfFalse:nTF
%    \end{macrocode}
%    \end{macro}
%
%
%
%   \begin{macro}{
%       \LengthDefineLocal,
%       \LengthDefineGlobal,
%       \LengthAdd,
%       \LengthSet,
%       \LengthOf,
%       \LengthIf}
%   \LaTeXe{} version of the Boolean Type system above.
%    \begin{macrocode}
\cs_gset_eq:NN \LengthDefineLocal   \UWMad_Length_DefineLocal:nn
\cs_gset_eq:NN \LengthDefineGlobal  \UWMad_Length_DefineGlobal:nn
\cs_gset_eq:NN \LengthAdd           \UWMad_Length_Add:nn
\cs_gset_eq:NN \LengthSet           \UWMad_Length_Set:nn
\cs_gset_eq:NN \LengthOf            \UWMad_Length_Of:n
\cs_gset_eq:NN \LengthIf            \UWMad_Length_If:nTF
%    \end{macrocode}
%    \end{macro}
%
%
%
%   \begin{macro}{
%        \CounterDefineLocal,
%        \CounterDefineGlobal,
%        \CounterAdd,
%        \CounterStep,
%        \CounterSet,
%        \CounterSetAndAdd,
%        \CounterSetAndStep,
%        \CounterValue,
%        \CounterIf,
%        \CounterCompare,
%        \CounterArabic,
%        \CounterAlpha,
%        \CounterALPHA,
%        \CounterRoman,
%        \CounterROMAN}
%   \LaTeXe{} version of the Counter Type system above.
%    \begin{macrocode}
\cs_gset_eq:NN \CounterDefineLocal  \UWMad_Counter_DefineLocal:nn
\cs_gset_eq:NN \CounterDefineGlobal \UWMad_Counter_DefineGlobal:nn
\cs_gset_eq:NN \CounterAdd          \UWMad_Counter_Add:nn
\cs_gset_eq:NN \CounterStep         \UWMad_Counter_Step:n
\cs_gset_eq:NN \CounterSet          \UWMad_Counter_Set:nn
\cs_gset_eq:NN \CounterSetAndAdd    \UWMad_Counter_SetAndAdd:nn
\cs_gset_eq:NN \CounterSetAndStep   \UWMad_Counter_SetAndStep:nn
\cs_gset_eq:NN \CounterValue        \UWMad_Counter_Value:n
\cs_gset_eq:NN \CounterIf           \UWMad_Counter_If:nTF
\cs_gset_eq:NN \CounterCompare      \UWMad_Counter_Compare:nnnTF
\cs_gset_eq:NN \CounterArabic       \UWMad_Counter_arabic:n
\cs_gset_eq:NN \CounterAlpha        \UWMad_Counter_alpha:n
\cs_gset_eq:NN \CounterALPHA        \UWMad_Counter_Alpha:n
\cs_gset_eq:NN \CounterRoman        \UWMad_Counter_roman:n
\cs_gset_eq:NN \CounterROMAN        \UWMad_Counter_Roman:n
%    \end{macrocode}
%    \end{macro}
%
%
%
%   Close |\LoadTypesAPI|.
%    \begin{macrocode}
}
%    \end{macrocode}
%
%
%
%   \UWSubSubModule{Collections}
%
%
%
%
%   \iffalse
%</Code>
%   \fi
%   \iffalse
%<*Code>
%   \fi
%
%
%   \UWModule{Layout And Styles}
%
%
%
%
%    \begin{macrocode}
    \geometry{
        margin      = 1.0in,
        includehead = true,
        paper       = letterpaper
    }
%
    \creflabelformat{equation}{#2#1#3}
%
    \captionsetup[table]{
        format=hang,
        labelsep=colon,
        justification=justified,
        labelfont={sl},
        textfont=sl,
        font={sc,small,stretch=1.1},
        width=0.9\textwidth,
        position=above,
        skip=0.25em
    }
%
    \captionsetup[figure]{
        format=hang,
        labelsep=colon,
        justification=justified,
        labelfont={sl},
        textfont=sl,
        font={small,stretch=1.1},
        width=0.9\textwidth,
        position=above,
        skip=0.5em
    }
%
\definecolor{UWMadGreen}{rgb}{0,0.7,0}
%
% Default hyperref behavior that is *not* PDF Metadata
\bool_if:NTF \g__UWMad_Hyperlinks_bool {
    \hypersetup {
        colorlinks         = true       ,
        linkcolor          = blue       ,
        citecolor          = UWMadGreen ,
        urlcolor           = violet     ,
        pdfdisplaydoctitle = true       ,
        pdfview            = {FitH}     ,
        pdfstartview       = {FitH}     ,
        pdfpagelayout      = OneColumn  ,
        plainpages         = false      ,
        hypertexnames      = true       ,
        bookmarksopenlevel = 3          ,
        bookmarksopen      = true       ,
        unicode            = true
    }
} {}
%
%
\pagestyle{myheadings}
\pagenumbering{roman}
\doublespacing
\setlength{\parindent}{ 0pt}
\setlength{\parskip}  {10pt}
\setlength{\headsep}  {15pt}
%    \end{macrocode}
%
%
%
%   \iffalse
%</Code>
%   \fi
%\iffalse
%<*Code>
%\fi
%
%
%   \UWModule{Sectioning}
%
%
%    \begin{macrocode}
%
%^^A  =========================================================================== %
%^^A                     Redefinition of Chapter Commands                         %
%^^A  =========================================================================== %

\let\DefaultChapter\@chapter

\tl_new:N  \g_UWMad_Sectioning_ChapterPageStyle_tl
\tl_set:Nn \g_UWMad_Sectioning_ChapterPageStyle_tl
    {\thispagestyle{myheadings}}

\def\@chapter{
    \tl_use:N \g_UWMad_Sectioning_ChapterPageStyle_tl
    \ifnum \value{chapter}=0
        \pagenumbering{arabic}
    \fi
    \DefaultChapter
}



%^^A  =========================================================================== %
%^^A                           New Appendix Command                               %
%^^A  =========================================================================== %

% Appendix counter
\UWMad_Counter_DefineGlobal:nn{Appendix}{0}

\newcommand{\AppendixInitializer}{
    \par
    \setcounter{section}{0}
    \def\@chapapp{\appendixname}
    \def\thechapter{
        \@Alph{
            \UWMad_Counter_Value:n{Appendix}
        }
    }
}

\renewcommand{\appendix}{
    \UWMad_Counter_Compare:nnnTF{Appendix}{=}{0} {
        \AppendixInitializer
    }{}
    \UWMad_Counter_Step:n{Appendix}
    \chapter
}





%^^A =========================================================================== %
%^^A               Front Matter Environment/Command Definitions                   %
%^^A  =========================================================================== %
\UWMad_Counter_DefineGlobal:nn{FrontMatter}{0}


\cs_new:Nn \__UWMad_Sectioning_FrontMatterRegister:nn {
    \addcontentsline{toc}{#1}{#2}
    \UWMad_Counter_Step:n{FrontMatter}
}

\cs_undefine:N \abstract
\cs_undefine:N \endabstract

\DeclareDocumentCommand \dedications     { O{Dedications}    } {
    \chapter*{#1}
    \__UWMad_Sectioning_FrontMatterRegister:nn{chapter}{#1}
}
\DeclareDocumentCommand \acknowledgments { O{acknowledgments}} {
    \chapter*{#1}
    \__UWMad_Sectioning_FrontMatterRegister:nn{chapter}{#1}
}
\DeclareDocumentCommand \abstract        { O{Abstract}       } {
    \chapter*{#1}
    \__UWMad_Sectioning_FrontMatterRegister:nn{chapter}{#1}
}
\DeclareDocumentCommand \umiabstract     { O{Abstract}       } {
    \chapter*{#1}
    \__UWMad_Sectioning_FrontMatterRegister:nn{chapter}{#1}
}
\DeclareDocumentCommand \preface         { O{Preface}        } {
    \chapter*{#1}
    \__UWMad_Sectioning_FrontMatterRegister:nn{chapter}{#1}
}





%^^A  =========================================================================== %
%^^A                  List of Contents, Tables, and Figures                       %
%^^A  =========================================================================== %

\cs_gset_eq:NN \TableOfContentsDefault \tableofcontents
\cs_gset_eq:NN \ListOfTablesDefault    \listoftables
\cs_gset_eq:NN \ListOfFiguresDefault   \listoffigures

\cs_undefine:N \tableofcontents
\cs_undefine:N \listoftables
\cs_undefine:N \listoffigures
\cs_undefine:N \contentsname

% Register the Table of Contents to the Table of Contents
\tl_new:N   \g__UWMad_Sectioning_TOCName_tl
\tl_gset:Nn \g__UWMad_Sectioning_TOCName_tl {Table~of~Contents}

\tl_new:N   \g__UWMad_Sectioning_LOTName_tl
\tl_gset:Nn \g__UWMad_Sectioning_LOTName_tl {List~of~Tables}

\tl_new:N   \g__UWMad_Sectioning_LOFName_tl
\tl_gset:Nn \g__UWMad_Sectioning_LOFName_tl {List~of~Figures}

\DeclareDocumentCommand \tableofcontents { } {
    \group_begin:
        \tl_if_empty:nTF {\contentsname}{
            \cs_set_eq:NN \contentsname \g__UWMad_Sectioning_TOCName_tl
        }{}
        \setstretch{1.05}
        \phantomsection
        \__UWMad_Sectioning_FrontMatterRegister:nn
            {chapter}
            {\contentsname}
        \TableOfContentsDefault
        \clearpage
    \group_end:
}
%
%
% Register the List of Tables to the Table of Contents
\DeclareDocumentCommand \listoftables { } {
    \group_begin:
        \cs_set_eq:NN \listtablename \g__UWMad_Sectioning_LOTName_tl
        \setstretch{1.05}
        \__UWMad_Sectioning_FrontMatterRegister:nn
            {chapter}
            {\listtablename}
        \ListOfTablesDefault
        \clearpage
    \group_end:
}
%
%
% Register the List of Figures to the Table of Contents
\DeclareDocumentCommand \listoffigures { } {
    \group_begin:
        \cs_set_eq:NN \listfigurename \g__UWMad_Sectioning_LOFName_tl
        \setstretch{1.05}
        \__UWMad_Sectioning_FrontMatterRegister:nn
            {chapter}
            {\listfigurename}
        \ListOfFiguresDefault
        \clearpage
    \group_end:
}

\cs_set_eq:NN \TableOfContents \tableofcontents
\cs_set_eq:NN \ListOfTables    \listoftables
\cs_set_eq:NN \ListOfFigures   \listoffigures
%
%
%
%
%
%
%^^A  =========================================================================== %
%^^A                  Table of Contents 'Headers' (i.e., Parts)                   %
%^^A  =========================================================================== %
%
\UWMad_Hash_Define:n {SectionToLevel}
\UWMad_Hash_Set:nnn  {SectionToLevel}{part}          {-1}
\UWMad_Hash_Set:nnn  {SectionToLevel}{chapter}       {0}
\UWMad_Hash_Set:nnn  {SectionToLevel}{section}       {1}
\UWMad_Hash_Set:nnn  {SectionToLevel}{subsection}    {2}
\UWMad_Hash_Set:nnn  {SectionToLevel}{subsubsection} {3}
\UWMad_Hash_Set:nnn  {SectionToLevel}{paragraph}     {4}
\UWMad_Hash_Set:nnn  {SectionToLevel}{subparagraph}  {5}
%
\UWMad_Hash_Define:n {LevelToSection}
\UWMad_Hash_Set:nnn  {LevelToSection}{-1}{part}
\UWMad_Hash_Set:nnn  {LevelToSection}{0} {chapter}
\UWMad_Hash_Set:nnn  {LevelToSection}{1} {section}
\UWMad_Hash_Set:nnn  {LevelToSection}{2} {subsection}
\UWMad_Hash_Set:nnn  {LevelToSection}{3} {subsubsection}
\UWMad_Hash_Set:nnn  {LevelToSection}{4} {paragraph}
\UWMad_Hash_Set:nnn  {LevelToSection}{5} {subparagraph}
%
\UWMad_Hash_Define:n {NextSectioningCommand}
\UWMad_Hash_Set:nnn
    {NextSectioningCommand}{part}          {chapter}
\UWMad_Hash_Set:nnn
    {NextSectioningCommand}{chapter}       {section}
\UWMad_Hash_Set:nnn
    {NextSectioningCommand}{section}       {subsection}
\UWMad_Hash_Set:nnn
    {NextSectioningCommand}{subsection}    {subsubsection}
\UWMad_Hash_Set:nnn
    {NextSectioningCommand}{subsubsection} {paragraph}
\UWMad_Hash_Set:nnn
    {NextSectioningCommand}{paragraph}     {subparagraph}
%
\UWMad_Hash_Define:n {PreviousSectioningCommand}
\UWMad_Hash_Set:nnn
    {PreviousSectioningCommand}{part}          {chapter}
\UWMad_Hash_Set:nnn
    {PreviousSectioningCommand}{chapter}       {section}
\UWMad_Hash_Set:nnn
    {PreviousSectioningCommand}{section}       {subsection}
\UWMad_Hash_Set:nnn
    {PreviousSectioningCommand}{subsection}    {subsubsection}
\UWMad_Hash_Set:nnn
    {PreviousSectioningCommand}{subsubsection} {paragraph}
\UWMad_Hash_Set:nnn
    {PreviousSectioningCommand}{paragraph}     {subparagraph}

\newcommand{\LevelToSection}[1]{% #1 = Counter for desired level
    \HashExpandableGet{LevelToSection}{#1}}

\newcommand{\SectionToLevel}[1]{% #1 = section
    \HashExpandableGet{SectionToLevel}{#1}}



%^^A  =========================================================================== %
%^^A       Paragraphs and Subparagraphs are Layed out like subsubsections         %
%^^A  =========================================================================== %
\renewcommand\paragraph{\@startsection{paragraph}{4}{\z@}%
                                    {-3.25ex\@plus -1ex \@minus -.2ex}%
                                    {1.5ex \@plus .2ex}%
                                    {\normalfont\normalsize\bfseries}}
\renewcommand\subparagraph{\@startsection{subparagraph}{5}{\parindent}%
                                    {-3.25ex\@plus -1ex \@minus -.2ex}%
                                    {1.5ex \@plus .2ex}%
                                    {\normalfont\normalsize\bfseries}}
%    \end{macrocode}
%   \iffalse
%</Code>
%   \fi
%   \iffalse
%<*Code>
%   \fi
%
%
%   \UWModule{Math}
%
%
%
%
%    \begin{macrocode}
%
%
%
\tex_everydisplay:D \exp_after:wN {
    \tex_the:D \tex_everydisplay:D
    \cs_set_eq:NN \frac \dfrac
}
%
%
%
%
%
% =========================================================================== %
%                             Derivative Commands                             %
% =========================================================================== %
%
\tl_new:N   \g_UWMad_Math_derivSymbol_tl
\tl_gset:Nn \g_UWMad_Math_derivSymbol_tl   {\mathrm{d}}
\tl_new:N   \g_UWMad_Math_pderivSymbol_tl
\tl_gset:Nn \g_UWMad_Math_pderivSymbol_tl  {\partial}
\tl_new:N   \g_UWMad_Math_tderivSymbol_tl
\tl_gset:Nn \g_UWMad_Math_tderivSymbol_tl  {\mathrm{D}}
\tl_new:N   \g_UWMad_Math_DelimiterDefaultLeft_tl
\tl_gset:Nn \g_UWMad_Math_DelimiterDefaultLeft_tl  {[}
\tl_new:N   \g_UWMad_Math_DelimiterDefaultRight_tl
\tl_gset:Nn \g_UWMad_Math_DelimiterDefaultRight_tl {]}
\tl_new:N   \l_UWMad_Math_DelimiterLeft_tl
\tl_new:N   \l_UWMad_Math_DelimiterRight_tl
%
%
\DeclareDocumentCommand \derivSymbol { } {
    \g_UWMad_Math_derivSymbol_tl
}
\DeclareDocumentCommand \pderivSymbol { } {
    \g_UWMad_Math_pderivSymbol_tl
}
\DeclareDocumentCommand \tderivSymbol { } {
    \g_UWMad_Math_tderivSymbol_tl
}
%
%
\DeclareDocumentCommand \derivSymbolChange { m } {
    \tl_set:Nn \g_UWMad_Math_derivSymbol_tl {#1}
}
\DeclareDocumentCommand \pderivSymbolChange { m } {
    \tl_set:Nn \g_UWMad_Math_pderivSymbol_tl {#1}
}
\DeclareDocumentCommand \tderivSymbolChange { m } {
    \tl_set:Nn \g_UWMad_Math_tderivSymbol_tl {#1}
}
%
%
\DeclareDocumentCommand \derivSymbolChangeDefault { m } {
    \tl_gset:Nn \g_UWMad_Math_derivSymbol_tl {#1}
}
\DeclareDocumentCommand \pderivSymbolChangeDefault { m } {
    \tl_gset:Nn \g_UWMad_Math_pderivSymbol_tl {#1}
}
\DeclareDocumentCommand \tderivSymbolChangeDefault { m } {
    \tl_gset:Nn \g_UWMad_Math_tderivSymbol_tl {#1}
}
%
%
\DeclareDocumentCommand \DelimiterChangeDefault { m m } {
    \tl_gset:Nn  \g_UWMad_Math_DelimiterDefaultLeft_tl  {#1}
    \tl_gset:Nn  \g_UWMad_Math_DelimiterDefaultRight_tl {#2}
}
%
%
\DeclareDocumentCommand \DerivativeGeneral { +m +m m m } {
    \frac{ #4^{#3} #1      }
         { #4      #2^{#3} }
}
\DeclareDocumentCommand \DerivativeGeneralBig { +m +m m m m m} {

    \IfNoValueTF {#5} {
        \tl_set_eq:NN
            \l_UWMad_Math_DelimiterLeft_tl
            \g_UWMad_Math_DelimiterDefaultLeft_tl
    } {
        \tl_set:Nn \l_UWMad_Math_DelimiterLeft_tl {#5}
    }

    \IfNoValueTF {#6} {
        \tl_set_eq:NN
            \l_UWMad_Math_DelimiterRight_tl
            \g_UWMad_Math_DelimiterDefaultRight_tl
    } {
        \tl_set:Nn \l_UWMad_Math_DelimiterRight_tl {#6}
    }

    \frac{ #4^{#3}    }
         { #4 #2^{#3} }
    \!\!
    \left\l_UWMad_Math_DelimiterLeft_tl
        #1
    \right\l_UWMad_Math_DelimiterRight_tl
}
%
%
\DeclareDocumentCommand \deriv { +m +m G{} } {
    \DerivativeGeneral
        {#1}{#2}{#3}{\derivSymbol}
}
\DeclareDocumentCommand \pderiv { +m +m G{} } {
    \DerivativeGeneral
        {#1}{#2}{#3}{\pderivSymbol}
}
\DeclareDocumentCommand \tderiv { +m +m G{} } {
    \DerivativeGeneral
        {#1}{#2}{#3}{\tderivSymbol}
}
%
%
\DeclareDocumentCommand \derivbig { o +m o +m G{} } {
    \DerivativeGeneralBig
        {#2}{#4}{#5}{\derivSymbol}{#1}{#3}
}
\DeclareDocumentCommand \pderivbig { o +m o +m G{} } {
    \DerivativeGeneralBig
        {#2}{#4}{#5}{\pderivSymbol}{#1}{#3}
}
\DeclareDocumentCommand \tderivbig { o +m o +m G{} } {
    \DerivativeGeneralBig
        {#2}{#4}{#5}{\tderivSymbol}{#1}{#3}
}
%
%
%
%
\DeclareMathOperator*{\Sup}    {Sup}
\DeclareMathOperator*{\Inf}    {Inf}
\DeclareMathOperator*{\Lim}    {Lim}
\DeclareMathOperator*{\Min}    {Min}
\DeclareMathOperator*{\Max}    {Max}
\DeclareMathOperator*{\ArgMin} {ArgMin}
\DeclareMathOperator*{\ArgMax} {ArgMax}
\DeclareMathOperator{\Abs}     {Abs}
\DeclareMathOperator{\Ln}      {Ln}
\DeclareMathOperator{\Log}     {Log}
\DeclareMathOperator{\Exp}     {Exp}
\DeclareMathOperator{\Cos}     {Cos}
\DeclareMathOperator{\Sin}     {Sin}
\DeclareMathOperator{\Tan}     {Tan}
\DeclareMathOperator{\Sec}     {Sec}
\DeclareMathOperator{\Csc}     {Csc}
\DeclareMathOperator{\Cot}     {Cot}
\DeclareMathOperator{\Cosh}    {Cosh}
\DeclareMathOperator{\Sinh}    {Sinh}
\DeclareMathOperator{\Tanh}    {Tanh}
\DeclareMathOperator{\Sech}    {Sech}
\DeclareMathOperator{\Csch}    {Csch}
\DeclareMathOperator{\Coth}    {Coth}
\DeclareMathOperator{\ArcCos}  {ArcCos}
\DeclareMathOperator{\ArcSin}  {ArcSin}
\DeclareMathOperator{\ArcTan}  {ArcTan}
\DeclareMathOperator{\ArcSec}  {ArcSec}
\DeclareMathOperator{\ArcCsc}  {ArcCsc}
\DeclareMathOperator{\ArcCot}  {ArcCot}
\DeclareMathOperator{\ArcCosh} {ArcCosh}
\DeclareMathOperator{\ArcSinh} {ArcSinh}
\DeclareMathOperator{\ArcTanh} {ArcTanh}
\DeclareMathOperator{\ArcSech} {ArcSech}
\DeclareMathOperator{\ArcCsch} {ArcCsch}
\DeclareMathOperator{\ArcCoth} {ArcCoth}
%
%
%
%
%
% ===================================================================== %
%                          Miscellaneous Commands                       %
% ===================================================================== %
\cs_new:Nn \UWMad_Math_RootWithTail:nn {

    \hbox_set:Nn \l_tmpa_box {
        $
            \mathchoice
                {\root #1 \of {#2\:\!}}
                {\root #1 \of {#2\:\!}}
                {\root #1 \of {#2\:\!}}
                {\root #1 \of {#2\:\!}}
        $
    }
    %
    \dim_set:Nn \l_tmpa_dim {\box_ht:N \l_tmpa_box}
    \dim_set:Nn \l_tmpb_dim {0.8\l_tmpa_dim}
    %
    \hbox_set:Nn \l_tmpb_box {
        \tex_vrule:D height \l_tmpa_dim depth -\l_tmpb_dim
    }
    %
    \box_use:N \l_tmpa_box
    \box_move_down:nn {0.40pt}{\box_use:N \l_tmpb_box}
}
\DeclareDocumentCommand \Sqrt { O{} m } {
    \UWMad_Math_RootWithTail:nn{#1}{#2}
}
%
%
\DeclareExpandableDocumentCommand \IfMathModeTF { +m +m } {
    \mode_if_math:TF {
        #1
    }{
        $#2$
    }
}
\ExplSyntaxOff
    \DeclareDocumentCommand \subs { O{} +m } {%
        \IfMathModeTF{%
            _{\!\!\:#1\text{\scriptsize #2}}%
        }{%
            _{\!#1\text{\scriptsize #2}}%
        }%
    }%
    \DeclareDocumentCommand \sups { O{} +m } {%
        \IfMathModeTF{%
            ^{#1\text{\scriptsize #2}}%
        }{%
            ^{#1\text{\scriptsize #2}}%
        }%
    }%
    \DeclareDocumentCommand \subsups { O{} +m O{} +m } {%
        \IfMathModeTF{%
            _{#1\text{\scriptsize #2}}^{\!\!\:#3\text{\scriptsize #4}}%
        }{%
            _{#1\text{\scriptsize #2}}^{\!\!\!#3\text{\scriptsize #4}}%
        }%
    }%
\ExplSyntaxOn
%
\DeclareDocumentCommand \OneOver { +m } {
    \frac{1}{#1}
}
\DeclareDocumentCommand \oneo { +m } {
    \OneOver{#1}
}
\DeclareDocumentCommand \dd { m } {
    \mathrm{d}{#1}
}
\DeclareDocumentCommand \dprime { } {
    {\prime\prime}
}
\DeclareDocumentCommand \tprime { } {
    {\prime\prime\prime}
}
\DeclareDocumentCommand \LessThan        { } {<}
\DeclareDocumentCommand \GreaterThanThan { } {>}
%
%    \end{macrocode}
%   \iffalse
%</Code>
%   \fi
%   \iffalse
%<*Code>
%   \fi
%
%
%  \UWModule{ListOf}
%
%   The ListOf Module is a collection of commands that enables the easy
%   creation and typsetting of Lists.
%
%   Lists are taken to be any collection of entries that is to be typeset
%   with a particular style.  For example, a simple Nomenclature could be
%   considered a list of (symbol, description) entries to be typeset with a
%   fixed style for all entires.  The |ListOf| commands create a system
%   specifically for this scenario.
%
%   Of course, as the commands description will show, lists can be much more
%   complicated that two items.  For the |ListOf| system to function, an
%   author really only needs to define the |ListOf|, create a command to push
%   (enqueue) entries on to the |ListOf| queue, and at some point tell the
%   |ListOf| to typeset the entries it has stored (if display of the content
%   is desired).
%
%
%
%
%   \UWSubModule{ListOf Commands}
%
%   \begin{macro}{\UWMad_ListOf_Define:n}
%   Define a new |ListOf| with \marg{ID}. This command creates the
%   commands to store the section commands and title for each group,
%   the booleans to indicate if the sections should be numbered and
%   if the sections should be included in the table of contentst
%   (regardless of numbering), a hash to hold of the user-defined
%   hooks for the |ListOf|, and a queue to store the entries for
%   typesetting.
%
%    \begin{macrocode}
\cs_new:Nn \UWMad_ListOf_Define:n {
    \tl_const:cn {c__UWMad_ListOf#1_IsDefined_tl}{}
%
    \tl_new:c {g__UWMad_ListOf#1_Section_Main_tl}
    \tl_new:c {g__UWMad_ListOf#1_Section_Group_tl}
    \tl_new:c {g__UWMad_ListOf#1_Section_Subgroup_tl}
%
    \tl_new:c {g__UWMad_ListOf#1_Title_Main_tl}
    \tl_new:c {g__UWMad_ListOf#1_Title_Group_tl}
    \tl_new:c {g__UWMad_ListOf#1_Title_Subgroup_tl}
%
    \bool_new:c       {g__UWMad_ListOf#1_ClearAfterPrint_bool}
    \bool_gset_true:c {g__UWMad_ListOf#1_ClearAfterPrint_bool}
    \bool_new:c       {g__UWMad_ListOf#1_IsNumbered_bool}
    \bool_gset_true:c {g__UWMad_ListOf#1_IsNumbered_bool}
    \bool_new:c       {g__UWMad_ListOf#1_IncludeInTOC_bool}
    \bool_gset_true:c {g__UWMad_ListOf#1_IncludeInTOC_bool}
    \UWMad_Queue_Define:n               {g__ListOf#1_EntryQueue}
    \UWMad_Hash_Define:n                {g__ListOf#1_Hook}
}
%    \end{macrocode}
%   \end{macro}
%
%
%
%   \begin{macro}{\UWMad_ListOf_Delete:n}
%   Simply undefines all of the commands created in the |Define| command
%   above for the given \marg{ID}.
%
%    \begin{macrocode}
\cs_new:Nn \UWMad_ListOf_Delete:n {
    \cs_undefine:c {c__UWMad_ListOf#1_IsDefined_tl}
%
    \cs_undefine:c {g__UWMad_ListOf#1_Section_Main_tl}
    \cs_undefine:c {g__UWMad_ListOf#1_Section_Group_tl}
    \cs_undefine:c {g__UWMad_ListOf#1_Section_Subgroup_tl}
%
    \cs_undefine:c {g__UWMad_ListOf#1_Title_Main_tl}
    \cs_undefine:c {g__UWMad_ListOf#1_Title_Group_tl}
    \cs_undefine:c {g__UWMad_ListOf#1_Title_Subgroup_tl}
%

    \show\bool_new:N

    \cs_undefine:c {g__UWMad_ListOf#1_ClearAfterPrint_bool}
    \cs_undefine:c {g__UWMad_ListOf#1_IsNumbered_bool}
    \cs_undefine:c {g__UWMad_ListOf#1_IncludeInTOC_bool}
    \UWMad_Queue_Delete:n   {g__ListOf#1_EntryQueue}
    \UWMad_Hash_Delete:n    {g__ListOf#1_Hook}
}
%    \end{macrocode}
%   \end{macro}
%
%
%
%   \begin{macro}{\UWMad_ListOf_IfDefined:nT}
%   Checks to see if a |ListOf| with \marg{ID} has been created and
%   errors if not.
%
%    \begin{macrocode}
\cs_new:Nn \UWMad_ListOf_IfDefined:nT {
    \__UWMad_IfDefined:nnnnT
        {c__UWMad_ListOf}
        {#1}
        {_IsDefined_tl}
        {ListOf}
        {#2}
}
%    \end{macrocode}
%   \end{macro}
%
%
%
%   \begin{macro}{\UWMad_ListOf_MakeNumbered:n}
%   Makes the current section of the |ListOf| with \marg{ID} numbered.
%
%    \begin{macrocode}
\cs_new:Nn \UWMad_ListOf_MakeNumbered:n {
    \UWMad_ListOf_IfDefined:nT {#1} {
        \bool_set_true:c {g__UWMad_ListOf#1_IsNumbered_bool}
    }
}
%    \end{macrocode}
%   \end{macro}
%
%
%
%   \begin{macro}{\UWMad_ListOf_MakeNotNumbered:n}
%   Makes the current section of the |ListOf| with \marg{ID} unnumbered
%   (it ``stars'' the section).
%
%    \begin{macrocode}
\cs_new:Nn \UWMad_ListOf_MakeNotNumbered:n {
    \UWMad_ListOf_IfDefined:nT {#1} {
        \bool_set_false:c {g__UWMad_ListOf#1_IsNumbered_bool}
    }
}
%    \end{macrocode}
%   \end{macro}
%
%
%
%   \begin{macro}{\UWMad_ListOf_IfNumbered:nTF}
%   Branches to \marg{True Code} or \marg{False Code} depending on whether
%   the |ListOf| with \marg{ID} is numbered or not.
%
%    \begin{macrocode}
\cs_new:Nn \UWMad_ListOf_IfNumbered:nTF {
    \UWMad_ListOf_IfDefined:nT {#1} {
        \bool_if:cTF {g__UWMad_ListOf#1_IsNumbered_bool} {
            #2
        }{
            #3
        }
    }
}
%    \end{macrocode}
%   \end{macro}
%
%
%
%   \begin{macro}{\UWMad_ListOf_IncludeInTOC:n}
%   Makes the current section of the |ListOf| with \marg{ID} appear in
%   the Table of Contents (TOC) regardless of if it is numbered/unnumbered.
%
%    \begin{macrocode}
\cs_new:Nn \UWMad_ListOf_IncludeInTOC:n {
    \UWMad_ListOf_IfDefined:nT {#1} {
        \bool_set_true:c {c__UWMad_ListOf#1_IncludeInTOC_bool}
    }
}
%    \end{macrocode}
%   \end{macro}
%
%
%
%   \begin{macro}{\UWMad_ListOf_DoNotIncludeInTOC:n}
%   Makes the current section of the |ListOf| with \marg{ID} not appear in
%   the Table of Contents (TOC) regardless of if it is numbered/unnumbered.
%
%    \begin{macrocode}
\cs_new:Nn \UWMad_ListOf_DoNotIncludeInTOC:n {
    \UWMad_ListOf_IfDefined:nT {#1} {
        \bool_set_false:c {c__UWMad_ListOf#1_IncludeInTOC_bool}
    }
}
%    \end{macrocode}
%   \end{macro}
%
%
%
%   \begin{macro}{\UWMad_ListOf_IfIncludeInTOC:n}
%   Branches to \marg{True Code} or \marg{False Code} depending on whether
%   the |ListOf| with \marg{ID} is to be included or not.
%
%    \begin{macrocode}
\cs_new:Nn \UWMad_ListOf_IfIncludeInTOC:nTF {
    \UWMad_ListOf_IfDefined:nT {#1} {
        \bool_if:cTF {c__UWMad_ListOf#1_IncludeInTOC_bool} {
            #2
        }{
            #3
        }
    }
}
%    \end{macrocode}
%   \end{macro}
%
%
%
%   \begin{macro}{
%       \UWMad_ListOf_SetTitle_Main:nn,
%       \UWMad_ListOf_SetTitle_Group:nn,
%       \UWMad_ListOf_SetTitle_Subgroup:nn}
%   Sets the value of the title of the sections to \marg{Title} for the
%   |ListOf| with \marg{ID}
%
%    \begin{macrocode}
\cs_new:Nn \UWMad_ListOf_SetTitle_Main:nn {
    \UWMad_ListOf_IfDefined:nT {#1} {
        \tl_set:cn {g__UWMad_ListOf#1_TitleMain_tl}{#2}
    }
}
\cs_new:Nn \UWMad_ListOf_SetTitle_Group:nn {
    \UWMad_ListOf_IfDefined:nT {#1} {
        \tl_set:cn {g__UWMad_ListOf#1_TitleGroup_tl}{#2}
    }
}
\cs_new:Nn \UWMad_ListOf_SetTitle_Subgroup:nn {
    \UWMad_ListOf_IfDefined:nT {#1} {
        \tl_set:cn {g__UWMad_ListOf#1_TitleSubgroup_tl}{#2}
    }
}
%    \end{macrocode}
%   \end{macro}
%
%
%
%   \begin{macro}{
%       \UWMad_ListOf_GetTitle_Main:nn,
%       \UWMad_ListOf_GetTitle_Group:nn,
%       \UWMad_ListOf_GetTitle_Subgroup:nn}
%   Gets the value of the title of the sections to \marg{Title} for the
%   |ListOf| with \marg{ID}
%
%    \begin{macrocode}
\cs_new:Nn \UWMad_ListOf_GetTitle_Main:n {
    \UWMad_ListOf_IfDefined:nT {#1} {
        \tl_use:c {g__UWMad_ListOf#1_TitleMain_tl}
    }
}
\cs_new:Nn \UWMad_ListOf_GetTitle_Group:n {
    \UWMad_ListOf_IfDefined:nT {#1} {
        \tl_use:c {g__UWMad_ListOf#1_TitleGroup_tl}
    }
}
\cs_new:Nn \UWMad_ListOf_GetTitle_Subgroup:n {
    \UWMad_ListOf_IfDefined:nT {#1} {
        \tl_use:c {g__UWMad_ListOf#1_TitleSubgroup_tl}
    }
}
%    \end{macrocode}
%   \end{macro}
%
%
%
%   \begin{macro}{
%       \UWMad_ListOf_SetSection_Main:nn,
%       \UWMad_ListOf_SetSection_Group:nn,
%       \UWMad_ListOf_SetSection_Subgroup:nn}
%   Sets the value of the sectioning command for a particular group of the
%   |ListOf| with \marg{ID}.
%
%    \begin{macrocode}
\cs_new:Nn \UWMad_ListOf_SetSection_Main:nn {
    \UWMad_ListOf_IfDefined:nT {#1} {
        \tl_set:cn {g__UWMad_ListOf#1_Section_Main_tl}{#2}
    }
}
\cs_new:Nn \UWMad_ListOf_SetSection_Group:nn {
    \UWMad_ListOf_IfDefined:nT {#1} {
        \tl_set:cn {g__UWMad_ListOf#1_Section_Group_tl}{#2}
    }
}
\cs_new:Nn \UWMad_ListOf_SetSection_Subgroup:nn {
    \UWMad_ListOf_IfDefined:nT {#1} {
        \tl_set:cn {g__UWMad_ListOf#1_Section_Subgroup_tl}{#2}
    }
}
%    \end{macrocode}
%   \end{macro}
%
%
%
%   \begin{macro}{
%       \UWMad_ListOf_GetSection_Main:nn,
%       \UWMad_ListOf_GetSection_Group:nn,
%       \UWMad_ListOf_GetSection_Subgroup:nn}
%   Gets the value of the sectioning command for a particular group of the
%   |ListOf| with \marg{ID}.
%
%    \begin{macrocode}
\cs_new:Nn \UWMad_ListOf_GetSection_Main:n {
    \UWMad_ListOf_IfDefined:nT {#1} {
        \tl_use:c {g__UWMad_ListOf#1_Section_Main_tl}
    }
}
\cs_new:Nn \UWMad_ListOf_GetSection_Group:n {
    \UWMad_ListOf_IfDefined:nT {#1} {
        \tl_use:c {g__UWMad_ListOf#1_Section_Group_tl}
    }
}
\cs_new:Nn \UWMad_ListOf_GetSection_Subgroup:n {
    \UWMad_ListOf_IfDefined:nT {#1} {
        \tl_use:c {g__UWMad_ListOf#1_Section_Subgroup_tl}
    }
}
%    \end{macrocode}
%   \end{macro}
%
%
%
%   \begin{macro}{\UWMad_ListOf_PushEntry:nn}
%   Pushes \marg{Entry} on to the entry queue of the |ListOf| with \marg{ID}.
%
%    \begin{macrocode}
\cs_new:Nn \UWMad_ListOf_PushEntry:nn {
    \UWMad_Hash_Get:nn   {g__ListOf#1_Hook}{PrePush}
    \UWMad_Queue_Push:nn {g__ListOf#1_EntryQueue}{#2}
    \UWMad_Hash_Get:nn   {g__ListOf#1_Hook}{PostPush}
}
%    \end{macrocode}
%   \end{macro}
%
%
%
%   \begin{macro}{\UWMad_ListOf_SetHook:nnn}
%   Sets a \marg{HookName} to a \marg{HookCode} for the |ListOf|
%   with \marg{ID}.
%
%    \begin{macrocode}
\cs_new:Nn \UWMad_ListOf_SetHook:nnn {
    \UWMad_Hash_Set:nnn{g__ListOf#1_Hook}{#2}{#3}
}
%    \end{macrocode}
%   \end{macro}
%
%
%
%   \begin{macro}{\UWMad_ListOf_PrintEntries:n}
%   Prints all entries currently in the |ListOf| queue with \marg{ID} and
%   clears the queue.  The \texttt{PrePrint} and \texttt{PostPrint} hooks
%   are also called here.
%
%    \begin{macrocode}
\cs_new:Nn \UWMad_ListOf_PrintEntries:n {
    \UWMad_Hash_Get:nn   {g__ListOf#1_Hook}{PrePrint}
    \UWMad_Queue_Walk:nn {g__ListOf#1_EntryQueue}{##1}
    \UWMad_Queue_Clear:n {g__ListOf#1_EntryQueue}
    \UWMad_Hash_Get:nn   {g__ListOf#1_Hook}{PostPrint}
}
%    \end{macrocode}
%   \end{macro}
%
%
%
%   \begin{macro}{\UWMad_ListOf_PrintTitle:n}
%   Prints all entries currently in the |ListOf| queue with \marg{ID} and
%   clears the queue.  The \texttt{PrePrint} and \texttt{PostPrint} hooks
%   are also called here.
%
%    \begin{macrocode}
\cs_new:Nn \__UWMad_ListOf_CurrentSectioningCommmand:n {}
\cs_new:Nn \UWMad_ListOf_PrintTitle:nn {

    \cs_set_eq:Nc
        \__UWMad_ListOf_CurrentSectioningCommmand:n
        {\cs:w g__UWMad_ListOf#1_Section_#2_tl \cs_end:}

    \UWMad_ListOf_IfNumbered:nTF {#1} {
        \__UWMad_ListOf_CurrentSectioningCommmand:n
            {\tl_use:c {g__UWMad_ListOf#1_Title_#2_tl}}
    } {
        \UWMad_ListOf_IfIncludeInTOC:nTF {#1} {
            \phantomsection
            \addcontentsline
                {toc}
                {\tl_use:c {g__UWMad_ListOf#1_Section_#2_tl}}
                {\tl_use:c {g__UWMad_ListOf#1_Title_#2_tl}}
        } { }
        \__UWMad_ListOf_CurrentSectioningCommmand:n*
        {\tl_use:c {g__UWMad_ListOf#1_Title_#2_tl}}
    }
}
%    \end{macrocode}
%   \end{macro}
%
%
%
%   \begin{macro}{\UWMad_ListOf_StartGroup:nn}
%   A shortcut command that prints the entires in the current queue
%   and then starts the next section by printing the title.
%
%    \begin{macrocode}
\cs_new:Nn \UWMad_ListOf_StartGroup:nn {
    \UWMad_ListOf_PrintEntries:n{#1}
    \UWMad_ListOf_PrintTitle:nn {#1}{#2}
}
%    \end{macrocode}
%   \end{macro}
%
%
%    \begin{macrocode}
%
%
%
% Nomenclature Environmentment ----------------------------------------------------
\dim_new:N \g__UWMad_Nomenclature_WidestSymbol_dim
\dim_new:N \g__UWMad_Nomenclature_TitleSkip_dim
\dim_new:N \g__UWMad_Nomenclature_PrintSkip_dim
\dim_new:N \g__UWMad_Nomenclature_Entry_MarginLeft_dim
\dim_new:N \g__UWMad_Nomenclature_Entry_MarginBottom_dim
\dim_new:N \g__UWMad_Nomenclature_Entry_WidthSymbol_dim
\dim_new:N \g__UWMad_Nomenclature_Entry_WidthDescription_dim
\dim_new:N \g__UWMad_Nomenclature_Entry_Padding_dim
%
\dim_gset:Nn \g__UWMad_Nomenclature_PrintSkip_dim           {1.00em}
\dim_gset:Nn \g__UWMad_Nomenclature_Entry_MarginLeft_dim    {1.00em}
\dim_gset:Nn \g__UWMad_Nomenclature_Entry_MarginBottom_dim  {0.25em}
\dim_gset:Nn \g__UWMad_Nomenclature_Entry_Padding_dim       {0.75em}
%
%
%
\cs_new:Nn \UWMad_Nomenclature_UpdateWidest:n {
    \hbox_set:Nn \l_tmpa_box {#1}
    \dim_set:Nn  \l_tmpa_dim {\box_wd:N \l_tmpa_box}
    \dim_compare:nNnTF
        {\g__UWMad_Nomenclature_WidestSymbol_dim} < {\l_tmpa_dim} {
        \dim_gset:Nn
            \g__UWMad_Nomenclature_WidestSymbol_dim
            {\l_tmpa_dim}
    } { }
}
%
\cs_new:Nn \UWMad_Nomenclature_ZeroWidestSymbol: {
    \dim_gset:Nn \g__UWMad_Nomenclature_WidestSymbol_dim {0pt}
}
%
%
%
\cs_new:Nn \UWMad_Nomenclature_SetEntryWidths: {
    % Define Symbol minipage width
    \dim_gset:Nn \g__UWMad_Nomenclature_Entry_WidthSymbol_dim {
        1.01\g__UWMad_Nomenclature_WidestSymbol_dim
    }
    %
    % Define Description minipage width
    \dim_gset:Nn \g__UWMad_Nomenclature_Entry_WidthDescription_dim {
        0.99\textwidth -
        \g__UWMad_Nomenclature_Entry_MarginLeft_dim -
        \g__UWMad_Nomenclature_Entry_WidthSymbol_dim -
        \g__UWMad_Nomenclature_Entry_Padding_dim
    }
}
%
%
%
\cs_new:Nn \UWMad_Nomenclature_Entry:nn {

% Set the entry material in the temporary coffins
    \vcoffin_set:Nnn
        \l_tmpa_coffin
        {\g__UWMad_Nomenclature_Entry_WidthSymbol_dim}
        {#1}
    \vcoffin_set:Nnn
        \l_tmpb_coffin
        {\g__UWMad_Nomenclature_Entry_WidthDescription_dim}
        {#2}
%
% Typeset the material and skips
    \group_begin:
        \setstretch{1.1}
        \skip_horizontal:n {\g__UWMad_Nomenclature_Entry_MarginLeft_dim}
        \coffin_typeset:Nnnnn \l_tmpa_coffin {l}{t}{0pt}{0pt}
        \skip_horizontal:n {\g__UWMad_Nomenclature_Entry_Padding_dim}
        \coffin_typeset:Nnnnn \l_tmpb_coffin {l}{t}{0pt}{0pt}
        \skip_vertical:n {\g__UWMad_Nomenclature_Entry_MarginBottom_dim}
    \group_end:
}
%
%
%
%
\DeclareDocumentEnvironment {Nomenclature} { s O{chapter} G{Nomenclature} } {
%
    \UWMad_Hash_IfKeySet:nnTF {SectionToLevel}{#2} { } {
        \UWMad@ClassError
            {Nomenclature~environment~received~invalid~section~`#2'}
    }
%
%
    \UWMad_ListOf_Define:n {Nomenclature}
%
    \IfBooleanTF {#1} {
        \UWMad_ListOf_MakeNotNumbered:n {Nomenclature}
        \typeout{!!!!!!!!!!!!!!!!!!!!!!!!!!!!!!!!!!!!!! TRUE}
    }{
        \UWMad_ListOf_MakeNumbered:n    {Nomenclature}
    }
%
%
    \UWMad_ListOf_SetSection_Main:nn  {Nomenclature} {#2}
    \UWMad_ListOf_SetSection_Group:nn {Nomenclature} {
        \UWMad_Hash_Get:nn{NextSectioningCommand}{#2}
    }
    \UWMad_ListOf_SetSection_Subgroup:nn {Nomenclature} {
        \UWMad_Hash_Get:nn{NextSectioningCommand} {
            \UWMad_ListOf_GetSection_Group:n{Nomenclature}
        }
    }
%
%
    \DeclareDocumentCommand \Entry { m m } {
        \UWMad_ListOf_PushEntry:nn {Nomenclature} {
            \UWMad_Nomenclature_Entry:nn
                {##1}
                {##2}
        }
        \UWMad_Nomenclature_UpdateWidest:n{##1}
    }

    \DeclareDocumentCommand  \ResetColumnWidth { } {
        \UWMad_Nomenclature_ZeroWidestSymbol:
    }
%
%
%
    \UWMad_ListOf_SetHook:nnn {Nomenclature} {PrePrint} {
        \UWMad_Nomenclature_SetEntryWidths:
    }
    \UWMad_ListOf_SetHook:nnn {Nomenclature} {PostPrint} {
        \UWMad_Nomenclature_ZeroWidestSymbol:
    }
    \UWMad_ListOf_SetTitle_Main:nn {Nomenclature}{#3}
    \UWMad_ListOf_PrintTitle:nn {Nomenclature}{Main}
%
%
%
    \DeclareDocumentCommand \Group { s G{} } {
        \IfBooleanTF {##1} {
            \UWMad_ListOf_MakeNotNumbered:n {Nomenclature}
        }{
            \UWMad_ListOf_MakeNumbered:n    {Nomenclature}
        }
        \UWMad_ListOf_SetTitle_Group:nn {Nomenclature}{##2}
        \UWMad_ListOf_StartGroup:nnn{Nomenclature}{Group}
    }
%
%
    \DeclareDocumentCommand \SubGroup { s G{} } {
        \IfBooleanTF {##1} {
            \UWMad_ListOf_MakeNotNumbered:n {Nomenclature}
        }{
            \UWMad_ListOf_MakeNumbered:n    {Nomenclature}
        }
        \UWMad_ListOf_SetTitle_Subgroup:nn {Nomenclature}{##2}
        \UWMad_ListOf_StartGroup:nnn{Nomenclature}{Subgroup}
    }
%
}{
    \UWMad_ListOf_PrintEntries:n {Nomenclature}
    \UWMad_ListOf_Delete:n{Nomenclature}
}
%
%
\DeclareDocumentEnvironment {Acronym} { s O{chapter} G{Acronym} } {

    \IfBooleanTF {#1} {
        \begin{Nomenclature*}[#2]{#3}
    } {
        \begin{Nomenclature}[#2]{#3}
    }

    \cs_undefine:N \Entry
    \DeclareDocumentCommand \Entry { m m } {
        \UWMad_ListOf_PushEntry:nn {Nomenclature} {
            \hypertarget{Acronym:##1}{}
            \UWMad_Nomenclature_Entry:nn
                {##1}
                {##2}
        }
        \UWMad_Nomenclature_UpdateWidest:n{##1}
    }
} {
    \end{Nomenclature}
}

\DeclareDocumentCommand \Acro { m } {
    \hyperlink{Acronym:#1}{#1}
}
%
%
%
%
\cs_new:cpn      {Nomenclature*} {\Nomenclature*}
\cs_new_eq:cN {endNomenclature*} \endNomenclature
\cs_new:cpn      {Acronym*} {\Acronym*}
\cs_new_eq:cN {endAcronym*} \endAcronym
%
%
%
%    \end{macrocode}
%
%
%
%
%
%   \iffalse
%</Code>
%   \fi
%   \iffalse
%<*Code>
%   \fi
%^^A ------------------------------------------------------------------------ %
%^^A                    Metadata Writing/Importing Routines                   %
%^^A ------------------------------------------------------------------------ %
%
%   \UWModule{PDF and Other Metadata}
%
%   Since the metadata (i.e., properties) of a PDF must be set in the preamble
%   but typically a user defines them in the document, these routines write the
%   supported metadata that a user may define to an auxiliary file that is
%   then imported upon recompilation.  It uses the CSV commands above to define
%   and build the CSV list, and then uses the TeX |\write| to dump it to the file.
%
%
% Used to determine is the list was created
%
%    \begin{macrocode}
%
% Command used to append the data to the CSV list.  It is called in the
% metadata commands below
\UWMad_CSV_Define:n {MetaDataList}
\cs_new:Nn \UWMad_MetaData_PushToList:nn {
   \UWMad_CSV_Append:nn {MetaDataList}{#1={#2}}
}

\bool_new:N \g__UWMad_MetaData_GenerateAux_bool
\bool_new:N \g__UWMad_MetaData_IsDocument_bool

\file_if_exist:nTF{\jobname.UWMad.PDFMetaData.aux} {
    \file_input:n {\jobname.UWMad.PDFMetaData.aux}
}{}

\AtBeginDocument{
    \UWMad_CSV_IfNotEmpty:nTF{MetaDataList} {
        \exp_args:Nx \hypersetup {\UWMad_CSV_Get:n{MetaDataList}}
    } { }
    \bool_gset_true:N \g__UWMad_MetaData_IsDocument_bool
}
\AtEndDocument{
    \bool_if:NTF \g__UWMad_MetaData_GenerateAux_bool {
        \UWMad_CSV_IfNotEmpty:nTF {MetaDataList} {
            \iow_new:N   \g__UWMad_PDFMetaData_HyperSetup_io
            \iow_open:Nn \g__UWMad_PDFMetaData_HyperSetup_io {
                \jobname.UWMad.PDFMetaData.aux
            }
            \iow_now:Nx  \g__UWMad_PDFMetaData_HyperSetup_io {
                \noexpand\ExplSyntaxOff
                    \noexpand\hypersetup{\UWMad_CSV_Get:n{MetaDataList}}
                \noexpand\ExplSyntaxOn
            }
            \iow_close:N \g__UWMad_PDFMetaData_HyperSetup_io
        } { }
    } { }
}



% ------------------------------------------------------------------------ %
%           Metadata Defining/Storing Commands (User's Interface)          %
% ------------------------------------------------------------------------ %
%
\tl_new:N \g__UWMad_ThesisInfo_Title_tl
\tl_new:N \g__UWMad_ThesisInfo_Author_tl
\tl_new:N \g__UWMad_ThesisInfo_DefenseDate_tl
\tl_new:N \g__UWMad_ThesisInfo_Department_tl
\tl_new:N \g__UWMad_ThesisInfo_Program_tl
\tl_new:N \g__UWMad_ThesisInfo_Degree_tl
\tl_new:N \g__UWMad_ThesisInfo_AdvisorName_tl
\tl_new:N \g__UWMad_ThesisInfo_AdvisorPosition_tl
\tl_new:N \g__UWMad_ThesisInfo_AdvisorAssociation_tl
\tl_new:N \g__UWMad_ThesisInfo_AdvisorMarker_tl
\tl_new:N \g__UWMad_ThesisInfo_Institution_tl
%
%
\bool_new:N \g__UWMad_ThesisInfo_IsSet_Title_bool
\bool_new:N \g__UWMad_ThesisInfo_IsSet_Author_bool
\bool_new:N \g__UWMad_ThesisInfo_IsSet_DefenseDate_bool
\bool_new:N \g__UWMad_ThesisInfo_IsSet_Program_bool
\bool_new:N \g__UWMad_ThesisInfo_IsSet_Degree_bool
\bool_new:N \g__UWMad_ThesisInfo_IsSet_Institution_bool
\bool_new:N \g__UWMad_ThesisInfo_IsSet_Advisor_bool
%
%
%
%   User front-ends (Required)
\DeclareDocumentCommand \Title { m } {
    \tl_gset:Nn \g__UWMad_ThesisInfo_Title_tl {#1}
    \title{#1}
    \UWMad_MetaData_PushToList:nn{pdftitle}    {#1}
    \bool_if:NTF \g__UWMad_MetaData_IsDocument_bool {
        \bool_gset_true:N \g__UWMad_MetaData_GenerateAux_bool
    } { }
    \bool_gset_true:N \g__UWMad_ThesisInfo_IsSet_Title_bool
}
\DeclareDocumentCommand \Author { m } {
    \tl_gset:Nn \g__UWMad_ThesisInfo_Author_tl {#1}
    \author{#1}
    \UWMad_MetaData_PushToList:nn{pdfauthor}   {#1}
    \bool_if:NTF \g__UWMad_MetaData_IsDocument_bool {
        \bool_gset_true:N \g__UWMad_MetaData_GenerateAux_bool
    } { }
    \bool_gset_true:N \g__UWMad_ThesisInfo_IsSet_Author_bool
}
\DeclareDocumentCommand \Program { m } {
    \tl_gset:Nn \g__UWMad_ThesisInfo_Program_tl {#1}
    \bool_gset_true:N \g__UWMad_ThesisInfo_IsSet_Program_bool
}
\DeclareDocumentCommand \Degree { m } {
    \tl_gset:Nn \g__UWMad_ThesisInfo_Degree_tl {#1}
    \bool_gset_true:N \g__UWMad_ThesisInfo_IsSet_Degree_bool
}
\DeclareDocumentCommand \DefenseDate { m } {
    \tl_gset:Nn \g__UWMad_ThesisInfo_DefenseDate_tl {#1}
    \bool_gset_true:N \g__UWMad_ThesisInfo_IsSet_DefenseDate_bool
}
\cs_gset_eq:NN \DefenceDate \DefenseDate
\cs_gset_eq:NN \Date        \DefenseDate
\DeclareDocumentCommand \Institution { m } {
    \tl_gset:Nn       \g__UWMad_ThesisInfo_Institution_tl {#1}
    \bool_gset_true:N \g__UWMad_ThesisInfo_IsSet_Institution_bool
}
\cs_set_eq:NN \University \Institution
%
%
%   User front-ends (Optional)
\DeclareDocumentCommand \Department { m } {
    \tl_gset:Nn \g__UWMad_ThesisInfo_Department_tl {#1}
}
\DeclareDocumentCommand \Advisor { m m m } {
    \bool_gset_true:N \g__UWMad_ThesisInfo_IsSet_Advisor_bool
    \tl_gset:Nn \g__UWMad_ThesisInfo_AdvisorName_tl        {#1}
    \tl_gset:Nn \g__UWMad_ThesisInfo_AdvisorPosition_tl    {#2}
    \tl_gset:Nn \g__UWMad_ThesisInfo_AdvisorAssociation_tl {#3}
    \tl_gset:Nn \g__UWMad_ThesisInfo_AdvisorMarker_tl      {Advisor}
 }
\DeclareDocumentCommand \Adviser { m m m } {
    \bool_gset_true:N \g__UWMad_ThesisInfo_IsSet_Advisor_bool
    \tl_gset:Nn \g__UWMad_ThesisInfo_AdvisorName_tl        {#1}
    \tl_gset:Nn \g__UWMad_ThesisInfo_AdvisorPosition_tl    {#2}
    \tl_gset:Nn \g__UWMad_ThesisInfo_AdvisorAssociation_tl {#3}
    \tl_gset:Nn \g__UWMad_ThesisInfo_AdvisorMarker_tl      {Adviser}
}
%
%
%
%
%   User front-end accessors
\DeclareDocumentCommand \TheTitle { } {
    \g__UWMad_ThesisInfo_Title_tl
}
\DeclareDocumentCommand \TheAuthor { } {
    \g__UWMad_ThesisInfo_Author_tl
}
\DeclareDocumentCommand \TheProgram { m } {
    \g__UWMad_ThesisInfo_Program_tl
}
\DeclareDocumentCommand \TheDegree { m } {
    \g__UWMad_ThesisInfo_Degree_tl
}
\DeclareDocumentCommand \TheDefenseDate { m } {
    \g__UWMad_ThesisInfo_DefenseDate_tl
}
\cs_gset_eq:NN \TheDefenceDate \TheDefenseDate
\cs_gset_eq:NN \TheDate        \TheDefenseDate
\DeclareDocumentCommand \TheInstitution { m } {
    \g__UWMad_ThesisInfo_Institution_tl
}
\cs_set_eq:NN \TheUniversity \TheInstitution
%
\DeclareDocumentCommand \TheDepartment { m } {
    \g__UWMad_ThesisInfo_Department_tl
}
\DeclareDocumentCommand \TheAdvisor { m } {
    \g__UWMad_ThesisInfo_AdvisorName_tl
}
%
%
%
\int_new:N \l__UWMad_ThesisInfo_CommitteeCount_int
\UWMad_ListOf_Define:n {CommitteeList}
\DeclareDocumentCommand \CommitteeMember { m m m } {
    \int_incr:N \l__UWMad_ThesisInfo_CommitteeCount_int
    \UWMad_ListOf_PushEntry:nn {CommitteeList} {
        #1,~#2,~#3\skip_vertical:n{-1em}
    }
}
\DeclareDocumentCommand \PrintCommitteeMemberList { } {
    \bool_if:NTF \g__UWMad_ThesisInfo_IsSet_Advisor_bool {
        \g__UWMad_ThesisInfo_AdvisorName_tl{},~
        \g__UWMad_ThesisInfo_AdvisorPosition_tl{},~
        \g__UWMad_ThesisInfo_AdvisorAssociation_tl{}~
        (\g__UWMad_ThesisInfo_AdvisorMarker_tl{})
        \skip_vertical:n{-1em}
    } { }
    \UWMad_ListOf_PrintEntries:n {CommitteeList}
}
%
%
%
%
%
\tl_new:N \g__UWMad_PDFMetaData_Subject_tl
\tl_new:N \g__UWMad_PDFMetaData_Keywords_tl
\tl_new:N \g__UWMad_PDFMetaData_Producer_tl
\tl_new:N \g__UWMad_PDFMetaData_Creator_tl
%
%
%
%   User front-end (Optional)
\DeclareDocumentCommand \Subject { m } {
    \tl_gset:Nn \g__UWMad_PDFMetaData_Subject_tl {#1}
    \UWMad_MetaData_PushToList:nn{pdfsubject}  {#1}
    \bool_if:NTF \g__UWMad_MetaData_IsDocument_bool {
        \bool_gset_true:N \g__UWMad_MetaData_GenerateAux_bool
    } { }
}
\DeclareDocumentCommand \Keywords { m } {
    \tl_gset:Nn \g__UWMad_PDFMetaData_Keywords_tl {#1}
    \UWMad_MetaData_PushToList:nn{pdfproducer} {#1}
    \bool_if:NTF \g__UWMad_MetaData_IsDocument_bool {
        \bool_gset_true:N \g__UWMad_MetaData_GenerateAux_bool
    } { }
}
\DeclareDocumentCommand \Producer { m } {
    \tl_gset:Nn \g__UWMad_PDFMetaData_Producer_tl {#1}
    \UWMad_MetaData_PushToList:nn{pdfcreator}  {#1}
    \bool_if:NTF \g__UWMad_MetaData_IsDocument_bool {
        \bool_gset_true:N \g__UWMad_MetaData_GenerateAux_bool
    } { }
}
\DeclareDocumentCommand \Creator { m } {
    \tl_gset:Nn \g__UWMad_PDFMetaData_Creator_tl {#1}
    \UWMad_MetaData_PushToList:nn{pdfkeywords} {#1}
    \bool_if:NTF \g__UWMad_MetaData_IsDocument_bool {
        \bool_gset_true:N \g__UWMad_MetaData_GenerateAux_bool
    } { }
}
%
%
%
%   User front-end accessors.
\DeclareDocumentCommand \TheSubject { } {
    \g__UWMad_PDFMetaData_Subject_tl
}
\DeclareDocumentCommand \TheKeywords { } {
    \g__UWMad_PDFMetaData_Keywords_tl
}
\DeclareDocumentCommand \TheProducer { } {
    \g__UWMad_PDFMetaData_Producer_tl
}
\DeclareDocumentCommand \TheCreator { } {
    \g__UWMad_PDFMetaData_Creator_tl
}
%
%    \end{macrocode}
%
%
%
%
%   \iffalse
%</Code>
%   \fi
%   \iffalse
%<*Code>
%   \fi
%
%
%^^A ====================================================================== %
%^^A                            Title Page                                  %
%^^A ====================================================================== %
%
%   \UWModule{Special Pages}
%
%   \UWSubModule{MakeTitlePage}
%
%    \begin{macrocode}
% That phrase that occurs on every title page design the class author has seen
\DeclareDocumentCommand \FulfillmentClause { } {
    {
    \setstretch{1.1}
    A~\TheDocument{}~submitted~in~partial~fulfillment~of~the~
    requirements~for~the~degree~of
    }
}

\DeclareDocumentCommand \TitlePageTitle { } {
    {
        \huge
        \textsc {\TheTitle{}}
    }
}

\DeclareDocumentCommand \TitlePageAuthor { } {
    {
        \large
        by  \\[0.50em]
        \TheAuthor{}
    }
}

\DeclareDocumentCommand \TitlePageFulFillment { } {
    \FulfillmentClause{}
}

\DeclareDocumentCommand \TitlePageDegree { } {
    \TheDegree{}
}

\DeclareDocumentCommand \TitlePageProgram { } {
    \TheProgram{}
}

\DeclareDocumentCommand \TitlePageInstitution { } {
    \textsc{\TheInstitution{}}
}


\DeclareDocumentCommand \MakeTitlePage { } {
    \thispagestyle{empty}
    \begin{center}
        \TitlePageTitle{}       \\[1.0em]
        \TitlePageAuthor{}      \\[1.0em]
        \TitlePageFulFillment{} \\[1.0em]
        \TitlePageDegree{}      \\[1.0em]
        \TitlePageProgram{}     \\[1.0em]
        \vfill
        \TitlePageInstitution{}
    \end{center}
    \clearpage
}

\show\title








%^^A ====================================================================== %
%^^A                            License Page                                %
%^^A ====================================================================== %
%
% Full Copyright
\newcommand{\AllRightsReserved}{
    \large
    \copyright{} Copyright by \TheLicenseAuthor(\the\year)\\
    All Rights Reserved
}
\cs_gset_eq:NN \Copyright \AllRightsReserved
%
%
%
%
%
%^^A ====================================================================== %
%^^A                       Creative Commons Licenses                        %
%^^A ====================================================================== %
%
%   Token lists
\tl_new:N    \g__UWMad_CCLicense_Porting_tl
\tl_new:N    \g__UWMad_CCLicense_Version_tl
\tl_new:N    \g__UWMad_CCLicense_TypeAbbreviation_tl
\tl_new:N    \g__UWMad_CCLicense_TypeWords_tl
\tl_new:N    \g__UWMad_CCLicense_URL_Front_tl
\tl_new:N    \g__UWMad_CCLicense_URL_Middle_tl
\tl_new:N    \g__UWMad_CCLicense_URL_Back_tl
\tl_new:N    \g__UWMad_CCLicense_URL_tl
\tl_const:Nn \g__UWMad_CCLicense_Valid_Prefix {__UWMad_CCLicense_Valid_}
%
%   Booleans
\bool_new:N \g__UWMad_CCLicense_UseCreativeCommons_bool
\bool_new:N \g__UWMad_CCLicense_UseAttribution_bool
\bool_new:N \g__UWMad_CCLicense_UseShareAlike_bool
\bool_new:N \g__UWMad_CCLicense_UseNoDerivatives_bool
\bool_new:N \g__UWMad_CCLicense_UseNonCommercial_bool
\bool_new:N \g__UWMad_CCLicense_IsValid_bool
\bool_gset_true:N \g__UWMad_CCLicense_UseAttribution_bool
%
%   Valid license types
\cs_new:cn {\g__UWMad_CCLicense_Valid_Prefix by :}      {}
\cs_new:cn {\g__UWMad_CCLicense_Valid_Prefix by-sa :}   {}
\cs_new:cn {\g__UWMad_CCLicense_Valid_Prefix by-nd :}   {}
\cs_new:cn {\g__UWMad_CCLicense_Valid_Prefix by-nc :}   {}
\cs_new:cn {\g__UWMad_CCLicense_Valid_Prefix by-nc-sa :}{}
\cs_new:cn {\g__UWMad_CCLicense_Valid_Prefix by-nc-nd :}{}
%
%   Defaults
\tl_gset:Nn \g__UWMad_CCLicense_Porting_tl {
    International
}
\tl_gset:Nn \g__UWMad_CCLicense_Version_tl {
    4.0
}
%
%   URL definitions
\tl_gset:Nn \g__UWMad_CCLicense_URL_Front_tl {
    creativecommons.org/licenses/
}
\tl_gset:Nn \g__UWMad_CCLicense_URL_Middle_tl {
    \g__UWMad_CCLicense_TypeAbbreviation_tl/
}
\tl_gset:Nn \g__UWMad_CCLicense_URL_Back_tl {
    \g__UWMad_CCLicense_Version_tl
}
\tl_gset:Nn \g__UWMad_CCLicense_URL_tl {
    http://
    \g__UWMad_CCLicense_URL_Front_tl
    \g__UWMad_CCLicense_URL_Middle_tl
    \g__UWMad_CCLicense_URL_Back_tl
}
%
%
%
%   Type Creator
\cs_new:Nn \__UWMad_CCLicense_CreateType: {

        \bool_if:NTF \g__UWMad_CCLicense_UseAttribution_bool {

            \tl_put_right:Nn \g__UWMad_CCLicense_TypeAbbreviation_tl {
                by
            }
            \tl_put_right:Nn \g__UWMad_CCLicense_TypeWords_tl {
                Attribution
            }

        } { }

        \bool_if:NTF \g__UWMad_CCLicense_UseNonCommercial_bool {

            \tl_put_right:Nn \g__UWMad_CCLicense_TypeAbbreviation_tl {
                -nc
            }
            \tl_put_right:Nn \g__UWMad_CCLicense_TypeWords_tl {
                -NonCommercial
            }

        } { }

        \bool_if:NTF \g__UWMad_CCLicense_UseShareAlike_bool {

            \tl_put_right:Nn \g__UWMad_CCLicense_TypeAbbreviation_tl {
                -sa
            }
            \tl_put_right:Nn \g__UWMad_CCLicense_TypeWords_tl {
                -ShareAlike
            }

        } { }

        \bool_if:NTF \g__UWMad_CCLicense_UseNoDerivatives_bool {

            \tl_put_right:Nn \g__UWMad_CCLicense_TypeAbbreviation_tl {
                -nd
            }
            \tl_put_right:Nn \g__UWMad_CCLicense_TypeWords_tl {
                -NoDerivatives
            }

        } { }
}
%
%
%
%   Type Validator
\cs_new:Nn \__UWMad_CCLicense_CheckTypeValidity: {
    \cs_if_exist:cTF {
        \g__UWMad_CCLicense_Valid_Prefix
        \g__UWMad_CCLicense_TypeAbbreviation_tl :
    } {

        \bool_gset_true:N \g__UWMad_CCLicense_IsValid_bool

    } {

        \msg_new:nnn {UWMadThesis} {CCLicense / InvalidLicenseType} {
            The~license~type~`\g__UWMad_CCLicense_TypeAbbreviation_tl'~
            is~not~a~valid~Creative~Commons~license.
        }
        \msg_error:nn {UWMadThesis} {CCLicense / InvalidLicenseType}

    }
}
%
%
%
%   Page Printer
\cs_new:Nn \__UWMad_CCLicense_PrintPage: {

    {
        \clearpage
        \thispagestyle{empty}
        \null
        \tex_vfill:D
        \phantomsection
        \addcontentsline {toc} {chapter} {License}
        \begin{center}
            \Large
            \iffalse
            This~work~is~released~under~a~
            Creative~Commons~
            \g__UWMad_CCLicense_TypeWords_tl{}~
            \g__UWMad_CCLicense_Version_tl{}~
            \g__UWMad_CCLicense_Porting_tl{}~
            license.\\[0.1em]

            (\g__UWMad_CCLicense_URL_tl{})
            \fi

            This~work~is~released~under~a~
            \href {\g__UWMad_CCLicense_URL_tl} {
                Creative~Commons~
                \g__UWMad_CCLicense_TypeWords_tl{}~
                \g__UWMad_CCLicense_Version_tl{}~
                \g__UWMad_CCLicense_Porting_tl{}
            }~
            license.\\[0.1em]

            (\g__UWMad_CCLicense_URL_tl{}/)
        \end{center}
    }
}
%
%
%   User front ends
\DeclareDocumentCommand \CreativeCommons { } {
    \bool_gset_true:N \g__UWMad_CCLicense_UseCreativeCommons_bool
}
\DeclareDocumentCommand \NonCommercial { } {
    \bool_gset_true:N \g__UWMad_CCLicense_UseNonCommercial_bool
}
\DeclareDocumentCommand \ShareAlike { } {
    \bool_gset_true:N \g__UWMad_CCLicense_UseShareAlike_bool
}
\DeclareDocumentCommand \NoDerivatives { } {
    \bool_gset_true:N \g__UWMad_CCLicense_UseNoDerivatives_bool
}
%
%
%
%
%
%^^A ====================================================================== %
%^^A                        License Page environment                        %
%^^A ====================================================================== %
\DeclareDocumentEnvironment {LicensePage} { } { } {

    \bool_if:NTF \g__UWMad_CCLicense_UseCreativeCommons_bool {
        \__UWMad_CCLicense_CreateType:
        \__UWMad_CCLicense_CheckTypeValidity:
        \bool_if:NTF \g__UWMad_CCLicense_IsValid_bool {
            \__UWMad_CCLicense_PrintPage:
        } { }
    } { }

}

%    \end{macrocode}
%
%
%
%   \iffalse
%</Code>
%   \fi
%    \begin{macrocode}
\ExplSyntaxOff
%    \end{macrocode}
%
%   \Finale
\endinput
