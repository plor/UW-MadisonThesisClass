\chapter{Programming Module Tests}

\section{Testing Level-Section Commands}
    \subsection{\Cmd{LevelToSection} Command Test}
    \begin{tabular}{lll}
        Passing level -1 & returns `\LevelToSection{-1}'&  should be `part'          \\
        Passing level  0 & returns `\LevelToSection{0}' &  should be `chapter'       \\
        Passing level  1 & returns `\LevelToSection{1}' &  should be `section'       \\
        Passing level  2 & returns `\LevelToSection{2}' &  should be `subsection'    \\
        Passing level  3 & returns `\LevelToSection{3}' &  should be `subsubsection' \\
        Passing level  4 & returns `\LevelToSection{4}' &  should be `paragraph'     \\
        Passing level  5 & returns `\LevelToSection{5}' &  should be `subparagraph'
    \end{tabular}

    \subsection{\Cmd{SectionToLevel} Command Test}
    \begin{tabular}{lll}
        Passing section part            & returns `\SectionToLevel{part}'           &  should be `-1'   \\
        Passing section chapter         & returns `\SectionToLevel{chapter}'        &  should be `0'    \\
        Passing section section         & returns `\SectionToLevel{section}'        &  should be `1'    \\
        Passing section subsection      & returns `\SectionToLevel{subsection}'     &  should be `2'    \\
        Passing section subsubsection   & returns `\SectionToLevel{subsubsection}'  &  should be `3'    \\
        Passing section paragraph       & returns `\SectionToLevel{paragraph}'      &  should be `4'    \\
        Passing section subparagraph    & returns `\SectionToLevel{subparagraph}'   &  should be `5'
    \end{tabular}

\section{Counter System Tests}
{
\setstretch{1.3}
    \DefineNewLocalCounter{NewLocalCounter}{1}
    \DefineNewGlobalCounter{NewGlobalCounter}{10}
    The local  counter has been created and has a value of \CounterValue{NewLocalCounter}.\\
    The global counter has been created and has a value of \CounterValue{NewGlobalCounter}.

    {
    \StepCounter{NewLocalCounter}
    \StepCounter{NewGlobalCounter}
    This is a new scope, and both counters  have been stepped.

    The local  counter value is \CounterValue{NewLocalCounter}.\\
    The global counter value is \CounterValue{NewGlobalCounter}.
    }

    Back to the main scope.

    The local  counter has a value of \CounterValue{NewLocalCounter}.\\
    The global counter has a value of \CounterValue{NewGlobalCounter}.
    
    {
    \AddToCounter{NewLocalCounter}{5}
    \AddToCounter{NewGlobalCounter}{5}
    This is a new scope, and both counters have been incremented by 5.

    The local  counter has a value of \CounterValue{NewLocalCounter}.\\
    The global counter has a value of \CounterValue{NewGlobalCounter}.
    }

    Back to the main scope.

    The local  counter has a value of \CounterValue{NewLocalCounter}.\\
    The global counter has a value of \CounterValue{NewGlobalCounter}.

    {
    \SetAndAddToCounter{NewLocalCounter} {5}{5}
    \SetAndAddToCounter{NewGlobalCounter}{5}{5}
    This is a new scope, and both counters have been set to 5 and incremented by 5.

    The local  counter has a value of \CounterValue{NewLocalCounter}.\\
    The global counter has a value of \CounterValue{NewGlobalCounter}.
    }

    Back to the main scope.

    The local  counter has a value of \CounterValue{NewLocalCounter}.\\
    The global counter has a value of \CounterValue{NewGlobalCounter}.
}


\section{Nested loops}
{\singlespacing

    \subsection{\Cmd{For} loops}
        \For[1]{1}{3}{
            Outer Loop: \the\ForLoopCounter
            
            {
            \For[1]{1}{3}{
                \hspace{3em}Inner Loop: \the\ForLoopCounter
                
            }
            }
        }
        
    \subsection{\Cmd{ForEach} loops}
        
        \subsubsection{Array Test}
            \ArrayMake{Array}
            \ArrayPush{Array}{1}
            \ArrayPush{Array}{2}
            \ArrayPush{Array}{3}
            \ForEach[1]{Array}{
                Outer Loop: \ArrayGet{Array}{ForLoopCounter}
                
                {
                    \ForEach[1]{Array}{
                        \hspace{3em}Inner Loop: \ArrayGet{Array}{ForLoopCounter}
                        
                    }
                }
            }
}


\section{\Cmd{GlobalNewIf}}
\newif\ifLocalSwitch\LocalSwitchtrue
\GlobalNewIf{GlobalSwitch}\GlobalSwitchtrue
Both |LocalSwitch| and |GlobalSwitch| have been created.
\begin{itemize}
    \item |LocalSwitch| is currently set to \ifLocalSwitch true\else false\fi.
    \item |GlobalSwitch| is currently set to \ifGlobalSwitch true\else false\fi.
\end{itemize}

\begingroup
\singlespacing
\LocalSwitchfalse
\GlobalSwitchfalse
\textbf{This is a new scope and both switches have been flipped to |false|:}
\begin{itemize}
    \item |LocalSwitch| is currently set to \ifLocalSwitch true\else false\fi.
    \item |GlobalSwitch| is currently set to \ifGlobalSwitch true\else false\fi.
\end{itemize}
\endgroup

\textbf{Back to the main scope:}
\begin{itemize}
    \item |LocalSwitch| is currently set to \ifLocalSwitch true\else false\fi.
    \item |GlobalSwitch| is currently set to \ifGlobalSwitch true\else false\fi.
\end{itemize}

\textbf{TEST:} The |GlobalSwitch|'s state should retain its value after exiting back to the main scope while |LocalSwitch| should not.


