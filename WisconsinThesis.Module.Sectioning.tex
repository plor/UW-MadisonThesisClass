% =============================================================================================== %
%                        WisconsinThesis Class: Sectioning Module                                 %
% =============================================================================================== %


% =========================================================================== %
%                        Redefined Chapter Head Command                       %
% =========================================================================== %
%
%   LaTeX's standard Report Class is used as a base; however, the chapter heading
%   customization leaves something to be desired.  So, in this section,
%   \@makechapterhead is redefined with defined lengths and styles available to the 
%   users to redfine at their own risk.
%
%   Terms for this section:
%       Chapter ID:     A combination of text that identifies what follows is a numbered chapter (e.g. "Chapter 1")
%       Chapter Title:  Text supplied by the user that names the numbered chapter (e.g., "Differential Calculus")
%       Chapter Head:   A styled combination of a Chapter ID and a Chapter Title at the beginning of a chapter.
%
%   Lengths:
%       \ChapterHeadVSpaceStart:    Vertical space between top of set-able area (i.e., a page) to the start of chapter head.
%       \ChapterHeadIDTitleSkip:    Vertical space between the Chapter ID and Chapter Title
%       \ChapterHeadVSpaceEnd:      Vertical space between end of chapter head and the chapter text.
%       \ChapterHeadParIndent:      Horizontal indentation of chapter ID and chapter Title
%
%

% Lengths for spacing the chapter head as desired.
\DefineNewLength{\ChapterHeadVSpaceStart}  { 40pt}         % Set: Distance between top of text body and top of Chapter Head
\DefineNewLength{\ChapterHeadVSpaceEnd}    { 20pt}         % Set: Distance between bottom of Chapter Head and top of chapter content
\DefineNewLength{\ChapterHeadParIndent}    {  0pt}         % Set: Indentation of Chapter Head
\DefineNewLength{\ChapterHeadIDTitleSkip}  {-1.1\parskip}  % Set: Distance between Chapter ID and Chapter Title


% Styles for the chapter head
\newcommand{\ChapterHeadJustification}  {\raggedright}
\newcommand{\ChapterHeadFont}           {\normalfont}
\newcommand{\ChapterHeadID}             {\@chapapp\space\thechapter}
\newcommand{\ChapterHeadStyleID}   [1]  {\large{\textbf{#1}}}
\newcommand{\ChapterHeadStyleTitle}[1]  {\LARGE{\textbf{#1}}}

% Replacement command for \@makechapterhead
\DeclareRobustCommand{\ChapterHead@MakeNormal}[1]{
    {   
        \setlength{\topskip}{\ChapterHeadVSpaceStart}   %  Set:  Vertical whitespace between the top of the page and the Chapter Head
        \parindent \ChapterHeadParIndent                %  Set:  Chapter Head indentation from the left margin
                   \ChapterHeadJustification            %  Set:  Chapter Head justification
                   \ChapterHeadFont                     %  Set:  Chapter Head font
        \ifnum \c@secnumdepth > \m@ne                   %  Test: Is the SectionDepth counter is above -1
            \ChapterHeadStyleID{\ChapterHeadID}         %  True: Set: the Chapter ID with the above style
            \par\nobreak                                %        Set: End the paragraph and forbid breaking before the next paragraph
            \vskip \ChapterHeadIDTitleSkip              %        Set: Adds vertical space between the Chapter ID and the Chapter Title
        \fi                                             %  
        \interlinepenalty\@M                            %  Define: A penalty declaration of 10,000 for page breaking before the Chapter Title
        \ChapterHeadStyleTitle{#1}                      %  Set:    Chapter Title with the above style
        \par\nobreak                                    %  Set:    End the paragraph and forbid breaking before the next paragraph
        \addtolength{\ChapterHeadVSpaceEnd}{-\parskip}  %  Set:    Temporary correction to the spacing to account for a large \parskip
        \vskip \ChapterHeadVSpaceEnd                    %  Set:    Add vertical space between the Chapter Title and the following set text
    }
}


% Replacement command for \@makeschapterhead (the starred chapter chapter head)
\DeclareRobustCommand{\ChapterHead@MakeStarred}[1]{
    {   \setlength{\topskip}{\ChapterHeadVSpaceStart}   %  Adds vertical whitespace between the top of the page and the Chapter Head
        \parindent \ChapterHeadParIndent                %  Defines the Chapter Head indentation from the left margin
                   \ChapterHeadJustification            %  Defines the Chapter Head justification
                   \ChapterHeadFont                     %  Defines the Chapter Head font
        \interlinepenalty\@M                            %  A penalty declaration of 10,000 for page breaking the Chapter ID and Chapter Title
        \ChapterHeadStyleTitle{#1}                      %  Sets the Chapter Title with the above style
        \par\nobreak                                    %  Ends the paragraph (\par) and forbids putting a break before the next paragraph (\nobreak)
        \addtolength{\ChapterHeadVSpaceEnd}{-\parskip}  %  A local temporary correction to the spacing to account for a large \parskip
        \vskip \ChapterHeadVSpaceEnd                    %  Adds vertical space (technically, glue) between the Chapter Title and the following set text
    }
}


% Overwrite the default commands
\let\@makechapterhead\ChapterHead@MakeNormal
\let\@makeschapterhead\ChapterHead@MakeStarred





% =========================================================================== %
%                    Redefinition of Chapter Commands                         %
% =========================================================================== %
%
%   The \chapter command has also been redefined to use the \thispagestyle{myheadings}
%   command to be in compliance with the page number thesis guidelines of UW-Madison.
%
%   The \@chapter command (which is called for unstarred \chapter{} uses) is also redfined
%   such that upon the first unstarred use of \chapter, the page numbering is switched to arabic.
%
%   A fair amount of the macro code is taken directly from the original \defs of \chapter 
%   and \@chapter in report.cls ("2007/10/19 v1.4h Standard LaTeX document class") with additional 
%   comments.
%

\DeclareRobustCommand{\chapter}{
    \if@openright\cleardoublepage\else\clearpage\fi %  Double-skips if the 'openright' option is used.
    \global\@topnum\z@                              %  Prevents figures from being place at the top of the page
    \thispagestyle{myheadings}                      %  Places the page number in the upper-right corner
    \@afterindentfalse                              %  Turns off indentation of the first paragraph following the chapter head.
    \secdef\@chapter\@schapter                      %  Runs commands for the unstarred (\@chapter) and starred (\@schapter) uses of \chapter
}


\def\@chapter[#1]#2{

    \ifnum \value{chapter}=0                       % Test:  Is this the first chapter?
        \pagenumbering{arabic}                     % True:  Switch to arabic page numbers 
    \fi

    \ifnum \c@secnumdepth >\m@ne                   % Test: Is the SectionDepth counter is above -1
        \refstepcounter{chapter}                   % True: Define: Add one to chapter counter
        \typeout{\@chapapp\space\thechapter.}      %       Type:   Chapter ID
        \addcontentsline{toc}{chapter}             %       Write:  Add a contentsline line to the ToC file at the chapter level
        {\protect\numberline{\thechapter}#1}       %               Write out chapter number and title
    \else                                          % False:
        \addcontentsline{toc}{chapter}{#1}         %       Write:  Add a contentsline line with Chapter Title only
    \fi
    \chaptermark{#1}                               % Set: For \pagestyle{myheadings} this just gobbles the argument
    \addtocontents{lof}{\protect\addvspace{10\p@}} % Write: To LoF
    \addtocontents{lot}{\protect\addvspace{10\p@}} % Write: To LoT
    \if@twocolumn
        \@topnewpage[\@makechapterhead{#2}]        % Set: The Chapter Head on a new page
    \else
        \@makechapterhead{#2}                      % Set: The Chapter Head
        \@afterheading                             % Set:
    \fi
}





% =========================================================================== %
%             Redefined Sectioning Commands (i.e., \parskip clean-up)         %
% =========================================================================== %
%
%   The default commands for sectioning do not account for a large parskip (i.e., if parskip
%   is set large, the spacing between a section's Head and the first pargraph will be large).
%   The redefinitions below use the default sectioning commands and forcibly removes the 
%   added \parskip space.  Therefore, the vertical space between the section title and the 
%   first paragraph will remain constant.
%
%   \chapter is not included here because that is handled into the above redefinitions
%   of the \chapter and \@makechapterhead commands.
%

% Copy the default commands into aliases.
\let  \SectionDefault        \section
\let  \SubSectionDefault     \subsection
\let  \SubSubSectionDefault  \subsubsection
\let  \ParagraphDefault      \paragraph
\let  \SubParagraphDefault   \subparagraph



% Redefine \section
\DeclareRobustCommand{\section}[2][]{
    \ifthenelse{\equal{#1}{}}           % Test:   Is the first (optional) input empty?
        {\SectionDefault[#2]{#2}}       % True:   Use the non-optional
        {\SectionDefault[#1]{#2}}       % False:  Use the optional argument
    \vspace*{-\parskip}                 % Set:    Remove the whitespace added by the non-zero \parskip
}

% Redefine \subsection
\DeclareRobustCommand{\subsection}[2][]{
    \ifthenelse{\equal{#1}{}}           % Test:   Is the first (optional) input empty?
        {\SubSectionDefault[#2]{#2}}    % True:   Use the non-optional
        {\SubSectionDefault[#1]{#2}}    % False:  Use the optional argument
    \vspace*{-\parskip}                 % Set:    Remove the whitespace added by the non-zero \parskip
}

% Redefine \subsubsection
\DeclareRobustCommand{\subsubsection}[2][]{
    \ifthenelse{\equal{#1}{}}           % Test:   Is the first (optional) input empty?
        {\SubSubSectionDefault[#2]{#2}} % True:   Use the non-optional
        {\SubSubSectionDefault[#1]{#2}} % False:  Use the optional argument
    \vspace*{-\parskip}                 % Set:    Remove the whitespace added by the non-zero \parskip
}

% Redefine \paragraph
\DeclareRobustCommand{\paragraph}[2][]{
    \ifthenelse{\equal{#1}{}}           % Test:   Is the first (optional) input empty?
        {\ParagraphDefault[#2]{#2}}     % True:   Use the non-optional
        {\ParagraphDefault[#1]{#2}}     % False:  Use the optional argument
    \vspace*{-\parskip}                 % Set:    Remove the whitespace added by the non-zero \parskip
}

% Redefine \subparagraph
\DeclareRobustCommand{\subparagraph}[2][]{
    \ifthenelse{\equal{#1}{}}           % Test:   Is the first (optional) input empty?
        {\SubParagraphDefault[#2]{#2}}  % True:   Use the non-optional
        {\SubParagraphDefault[#1]{#2}}  % False:  Use the optional argument
    \vspace*{-\parskip}                 % Set:    Remove the whitespace added by the non-zero \parskip
}





% =========================================================================== %
%                          New Appendix Command                               %
% =========================================================================== %

% Appendix counter
\DefineNewCounter{appendix}    {0}  %  New appendix counter used in \Chapter@Appendix

%
%   This command initializes the \appendix commands (this was originally the 
%   \appendix command, but that will be replaced as a \chapter alias).
%
%   The following taken directly from the \appendix definition in report.cls 
%   ("2007/10/19 v1.4h Standard LaTeX document class") and nothing in the expansion
%   has changed.
%
\newcommand{\AppendixInitializer}{
    \par
    \setcounter{section}{0}            %
    \def\@chapapp{\appendixname}       %
    \def\thechapter{\Alph{appendix}}
}


%
%   This command redefines the \appendix to act as a chapter alias.
%
\renewcommand{\appendix}{
    \ifnum \value{appendix}=0
        \AppendixInitializer
    \fi
    
    \stepcounter{appendix}
    \chapter
}





% =========================================================================== %
%              Front Matter Environment/Command Definitions                   %
% =========================================================================== %
\DefineNewCounter{FrontMatterCount}{0}

\DeclareRobustCommand{\UWMad@FrontMatterRegister}[1]{
    \addcontentsline{toc}{chapter}{#1}
    \stepcounter{FrontMatterCount}
}

\ifUWMad@FrontMatterCommands
    % Preliminary page commands
        \DeclareRobustCommand{\dedications}    [1][Dedications]    {\chapter*{#1}\UWMad@FrontMatterRegister{#1}}{}
        \DeclareRobustCommand{\acknowledgments}[1][Acknowledgments]{\chapter*{#1}\UWMad@FrontMatterRegister{#1}}{}
        \DeclareRobustCommand{\abstract}       [1][Abstract]       {\chapter*{#1}\UWMad@FrontMatterRegister{#1}}{}
        \DeclareRobustCommand{\umiabstract}    [1][Abstract]       {\chapter*{#1}\UWMad@FrontMatterRegister{#1}}{}
        \DeclareRobustCommand{\preface}        [1][Preface]        {\chapter*{#1}\UWMad@FrontMatterRegister{#1}}{}
\else
    % Preliminary page environments
        \newenvironment  {dedications}    [1][Dedications]    {\chapter*{#1}\UWMad@FrontMatterRegister{#1}}{}
        \newenvironment  {acknowledgments}[1][Acknowledgments]{\chapter*{#1}\UWMad@FrontMatterRegister{#1}}{}
        \renewenvironment{abstract}       [1][Abstract]       {\chapter*{#1}\UWMad@FrontMatterRegister{#1}}{}
        \newenvironment  {umiabstract}    [1][Abstract]       {\chapter*{#1}\UWMad@FrontMatterRegister{#1}}{}
        \newenvironment  {preface}        [1][Preface]        {\chapter*{#1}\UWMad@FrontMatterRegister{#1}}{}
\fi





% =========================================================================== %
%                 List of Contents, Tables, and Figures                       %
% =========================================================================== %
% Register the Table of Contents to the Table of Contents
\renewcommand{\contentsname}{Table of Contents}
\let\TableOfContentsDefault\tableofcontents
\renewcommand{\tableofcontents}{
    {
        \setstretch{1.0}
        \phantomsection
        \UWMad@FrontMatterRegister{\contentsname}
        \TableOfContentsDefault
        %\UWMad@FrontMatterRegister{\contentsname}
        \clearpage
    }
}

% Register the List of Tables to the Table of Contents
\let\ListOfTablesDefault\listoftables
\renewcommand{\listoftables}{
    {
        \setstretch{1.0}
        \ListOfTablesDefault
        \UWMad@FrontMatterRegister{\listtablename}
        \clearpage
    }
}

% Register the List of Figures to the Table of Contents
\let\ListOfFiguresDefault\listoffigures
\renewcommand{\listoffigures}{
    {
        \setstretch{1.0}
        \ListOfFiguresDefault
        \UWMad@FrontMatterRegister{\listfigurename}
        \clearpage
    }
}





% =========================================================================== %
%                 Table of Contents 'Headers' (i.e., Parts)                   %
% =========================================================================== %
\newcommand{\AddToCHeaderStyle}[2][-0.35em]{
    \hspace{#1}#2\rule{0pt}{1em}
}

\DeclareRobustCommand{\AddToCHeader}[2][part]{
    \addtocontents{toc}{\protect\contentsline{#1}{\AddToCHeaderStyle{#2}}{}{}}
}



